\documentclass[12pt]{article}
\title{Proof Comprehension Exercises}

\begin{document}
\maketitle
\noindent
{\bf Question 1.} In this question we study an argument which claims to prove that the sequence $a_n = (-1)^n/n$ converges to zero as $n \to \infty$. We set the argument out in several steps. Let $\epsilon > 0$. \\

\medskip
(i) We calculate $|a_n - 0| = \left|\frac{(-1)^n}{n}\right| = \frac{1}{n}$.\\

(ii) We take $N$ to be any integer such that $N > {1/\epsilon}$, so that $1/N < \epsilon$. \\ 

(iii) Therefore $|a_N - 0| < \epsilon$.\\

(iv) Since $\epsilon > 0$ was arbitrary, and for each such $\epsilon$ we found an $N$ which works, we see that the definition of convergence is satisfied, and we conclude that $ a_n \to 0$ as $n \to \infty$.

\medskip
\begin{itemize}
\item [(a)] Is this argument correct?
\item [(b)] If it not correct, in which step(s) is the argument not correct?
\item [(c)] If it is not correct, how would you need to modify the it in order to make it correct?
\end{itemize}

\medskip
\noindent
{\bf Question 2.} In this question we study an argument which claims to prove that the sequence given by
$a_n = (-1)^n $
 converges to $1$ as $n \to \infty$. We set the argument out in several steps. Let $\epsilon > 0$. \\

\medskip
(i) We calculate $|a_n - 1| = 0$ when $n$ is even and $|a_n - 1| = 2$
when $n$ is odd.

\medskip
(ii) We take $N$ to be any even integer, so that $|a_N - 1| =0 <  \epsilon$.

\medskip
(iii) Since $\epsilon > 0$ was arbitrary, and for each such $\epsilon$ we found an $N$ (in fact any even $N$) which works, we see that the definition of convergence is satisfied, and we conclude that $ a_n \to 0$ as $n \to \infty$.

\begin{itemize}
\item[(a)] Is this argument correct?
\item[(b)] If it not correct, in which step(s) is the argument not correct?
\item[(c)] If it is not correct,  could you modify it in order to make it correct?
\item[(d)] If it is not correct, and you think it can be modified to be correct, give the modifed argument. If you think it cannot be modified to be correct, explain why not.

\end{itemize}

\medskip
\noindent
{\bf Question 3.} We claim that if $A$ and $B$ are nonempty subsets of the real numbers such that for all $a \in A$ and $b \in B$ we have $a \leq b$, then there is some real number $\xi$ such that
for all $a \in A$ and $b \in B$ we have
\[ a \leq \xi \leq b.\]
We say that such a $\xi$ {\bf separates}
$A$ and $B$.

\medskip
(i) Give an example of a pair of sets $A$ and $B$ satisfying the hypothesis of this claim. 

\medskip
(ii) Is the conclusion of the claim necessarily valid if either $A$ or $B$ is empty?

\medskip
(iii) Is the conclusion of the claim valid if the hypothesis that ``for all $a \in A$ and $b \in B$ we have $a \leq b$" is violated?

\medskip
(iv) It can be shown that $\xi = {\rm LUB}(A)$ separates $A$ and $B$. Explain why ${\rm LUB}(A)$ exists, and why this choice works. 

\medskip
(v) Describe the complete set of real numbers $\xi$ which separate $A$ and $B$. Give an example of a pair of sets $A$ and $B$ (satisfying the hypotheses of the claim) for which only the real number $0$ separates them.

\medskip
\noindent
{\bf Question 4.} This question explores something which seems completely obvious at first sight: that every whole number is either even or odd, and not both. On a second look it's perhaps not so obvious though. Let $n \in N$. We say that $n$ is even if there is a $k \in N$ such that $n = 2k$. We say that $n$ is odd if there is an $l \in N$ such that $n = 2l+1$.

\medskip
(i) Prove that there are no $n \in N$ which are both even and odd. Suppose for a contradiction that $n = 2k = 2l+1$ where $k, l \in N$. Then $k - l =1/2$. Since $1/2 > 0$ we have $k > l$, and therefore $k-l = 1/2$ is a member of $N$. Why is this last statement false, hence leading to our contradiction?

\medskip
(ii) For $n \in N$ let $P(n)$ be the statement ``$n$ is either even or odd". Prove by induction that $P(n)$ holds for all $n \in N$. 

\medskip
\noindent
From this we deduce that indeed every natural number is either even or odd, (and not both!), just as we would have expected all along.

\bigskip
{\bf Question 5 on derivation Archimedean property from the completeness axiom -- see Chris's Stack pages}

\bigskip
{\bf Question 6 -- Squeeze theorem} Suppose that $x_n \leq y_n \leq z_n$ for all $n$ and that $x_n \to L$ and $z_n \to L$. Show that $y_n \to L$. Establish this by filling in the gaps below.

{\bf Solution.} Let $\epsilon > 0$. Since $z_n \to L$, there exists ?? such that ?? $>$ ?? implies $ z_n < L + ??$. Since $x_n \to L$, there exists ?? such that ?? $>$ ?? implies $L- ?? < x_n$. Therefore, for ?? $ >$ ??, 
\[ L - ?? < x_n \leq y_n \leq z_n < ?? + X\]
and so for all such ?? $>$ ?? we have
\[ |y_n - L| < ??.\]

\bigskip
{\bf Question 7.} In the lecture notes we saw how to give meaning to $k$'th roots of positive numbers when $k \in {N}$. Let $x >0$ and $q \in {Q}$ with $q= m/n$ where $m, n \in N$. (i) Propose a definition of $x^q$. (ii) Propose another definition of $x^q$. (iii) Are these definitions the same? (iv) If $q = m/n$, then $q = (km)/(kn) = m'/n'$ for all $k \in N$. So $q = m'/n'$ for some other natural numbers $m'= km$ and $n'= kn$. Does this affect the definition of $x^q$?

{\bf Solution.} (i) Define $x^q := \left(x^{1/n}\right)^m$ where $x^{1/n}$ is the $n$'th root of $x$. (ii) Define $x^q: =  \left(x^m\right)^{1/n}$. (iii) Yes, because if we raise both proposed definitions to the power $n$ we obtain on the one hand $\left[\left(x^{1/n}\right)^m\right]^n
= \left(x^{1/n}\right)^{mn} = \left(x^{1/n}\right)^{nm} = \left[\left(x^{1/n}\right)^n\right]^m = x^m$ by the defining property of $x^{1/n}$, and on the other hand 
$\left[\left(x^m\right)^{1/n}\right]^n = x^m$ by the defining property of $(x^m)^{1/n}$. Thus both proposed definitions of $x^q$ satisfy the defining property for the $n$'th root of $x^m$ 
and by the uniqueness of this number, they must coincide. (iv) No. We need to see that $\left(x^{1/n'}\right)^{m'} = \left(x^{1/n}\right)^{m}$. But 
$x^{1/n'} = x^{1/kn} = \left(x^{1/n}\right)^{1/k}$ so that 
$\left(x^{1/n'}\right)^{k} = x^{1/n}$.
Therefore 
\[ \left(x^{1/n'}\right)^{m'} = 
\left(x^{1/n'}\right)^{mk} = \left(x^{1/n}\right)^m\]
as required.

\bigskip
{\bf Question 8.} ``If $x_n \to x$ and $x >0$, then there is some $N$ such that for all $n > N$, we have $x_n > 0$." Is the following proof of this statement correct? If not, can you correct it?

{\em Proof.} Suppose the conclusion fails. Then for some $N$, we have that for all $n >N$, $x_n \leq 0$. Since $x_n \to x$ we may use Proposition 6.4 (b) to conclude that $x \leq 0$. This contradicts the hypothesis that $x>0$, and so we conclude that the conclusion must hold.

\medskip
{\bf Solution.} There are two dodgy steps. One is in the application of Proposition 6.4 (b). That result requires $x_n \leq 0$ for all $n$, and we only know it for $n > N$. This is easily fixable, since a quick examination of the proof of Proposition 6.4 (b) shows that we only need $x_n \leq 0$ for all sufficiently large $n$ in order for it to go through. The second dodgy step is the assertion that the negation of the statement ``there is some $N$ such that for all $n>N$ we have $x_n > 0$" is ``there is some $N$ such that for all $n >N$, we have $x_n \leq 0$". This is {\em not} the required negation. The correct negation is: ``
for all $N$ there exists an $n >N$ with $x_n \leq 0$". It is possible to revisit the proof of Proposition 6.4 (b) and to show that it can still be made to work under this new scenario, but it's easier to argue directly as follows. Take $\epsilon = x/2 > 0$ in the definition of convergence of $(x_n)$ to $x$. Then there is some $N$ such that for $n > N$ we have $|x_n - x| < x/2$, in particular $ x - x_n < x/2$, or $x_n > x/2 > 0$. So for all $n > N$ we have $x_n > 0$.

\bigskip
{\bf Question 9.}
In high school you may have met recursively defined sequences of the form $x_1 = c$, 
\[ x_{n+1} = a x_n +b \; \mbox{ for } \; n \geq 1\]
where $a, b, c \in \mathbb{R}$. Show that when $|a| < 1$, $(x_n)$ converges and find its limit. Show that when $|a|>1$, $(x_n)$ diverges. What happens when $a = \pm 1$? 
\\

We already handled all but the special cases $a = \pm 1$ in the lecture notes. To complete the study, fill in the gaps in the following arguments:\\

When $a = 1$, $x_n = ??$, (find a formula depending on $b, c$ and $n$) as can be easily established by induction, and this sequence diverges to $+\infty$ when $ ?? >0$ and to $- \infty$ when $?? < 0$, and when $??= 0$, $x_n = ??$ for all $n$. \\

When $a = -1$, recall that we set $y_n = x_n - \frac{b}{2}$ and observed that $y_{n+1} = (-1)^n y_n$. Thus $y_n$ alternates between the values $\pm ?? $ (and when $c= ??$ this means that $y_n = ??$ for all $n$, that is, $x_n = ?? $ for all $n$).\\

\medskip
In summary: when $a = 1$, $(x_n)$ diverges {\em unless} $??=0$, in which case $x_n = ??$ for all $n$; and when $a = -1$, $x_n$ alternates between the two values $\pm (c - \frac{b}{2}) + \frac{b}{2}$, and therefore $(x_n)$ diverges, {\em unless} $b = 2c$, in which case $x_n = c$ for all $n$.

\noindent
{\bf Solution.}
When $a = 1$, $x_n = c + (n-1)b$, as can be easily established by induction, and this sequence diverges to $+\infty$ when $ b >0$ and to $- \infty$ when $b < 0$, and when $b= 0$, $x_n = c$ for all $n$. \\

When $a = -1$, recall that we set $y_n = x_n - \frac{b}{2}$ and observed that $y_{n+1} = (-1)^n y_n$. Thus $y_n$ alternates between the values $\pm (c - \frac{b}{2})$ (and when $c= \frac{b}{2}$ this means that $y_n = 0$ for all $n$, that is, $x_n = c $ for all $n$).\\

\medskip
In summary: when $a = 1$, $(x_n)$ diverges {\em unless} $b=0$, in which case $x_n = c$ for all $n$; and when $a = -1$, $x_n$ alternates between the two values $\pm (c - \frac{b}{2}) + \frac{b}{2}$, and therefore $(x_n)$ diverges, {\em unless} $b = 2c$, in which case $x_n = c$ for all $n$.

\bigskip
{\bf Question 10.}
Let $a_n = \left(1 + \frac{1}{n}\right)^n$. (i) Show that $(a_n)$ is increasing. (ii) Show that $a_n \leq 4$ for all $n$. (iii) Deduce that $(a_n)$ converges to some number $L$ satisfying $2 < L \leq 4$.
(For this question you may wish to recall both Bernoulli's inequality and the binomial theorem.)

\bigskip
{\bf Solution.} ** Solution in Lecture Notes**

\bigskip
{\bf Question 11.}
Define a sequence iteratively by $x_1 = 1$ and $x_{n+1} = \frac{2}{x_n}$ for $n \geq 1$.

(i) Show by induction that $1 \leq x_n \leq 2$ for all $n$.

(ii) Let $L = \lim_{n \to \infty} x_n$. Show that $1 \leq L \leq 2$.

(iii) Use the rules for limits to show that $L^2 = 2$ and deduce that $L =\sqrt{2}$.

(iv) Is the argument correct? (Hint: calculate the first few terms $x_1, x_2, x_3, x_4, \dots$) If not, where is the error?

(v) Compare the intention of this question with Heron's method for approximating $\sqrt{2}$.

\medskip
{\bf Solution.}
(i) Since $x_1 = 1$ we trivially have $1 \leq x_1 \leq 2$ thus establishing the base case. Now suppose inductively that for some $n$ we have $1 \leq x_n \leq 2$. Then $1/2 \leq 1/x_n \leq 1$ and so $1 \leq x_{n+1} = 2/x_n \leq 2$, establishing the inductive step. Therefore indeed $1 \leq x_n \leq 2$ for all $n$.

(ii) This follows immediately from Proposition 6.4 of the lecture notes.

(iii) By the rules for limits (Proposition 6.3 of the lecture notes), since $L \neq 0$ we have $2/x_n \to 2/L$ as $n \to \infty$ and $x_{n+1} \to L$ as $n \to \infty$. Consequently, $L = 2/L$, so $L^2 = 2$, thus $L = \pm \sqrt{2}$. Since we already know that $1 \leq L \leq 2$ we conclude that $L = +\sqrt{2}$. 

(iv) Calculating the fist few terms gives $1,2,1,2, \dots$ and it is clear that this pattern repeats. This sequence certainly does not converge to $\sqrt{2}$ or anything else for that matter, and so the argument must be wrong. The error was in assuming that $L$ existed at all. Without some extra information such as monotonicity of the sequence $(x_n)$ there is no reason for this to be true, and indeed it isn't true in this case.  

(v) The problem for us is that our sequence bounces between $x_n$ and $2/x_n$ and isn't monotonic. The sequence in Heron's method is monotonic, and its $(n+1)$'th term is obtained by taking the {\em average} of successive terms of our sequence i.e. Heron's $x_{n+1}$ is $\frac{1}{2} (x_n + 2/x_n)$. This tends to average out the non-convergent behaviour of our sequence while retaining its good properties.


\bigskip
\noindent
{\bf Question 12.} Recall Heron's example, Example 7.5 of the lecture notes. Consider a 
more general set-up where $x_1 = \alpha$
and 
\[ x_{n+1} = a x_n + \frac{b}{x_n}.\]
(In Heron's setting, $\alpha = 2$, $a = \frac{1}{2}$ and $b=1$.)

(i) What conditions on $\alpha, a$ and $b$ would be needed for $(x_n)$ to converge to  $\sqrt{3}$? $\sqrt{5}$? $\sqrt{51}$?

(ii) Let $\alpha = 3$, $a = 1/2$ and $b = 3/2$. Which three auxiliary facts would you need to establish in order to be able to conclude that $x_n \to \sqrt{3}$?

(iii) Establish the three auxiliary facts.

{\bf Solution.} (i) If $x_n \to L \neq 0$ then by the rules for limits we must have
\[ L = a L + \frac{b}{L}\]
i.e. $ (1-a)L^2 = b$ or $L^2 = \frac{b}{1-a}$.
So for $\sqrt{3}$ we'd try $a \neq 1$ and $b$ satisfying $\frac{b}{1-a} = 3$ (e.g. $a = 1/2, b= 3/2$), for $\sqrt{5}$ we'd try $a \neq 1$ and $b$ satisfying $\frac{b}{1-a} = 5$, (e.g. $a = 1/2, b= 5/2$), and for $\sqrt{51}$ we'd try $a \neq 1$ and $b$ satisfying $\frac{b}{1-a} = 51$ (e.g. $a = 1/2, b= 51/2$). (Notice that so far there is no condition on $\alpha$ and so far this is just a guess, nothing yet guarantees convergence of the sequences in question.)

(ii) Following Heron's method we need (a) $x_n > 0$ for all $n$, (b) $x_n^2 \geq 3$ for all $n$ and (c) $x_{n+1} \leq x_n$ for all $n$.
If we have these, then $(x_n)$ is a decreasing sequence which is bounded below by $\sqrt{3}$ and so by the MCT must converge to some $L$, which, by part (i), is precisely $\sqrt{3}$.

(iii) (a): $x_n > 0$ for all $n$ is easy by induction, we skip the details. 

(b): We clearly have $x_1^2 = 9 >3$. Suppose that for some $n$ we have $x_n^2 \geq 3$. We wish to conclude that $ \frac{1}{4}\left( x_n + \frac{3}{x_n}\right)^2 \geq 3$. This is the same as showing 
\[ x^2 \geq 3 \mbox{ implies } \left(x + \frac{3}{x}\right)^2 \geq 12.\]
Now 
\[ \left(x + \frac{3}{x}\right)^2 \geq 12 \iff
x^4 + 6x^2 +9 \geq 12x^2 \iff x^4 - 6x^2 + 9 \geq 0 
\]
\[ \iff (x^2 - 3)^2 \geq 0
\]
(which is true irrespective of whether or not $x^2 \geq 3$). 

(c): We have
\[ x_{n+1} - x_n = \frac{1}{2}\left( x_n + \frac{3}{x_n}\right) - x_n = - \frac{x_n}{2} + \frac{3}{2x_n} = \frac{3 - x_n^2}{2x_n} \leq 0
\]
by part (b), and so $(x_n)$ is decreasing.

\bigskip
\noindent
{\bf Question 13.} There is more mileage in Heron's method to approximate $\sqrt{2}$ than we have yet uncovered. In particular, 
it can be pushed further to get a {\em rate of convergence} of Heron's sequence to $\sqrt{2}$, thereby giving quantitative error estimates on the successive approximations to $\sqrt{2}$ that it yields. Recall that $x_1 = 2$ and $x_{n+1} = \frac{1}{2} (x_n + 2/x_n)$, and we have already proved that $\sqrt{2} \leq x_n \leq 2$ and $x_{n+1} \leq x_n$ for all $n$.  

\medskip
\noindent
(i) Show that 
\[ x_n - x_{n+1} = (x_n - x_{n-1}) \frac{x_n x_{n-1} -2}{2 x_n x_{n-1}}\]
for all $n$, and deduce that
\[ x_n - x_{n+1} \leq \frac{(x_{n-1} - x_{n})}{2}\]
for all $n$.

\medskip
\noindent
(ii) Deduce that $x_n - x_{n+1} \leq \frac{1}{2^{n}}$ for all $n \geq 1$.

\medskip
\noindent
(iii) Deduce that for $n > m$ we have $x_n - x_{m} \leq \frac{1}{2^{m-1}}$.

\medskip
\noindent
(iv) Deduce that $\sqrt{2} - x_m \leq  \frac{1}{2^{m-1}}$ for all $m$.

\bigskip
\noindent
{\bf Solution.} (i)
\[ x_n - x_{n+1} = \frac{1}{2}\left(\left(x_{n-1} + \frac{2}{x_{n-1}}\right) - \left(x_n + \frac{2}{x_n}\right)\right)
=  \frac{1}{2}\left(x_{n-1} - x_{n}\right) + \frac{1}{x_{n-1}} - \frac{1}{x_{n}}\]
\[= \left(x_{n-1} - x_{n}\right)\left(\frac{1}{2} + \frac{-1}{x_n x_{n-1}}\right) = \left(x_{n-1} - x_{n}\right)\frac{x_n x_{n-1} - 2}{2x_n x_{n-1}}.
\]
Now all the terms $\left(x_{n-1} - x_{n}\right)$, $\left(x_{n-1} - x_{n}\right)$, $(x_n x_{n-1} - 2)$ and $2x_n x_{n-1}$ are nonnegative, and $ x_n x_{n-1} - 2 \leq (2 \times 2) - 2 =2$ while $2 x_n x_{n-1} \geq 2 \times \sqrt{2} \sqrt{2} = 4$. Therefore,
\[ x_n - x_{n+1} = \left(x_{n-1} - x_{n}\right)\frac{x_n x_{n-1} - 2}{2x_n x_{n-1}} \leq \frac{(x_{n-1} - x_{n})}{2}\]
for all $n$, as required.

(ii) We prove this by induction. The case $n=1$ is just $x_1 - x_2 \leq 1/2$ which is true as $x_1 = 2$ and $x_2 = 3/2$. Assume that for some $n$ we have $x_{n-1} - x_{n} \leq \frac{1}{2^{n-1}}$. Then, by part (i), 
$x_n - x_{n+1} \leq \frac{(x_{n-1} - x_{n})}{2} \leq \frac{1}{2} \frac{1}{2^{n-1}} = \frac{1}{2^{n}}$. This completes the inductive step.

(iii) \[x_n - x_m = (x_n - x_{n-1}) + (x_{n-1} - x_{n-2}) + \dots + (x_{m+1} - x_m) \]
\[ \leq \frac{1}{2^n} + \frac{1}{2^{n-1}} + \dots + \frac{1}{2^m} \leq 
\frac{1}{2^{m-1}}\]
by the formula for a finite geometric progression.

(iv) Fix $m$ and simply let $n \to \infty$ in the conclusion of the previous part.

\bigskip
\noindent
{\bf Question 14.} Use the result of the previous question 
to find a rational number $x$ such that $0 < \sqrt{2} - x < 10^{-5}$.
\end{document}