%%%%%%%%%%%%%%%%%%%%%%%%%%%%%%%%%%%%%%%%%
% The Legrand Orange Book
% LaTeX Template
% Version 2.0 (9/2/15)
%
% This template has been downloaded from:
% http://www.LaTeXTemplates.com
%
% Mathias Legrand (legrand.mathias@gmail.com) with modifications by:
% Vel (vel@latextemplates.com)
%
% License:
% CC BY-NC-SA 3.0 (http://creativecommons.org/licenses/by-nc-sa/3.0/)
%
% Compiling this template:
% This template uses biber for its bibliography and makeindex for its index.
% When you first open the template, compile it from the command line with the 
% commands below to make sure your LaTeX distribution is configured correctly:
%
% 1) pdflatex main
% 2) makeindex main.idx -s StyleInd.ist
% 3) biber main
% 4) pdflatex main x 2
%
% After this, when you wish to update the bibliography/index use the appropriate
% command above and make sure to compile with pdflatex several times 
% afterwards to propagate your changes to the document.
%
% This template also uses a number of packages which may need to be
% updated to the newest versions for the template to compile. It is strongly
% recommended you update your LaTeX distribution if you have any
% compilation errors.
%
% Important note:
% Chapter heading images should have a 2:1 width:height ratio,
% e.g. 920px width and 460px height.
%
%%%%%%%%%%%%%%%%%%%%%%%%%%%%%%%%%%%%%%%%%

%----------------------------------------------------------------------------------------
%	PACKAGES AND OTHER DOCUMENT CONFIGURATIONS
%----------------------------------------------------------------------------------------

%\documentclass[11pt,fleqn,dvipsnames]{book} % Default font size and left-justified equations
\documentclass[11pt,dvipsnames]{book}






%----------------------------------------------------------------------------------------

\input{structure} % Insert the commands.tex file which contains the majority of the structure behind the template



%%agregué




%%%My stuff


%\usepackage[utf8x]{inputenc}
\usepackage[T1]{fontenc}
\usepackage{tgpagella}
%\usepackage{due-dates}
\usepackage[small]{eulervm}
\usepackage{amsmath,amssymb,amstext,amsthm,amscd,mathrsfs,eucal,bm,xcolor}
\usepackage{multicol}
\usepackage{array,color,graphicx}
%\usepackage{enumerate}


\usepackage{epigraph}
%\usepackage[colorlinks,citecolor=red,linkcolor=blue,pagebackref,hypertexnames=false]{hyperref}

%\theoremstyle{remark} 
%\newtheorem{definition}[theorem]{Definition}
%\newtheorem{example}[theorem]{\bf Example}
%\newtheorem*{solution}{Solution:}


\usepackage{centernot}


\usepackage{filecontents}

\usepackage{tcolorbox} 





% Ignore this part, this is the former way of hiding and unhiding solutions, new version is after this
%
%\begin{filecontents*}{MyPackage.sty}
%\NeedsTeXFormat{LaTeX2e}
%\ProvidesPackage{MyPackage}
%\RequirePackage{environ}
%\newif\if@hidden% \@hiddenfalse
%\DeclareOption{hide}{\global\@hiddentrue}
%\DeclareOption{unhide}{\global\@hiddenfalse}
%\ProcessOptions\relax
%\NewEnviron{solution}
%  {\if@hidden\else \begin{tcolorbox}{\bf Solution: }\BODY \end{tcolorbox}\fi}
%\end{filecontents*}
%
%



%\usepackage[unhide,all]{hide-soln} %show all solutions
\usepackage[unhide,odd]{hide-soln} %hide even number solutions
%\usepackage[hide]{hide-soln} %hide all solutions






\def\putgrid{\put(0,0){0}
\put(0,25){25}
\put(0,50){50}
\put(0,75){75}
\put(0,100){100}
\put(0,125){125}
\put(0,150){150}
\put(0,175){175}
\put(0,200){200}
\put(25,0){25}
\put(50,0){50}
\put(75,0){75}
\put(100,0){100}
\put(125,0){125}
\put(150,0){150}
\put(175,0){175}
\put(200,0){200}
\put(225,0){225}
\put(250,0){250}
\put(275,0){275}
\put(300,0){300}
\put(325,0){325}
\put(350,0){350}
\put(375,0){375}
\put(400,0){400}
{\color{gray}\multiput(0,0)(25,0){16}{\line(0,1){200}}}
{\color{gray}\multiput(0,0)(0,25){8}{\line(1,0){400}}}
}



%\usepackage{tikz}

%\pagestyle{headandfoot}
%\firstpageheader{\textbf{Proofs \& Problem Solving}}{\textbf{Homework 1}}{\textbf{\PSYear}}
%\runningheader{}{}{}
%\firstpagefooter{}{}{}
%\runningfooter{}{}{}

%\marksnotpoints
%\pointsinrightmargin
%\pointsdroppedatright
%\bracketedpoints
%\marginpointname{ \points}
%\totalformat{[\totalpoints~\points]}

\def\R{\mathbb{R}}
\def\Z{\mathbb{Z}}
\def\N{{\mathbb{N}}}
\def\Q{{\mathbb{Q}}}
\def\C{{\mathbb{C}}}
\def\hcf{{\rm hcf}}


%%end of my stuff


\usepackage[hang, small,labelfont=bf,up,textfont=it,up]{caption} % Custom captions under/above floats in tables or figures
\usepackage{booktabs} % Horizontal rules in tables
\usepackage{float} % Required for tables and figures in the multi-column environment - they




\usepackage{graphicx} % paquete que permite introducir imágenes

\usepackage{booktabs} % Horizontal rules in tables
\usepackage{float} % Required for tables and figures in the multi-column environment - they

\numberwithin{equation}{section} % Number equations within sections (i.e. 1.1, 1.2, 2.1, 2.2 instead of 1, 2, 3, 4)
\numberwithin{figure}{section} % Number figures within sections (i.e. 1.1, 1.2, 2.1, 2.2 instead of 1, 2, 3, 4)
\numberwithin{table}{section} % Number tables within sections (i.e. 1.1, 1.2, 2.1, 2.2 instead of 1, 2, 3, 4)


%\setlength\parindent{0pt} % Removes all indentation from paragraphs - comment this line for an assignment with lots of text

%%hasta aquí


\begin{document}





%----------------------------------------------------------------------------------------
%	TITLE PAGE
%----------------------------------------------------------------------------------------

\begingroup
\thispagestyle{empty}
\begin{tikzpicture}[remember picture,overlay]
\coordinate [below=12cm] (midpoint) at (current page.north);
\node at (current page.north west)
{\begin{tikzpicture}[remember picture,overlay]
\node[anchor=north west,inner sep=0pt] at (0,0) {\includegraphics[width=\paperwidth]{Figures/blank.png}}; % Background image
\draw[anchor=north] (midpoint) node [fill=ocre!30!white,fill opacity=0.6,text opacity=1,inner sep=1cm]{\Huge\centering\bfseries\sffamily\parbox[c][][t]{\paperwidth}{\centering Proofs and Problem Solving \\[15pt] % Book title
{\huge Week 3: Bounds and Limits}\\[20pt] % Subtitle
{\Large Notes  based on Martin Liebeck's \\ \textit{A Concise Introduction to Pure Mathematics}}}}; % Author name
\end{tikzpicture}};
\end{tikzpicture}
\vfill
\endgroup


%----------------------------------------------------------------------------------------
%	COPYRIGHT PAGE
%----------------------------------------------------------------------------------------

%\newpage
%~\vfill
%\thispagestyle{empty}

%\noindent Copyright \copyright\ 2013 John Smith\\ % Copyright notice

%\noindent \textsc{Published by Publisher}\\ % Publisher

%\noindent \textsc{book-website.com}\\ % URL

%\noindent Licensed under the Creative Commons Attribution-NonCommercial 3.0 Unported License (the ``License''). You may not use this file except in compliance with the License. You may obtain a copy of the License at \url{http://creativecommons.org/licenses/by-nc/3.0}. Unless required by applicable law or agreed to in writing, software distributed under the License is distributed on an \textsc{``as is'' basis, without warranties or conditions of any kind}, either express or implied. See the License for the specific language governing permissions and limitations under the License.\\ % License information

%\noindent \textit{First printing, March 2013} % Printing/edition date

%----------------------------------------------------------------------------------------
%	TABLE OF CONTENTS
%----------------------------------------------------------------------------------------

\chapterimage{Figures/blank.png} % Table of contents heading image

%\chapterimage{chapter_head_1.pdf} % Table of contents heading image

\pagestyle{empty} % No headers

 \tableofcontents % Print the table of contents itself

\cleardoublepage % Forces the first chapter to start on an odd page so it's on the right

\pagestyle{fancy} % Print headers again

%----------------------------------------------------------------------------------------
%	PART
%----------------------------------------------------------------------------------------

\setcounter{part}{2}

\part{Week 3: Analysis}



\chapterimage{Figures/blank.png} 

\setcounter{chapter}{4}

\chapter{Upper Bounds and Least Upper Bounds}

\setcounter{page}{1}




For the next two weeks, we'll learn about the basics of {\it analysis}. This branch of mathematics studies limits and approximations, and you may have already met these when studying calculus. If you end up enjoying this topic, you might also like the following courses in subsequent years:

\begin{itemize}
\item Fundamentals of Pure Mathematics, 2nd year. 
\item Honours Analysis, 3rd year.
\item Linear Analysis, Essentials in Analysis and Probability, Functional Analysis, Fourier Analysis, 4th year.
\end{itemize}

\medskip
{\bf Why care?} Firstly, a lot of what we now call analysis was not very rigorous prior to the 1800s. The mathematician Niels Abel went so far as to say the following about the state of analysis at the time:\\

{\it  I shall devote all my efforts to bring light into the immense obscurity that today reigns in Analysis. It so lacks any plan or system, that one is really astonished that so many people devote themselves to it -- and, still worse, it is absolutely devoid of any rigour.}\footnote{Niels Abel, {\it Oeuvres (1826).}}\\

To give an idea about what he was talking about, Leibniz (that is, together with Newton, one of the founders of modern calculus) believed at one time that 
\[
\sum_{n=1}^{\infty} (-1)^{n} = \frac{1}{2},
\]
but with modern rigour we now understand that this is sadly not true. In fact, this sum is meaningless, as we shall see later. Making analysis rigorous was a big problem since it is the foundation of calculus:  many of the computations you perform in calculus, such as taking derivatives and evaluating integrals, in fact rely on analysis "under the bonnet". More shockingly, we need tools from analysis even to just take $n$'th roots, as we shall see later as well. Before getting into limits, however, we will need to discuss upper and lower bounds for sets, and to introduce a final axiom for the real numbers.


\section{Upper and Lower Bounds}

\def\LUB{{\rm LUB}}
\def\GLB{{\rm GLB}}
\def\ve{\varepsilon}
\def\limn{\lim_{n\rightarrow\infty}}

\def\R{\mathbb{R}}




\begin{definition}
Let $A$ be a nonempty subset of $\R$. 
\begin{itemize}
\item $A$ is said to be {\it bounded above} if
there exists $M\in\R$ such that $x\leq M$ for all $x\in A$. Such a number $M$ is called an {\it upper bound} for $A$. If no upper bound exists, we say $A$ is {\it unbounded above}.

\item Similarly, $A$ is said to be {\it bounded below} if
there exists $m\in\R$ such that $x\geq m$ for all $x\in A$. Such a number $m$ is called a {\it lower bound} for $A$. If no lower bound exists, we say $A$ is {\it unbounded below}.

\item We say $A$ is {\it bounded} if it is both bounded above and bounded below.\\
\end{itemize}
\end{definition}

\begin{exercise}
\label{ex:bounded}
Show that $A$ is bounded if and only if there is $t\geq 0$ so that  $|x|\leq t$ for all $x\in A$.
\begin{solution}
If $A$ is bounded, there are $m,M\in\mathbb{R}$ so that $m\leq x\leq M$ for all $x\in A$. We claim that $|x|\leq \max\{|m|,|M|\}$ for all $x\in A$. Indeed, if $x\in A$ then 
\[
x\leq M\leq  |M| \leq \max\{|m|,|M|\},
\]
and similarly
\[
x\geq m \geq - |m| \geq -\max\{|m|,|M|\}
\]
so that
\[|x|\leq \max\{|m|,|M|\}\]
for all $x\in A$.
Conversely, if $|x|\leq t$ for all $x\in A$, then $-t\leq x\leq t$ for all $x\in A$ (since either $x\geq 0$ in which case $x=|x|\leq t$, or $x\leq 0$, so $x=-|x|\geq -t$). Thus, $-t$ is a lower bound and $t$ is an upper bound for $A$.
\end{solution}
\end{exercise}

Recall that we write intervals using the notation
\[
[a,b]=\{x\in\mathbb{R} \;\; | \;\; a\leq x\leq b\}, \;\;\;
(a,b)=\{x\in\mathbb{R} \;\; | \;\; a< x< b\}, 
\]
\[
(a,b]=\{x\in\mathbb{R} \;\; | \;\; a< x\leq b\}, \;\;\;
[a,b)=\{x\in\mathbb{R} \;\; | \;\; a\leq  x< b\}.
\]
For unbounded intervals, we write
\[
[a,\infty)=\{x\in\mathbb{R} \;\; | \;\; a\leq x \}, \;\;\;
(a,\infty)=\{x\in\mathbb{R} \;\; | \;\; a< x \}, 
\]
\[
(-\infty,b]=\{x\in\mathbb{R} \;\; | \;\;  x\leq b\}, \;\;\;
(-\infty,b)=\{x\in\mathbb{R} \;\; | \;\; x< b\}.
\]

 

\begin{example} 

\begin{enumerate}[label=(\alph*)]
\item Let $A=[0,3)$. $A$ is bounded above since $3$ is an upper
bound; $A$ is bounded below since $0$ is a lower bound.\\

\item Let $A = \{ t+\frac{1}{t}\, :\, t>0\}$. %\subseteq (0,\infty ).$ 
$A$ is
bounded below since $0$ is a lower bound. (We could have also shown it was bounded below by using the AM-GM inequality: for $t>0$,
\[
\frac{t+1/t}{2}\geq \sqrt{t\cdot \frac{1}{t}}=1,
\]
so $2$ is also a lower bound for $A$.) But $A$ is not bounded above, since, for any $M\in\mathbb{R}$ we can find $t>0$ such that $t+\frac{1}{t}\geq M$: if $M \leq 0$ then any $t>0$ works; if $M>0$ then $t= M$ works since $M+\frac{1}{M}\in A$ is greater than $M$.


\item Let $ A=\{ (-1)^n(1-\frac{1}{n})\, |\,
n\in\N\}=\{0,\frac{1}{2},-\frac{2}{3},\frac{3}{4},-\frac{4}{5},\frac{5}{6},\cdots
\}$. $A$ is bounded: 1 is an upper bound since for all $n$:
\[
(-1)^{n}\left(1-\frac{1}{n}\right)\leq 1-\frac{1}{n}<1
\]
and one can show similarly $-1$ is a lower bound.\\

\item Let $A=\{\sin n\, |\, n\in\N\}=\{ \sin 1,\, \sin 2,\, \sin 3,\,
\cdots \}.$ Because $-1 \leq \sin x \leq 1$ for all $x$, $A \subseteq
       [-1,1]$, so $A$ is bounded. 
\end{enumerate}
\end{example}



\section{Least Upper Bounds}

Note that a nonempty set which is bounded above will have many upper bounds: if $M$ is an upper bound for a set $A$, so is every $t\geq M$. It is often useful to know what is the best or {\it least} upper bound for a set. 

\begin{definition}
Given a nonempty set $A\subseteq \mathbb{R}$, a number $L$ is a {\it least upper bound} (LUB) or {\it supremum} for $A$ if
\begin{enumerate}[label=(\alph*)]
\item $L$ is an upper bound for $A$, that is, $x\leq L$ for all $x\in A$, and
\item for all $t<L$, $t$ is not an upper bound, that is, there is $x\in A$ so that $x>t$. 
\end{enumerate}
Similarly, we say $L$ is a {\it greatest lower bound} (GLB) or {\it infimum} for $A$ if
\begin{enumerate}[label=(\alph*)]
\item $L$ is a lower bound for $A$, that is, $x\geq L$ for all $x\in A$, and
\item for all $t>L$, $t$ is not a lower bound, that is, there is $x\in A$ so that $x<t$. 
\end{enumerate}
If an LUB exists for a set $A$, we denote it by $\LUB(A)$. Similarly, we denote a GLB for $A$ by $\GLB(A)$.
\end{definition}

In other words, $L$ is a least upper bound for a set $A$ if and only if it is the mimimum of all upper bounds for $A$. This allows us to speak of "the" LUB of a set rather than "an" LUB. When proving that a number is the LUB for some set, we follow a two-step process of verifying conditions (a) and (b) in the definition. 

\begin{example}
Consider the set $[0,1)$. The least upper bound is $1$ since (a) $1$ is clearly an upper bound for the set by the definition of $[0,1)$ and (b) if $t<1$, then $t$ is not an upper bound, since we can find $x\in [0,1)$ bigger than $t$: if $t<0$, we can just take $x=0$; otherwise, if $0\leq t<1$, then $\frac{1+t}{2}\in [0,1)$ is larger than $t$. 
\end{example}


\begin{example}
Let $S=\{x\in\mathbb{Q} \; | \; x<0\}$. We claim that $\LUB(S)=0$. 

\begin{enumerate}[label=(\alph*)]
\item Since every $x\in S$ is negative by definition, $0$ is an upper bound. 
\item Let $t<0$, we need to show $t$ is not an upper bound. Recall from Week 1 that every interval contains a rational, so in particular, the interval $(t,0)$ contains a rational $r$, and since $r<0$, we know $r\in S$. Since $r>t$, $t$ is not an upper bound for $S$.
\end{enumerate}
\end{example}

\begin{example}
Let $A=\{\frac{1}{n} \; | \; n\in\mathbb{N}\}$, we claim that $\GLB(A)=0$. 

Again, we verify the two conditions of being a GLB:
\begin{enumerate}[label=(\alph*)]
\item Since $\frac{1}{n}>0$ for all $n\in\mathbb{N}$, $0$ is clearly a lower bound for $A$.
\item Next, we need to show each $t>0$ is not a lower bound, that is, that there is $\frac{1}{n} \in A$ with $\frac{1}{n}<t$. Let $t>0$. By the Archimedean property, we can find $n\in\mathbb{N}$ so that $n>\frac{1}{t}$, and so $\frac{1}{n}<t$. Since $\frac{1}{n}\in A$, this proves $t$ is not a lower bound. 
\end{enumerate}
\end{example}



Given that a maximum or minimum of a set need not exist, why should the least upper bound (which is just the minimum of all upper bounds) exist at all? It turns out that this is an important axiom for the real numbers which we shall need to assume, and it cannot be derived from any of the axioms we have assumed so far:\\

\begin{tcolorbox}
{\noindent {\bf Completeness Axiom for the real numbers:} Every nonempty subset of $\mathbb{R}$ which is bounded above has a least upper bound. \\}
\end{tcolorbox}

\begin{exercise}
Why does the completeness axiom imply that every nonempty subset of $\mathbb{R}$ which is bounded below has a GLB?
\begin{solution}
See Exercise 5.3 (c) below.
\end{solution}
\end{exercise}
An important consequence of the Completeness Axiom is the existence of $n$'th roots of positive real numbers as described in Proposition 2.7. We will establish this as Theorem 6.5. Another very important consequence of the Completeness Axiom for the real numbers is the Archimedean property which we mentioned earlier. We will visit this implication later. For now you should feel free to use the Archimedean property in the exercises below. 
%
%
%The Archimedean property is actually a consequence of the Completeness Axiom:
%
%\begin{theorem}
%The Archimedean property holds, that is, for any $x, y \in \mathbb{R}$ with $x,y>0$, there is $n\in\mathbb{N}$ so that $ny>x$. 
%\end{theorem}
%
%\begin{proof}
%We will prove by contradiction: if $y>x$, we can just pick $n=1$. If $y\leq x$, consider the set $A=\{n\in\mathbb{N} \;\; | \; \; yn\leq x\}$. Since $ny\leq x$ for all $n\in A$, $n\leq \frac{x}{y}$ for all $n\in A$, so $A$ is bounded above, and the completeness axiom implies there is a least upper bound $L$. We claim $L\in A$. Indeed, since $L=\LUB(A)$, for any $t<L$, there is $n\in A$ with $t<n$

% Or: ETS $\mathbb{N}$ not bounded above. If it were, there would be a least upper bound $L$. Then there would be $N \in \mathbb{N}$ with $N > L-1/2$. But then $N+1$ > L + 1/2$.
%\end{proof}

\section{Exercises}


The exercises in Liebeck's book relevant to this section are in Chapter 22.

\begin{exercise}
Let $A,B\subseteq \mathbb{R}$ be nonempty sets that are bounded above. For each statement below, either prove it or provide a counterexample.
\begin{enumerate}[label=(\alph*)]
\item If $A$ is bounded above, then $\LUB(-A)=-\LUB(A)$, where $-A=\{-x \; | \; x\in A\}$.
\begin{solution}
This is not true: if $A=(-1,2)$, then $-A=(-2,1)$, so $-\LUB(A)=-2\neq \LUB(-A)=-1$. Another example is $A = (-\infty, 0)$ with $\LUB(A) = 0$, for which $-A$ is not even bounded above, and thus $-A$ has no LUB. Many more examples are possible.
\end{solution}
\item If $A$ is bounded above, then $-A$ is bounded below and $\GLB(-A)=-\LUB(A)$.
\begin{solution}
This is true: Let $L=\LUB(A)$, we claim $\GLB(-A)=-L$. First we show it is a lower bound: if $x\in -A$, then $-x\in A$, so $-x\leq L$, hence $x\geq -L$. Now we show it is the greatest lower bound: let $t>-L$, we will show there is $x\in -A$ with $x<t$. Note that $-t<L$, and since $L=\LUB(A)$, there is $y\in A$ with $y>-t$, so $-y<t$ and $-y\in -A$, thus $t$ is not a lower bound. This proves $\GLB(-A)=-L$.
\end{solution}
\item If $c\in \mathbb{R}$, then $\LUB(c+A)=c+\LUB(A)$ where $c+A=\{c+x \; | \; x\in A\}$. 
\begin{solution}
This is true: let $L=\LUB(A)$, we claim $\LUB(c+A)=c+L$. First we show $c+L$ is an upper bound: if $x\in c+A$, then $x-c\in A$, so $x-c\leq L$, so $x\leq c+L$. Now we show that it is a least upper bound, i.e., that for $t<c+L$, there is $x\in c+A$ with $x>t$: Let $t<c+L$, then $t-c<L$, and since $L=\LUB(A)$, there is $y\in A$ with $y>t-c$, and so $y+c>t$ and $y+c\in A$, so $t$ is not an upper bound for $A$.
\end{solution}
\item $\LUB(A)=\max(A)$. 
\begin{solution}
This is not true in general since $\max(A)$ may not be defined. A concrete conterexample is $A = (0,1)$ for which $\LUB(A) =1$, but which has no maximum. Many more examples are possible.
\end{solution}
\item If $c>0$, then $\LUB(cA)=c\LUB(A)$ where $cA=\{cx\; | \; x\in A\}$. 
\begin{solution}
This is true, its proof is very similar to (c), so we omit it.
\end{solution}

\item $\LUB(AB)=\LUB(A)\cdot \LUB(B)$ where $AB=\{xy \;\; | \;\; x\in A,\; y\in B\}$.
\begin{solution}
Not true: let $A=(-2,1)$ and $B=(-2,1)$, then $AB=(-2,4)$ so $\LUB(AB)=4\neq \LUB(A)\cdot \LUB(B)=1$. Many more examples are possible.
\end{solution}
\item if $A\subseteq B$, then $\LUB(A)\leq \LUB(B)$. 
\begin{solution}
This is true: note that $x\leq \LUB(B)$ for all $x\in B$, and so $x\leq \LUB(B)$ for all $x\in A$ since $A\subseteq B$. Hence, $\LUB(A)\leq \LUB(B)$.
\end{solution}
\end{enumerate}


\end{exercise}


\begin{exercise}
Suppose $A$ and $B$ are bounded above. Show that $\LUB(A+B)=\LUB(A)+\LUB(B)$, where $A+B=\{x+y\; | \; x\in A,\; y\in B\}$.
\begin{solution}
Let $L=\LUB(A)$ and $L'=\LUB(B)$, we'll show $\LUB(A+B)=L+L'$. First we show (a) that $L+L'$ is an upper bound for $A+B$: Note that if $z\in A+B$, then $z=x+y$ where $x\in A$ and $y\in B$, so 
\[
z=x+y\leq L+L'
\]
and so $L+L'$ is an upper bound. Now we show (b) that if $t<L+L'$, then there is $z\in A+B$ with $z>t$. Recall that since $L=\LUB(A)$, for any $s<L$ there is $x\in A$ with $x>s$, and since $L'=\LUB(B)$, for any $s'<L'$ there is $y\in B$ with $y>s'$, and so $x+y>s+s'$ and $x+y\in A+B$. Hence, we just need to find values of $s<L$ and $s'<L'$ so that $s+s= t$, since then $x+y>s+s'= t$. Let's rewrite $s$ and $s'$ as $s=L-r$ and $s'=L-'r'$ where $r,r'>0$. Then we want to find $r,r'>0$ so that $t=s+s'=L+L'-r-r'$, i.e. so that $r+r'=L+L'-t$, so we can just choose $r=r'=\frac{L+L'-t}{2}$. 
\end{solution}
\end{exercise}



\begin{exercise}
Determine which of the following sets are bounded or unbounded above. 
\begin{enumerate}[label=(\alph*)]
\item $A=\{x \in \R\; | \; x^2-3x+2<2\}$. 
\begin{solution}
If $x$ is in $A$, then
\[
x^2-3x=x(x-3)<0
\]
so exactly one of $x$ or $x-3$ is negative, which is only possible if $x-3< 0< x$. Thus for all $x \in A$, $0<x<3$, so $0$ and $3$ are lower and upper bounds for $A$ respectively. Therefore $A$ is bounded.
\end{solution}
\item $A= \{x \in \R \; | \; \cos x<1/2\}$. [You may assume any familiar properties of the cosine function.]
\begin{solution}
Note that $\cos \left(\frac{\pi}{2}+2\pi n\right)=0$ for all $n\in\mathbb{Z}$ and so $\frac{\pi}{2}+2\pi n$ is in $A$ for all $n$. In particular, $A$ is unbounded above and below: if $M\in\mathbb{R}$, by the Archimedean property, we can find $n$  so that $2\pi n>M-\frac{\pi}{2}$, i.e. so that $2\pi n+\frac{\pi}{2}>M$, hence $A$ is not bounded above. That it is not bounded below is similar. 
\end{solution}
\item $A = \{n^2-n \; | \; n\in\mathbb{N}\}$. 
\begin{solution}
If $n\in\mathbb{N}$, then $n\geq 1$ and so $n^2\geq n$, i.e. $n^2-n \geq 0$, hence $A$ is bounded below with $0$ being a lower bound. It is not bounded above. Indeed, let $M \in \R$. Note that for $n > 1$ we have $n^2-n=n(n-1)> n$, so by the Archimedean property, we may find $n>\max\{M,1\}$ such that $n^2-n=n(n-1)\geq n>M$. Thus $M$ cannot be an upper bound for $A$, and we deduce that $A$ is not bounded above.
\end{solution}
\item $A=\{ab \;\;| \;\;  a, b \in \R, \; a+b=1\}$.
\begin{solution}
Note that if $a+b=1$, then $b=1-a$, so that
\[
A = \{a(1-a)\;\; |\;\;a\in\mathbb{R}\} = \{- a^2 +a \;\; |\;\;a\in\mathbb{R}\}.
\]
Which real numbers $r$ are members of this set? It is precisely those $r \in \R$ such that the quadratic equation
\[ 
-a^2 + a =r
\]
has a real solution. To figure this out we complete the square:
\[ 
-a^2 + a =r \iff -\left(a-\frac{1}{2}\right)^2 = r -\frac{1}{4} \iff \left(a-\frac{1}{2}\right)^2 =\frac{1-4r}{4},
\]
and so long as $(1-4r)/4 \geq 0$ -- that is, $r \leq 1/4$ --  it has a real square root by Proposition 2.7. Thus $r \in A$ if and only if $r \leq 1/4$. Therefore 
\[
A = (- \infty, 1/4]
\]
and this is bounded above (with $1/4$ as an upper bound) but not bounded below.

\end{solution}
\end{enumerate}

\end{exercise}

\begin{exercise}
Find the least upper bounds for the following sets:
\begin{enumerate}[label=(\alph*)]
\item $\{\frac{1}{n} \; | \; n\in \mathbb{N}\}$.
\begin{solution}
Note that for $n\geq 1$, $\frac{1}{n}\leq 1$, so $1$ is an upper bound. Since $1$ is in the set, it is in fact a maximum and thus equal to the least upper bound.
\end{solution}
\item $\{-\frac{1}{n^2} \; | \; n\in\mathbb{N}\}$. 
\begin{solution}
We claim that the LUB is $0$. It is clearly an upper bound since all elements of the set are negative. Now we must show it is the least upper bound: let $t<0$, we'll show there is $-\frac{1}{n^2}>t$. This is the same as finding $n$ so that $n>(-t)^{-1/2}$, which is possible by the Archimedean property.
\end{solution}
\item $\{2-x^2 \; | \; x\in \R\setminus\{0\}\}$.
\begin{solution}
We claim the least upper bound is 2. Since $x^2\geq 0$, $2-x^2\leq 2$ for any $x\neq 0$, so $2$ is an upper bound. Now suppose $t<2$, we'll show there is $x$ so that $2-x^2>t$, i..e. so that $x^2<2-t$. Since $2-t>0$, we can just set $x=\sqrt{\frac{2-t}{2}}$ and this will work. This proves $2$ is the least upper bound.
\end{solution}
\item $\{x\in\mathbb{Q}: x^2<2\}$.
\begin{solution}
This is the same  as the set of $x\in\mathbb{Q}$ so that 
\[
0>x^2-2=(x-\sqrt{2})(x+\sqrt{2})
\]
so exactly one of these terms must be negative, so we must have $-\sqrt{2}<x<\sqrt{2}$, that is, 
\[
\{x\in\mathbb{Q}: x^2<2\}=(-\sqrt{2},\sqrt{2})
\]
We claim the least upper bound is $\sqrt{2}$: clearly $\sqrt{2}$ is an upper bound, so now we must show it is a least upper bound: let $t<\sqrt{2}$, we'll show there is $x$ in the set with $x>\sqrt{2}$. From Week 1, we saw that for any $a<b$, there is a rational $x$ with $a<x<b$, so in particular, there is $x$ with $t<x<\sqrt{2}$. This completes the proof that $\sqrt{2}$ is a LUB. The proof that $-\sqrt{2}$ is the GLB is similar.
\end{solution}
\end{enumerate}

\end{exercise}






\chapter{Limits}


\section{Motivation and the $\epsilon-N$ definition of a limit}


How does a computer compute $\sqrt{51}$? A computer can't possibly memorise all possible square roots, so it must compute it from scratch. Moreover, it can't possibly compute the entire decimal expansion (next week we will see that, because $\sqrt{51}$ is irrational, the decimal expansion never repeats), so it can only come up with an approximation. Thus, your computer requires an algorithm for computing $\sqrt{51}$ (or any root) to any degree of accuracy. But before implementing such an algorithm, we need to {\it justify} that it can indeed approximate $\sqrt{51}$ to any degree of accuracy. That is, if my algorithm spits out a sequence of numbers $x_1,x_2,...$, and I want an approximation that is within $10^{-10}$ of $\sqrt{51}$, I need to show that for all large enough $n$, $x_n$ is at most $10^{-10}$ away from $\sqrt{51}$, and I need this to happen for {\it every} degree of accuracy $\epsilon>0$ that I may require (not just $10^{-10}$). All this motivates the following:


\begin{definition}[ The $\epsilon-N$ definition of a limit]
Let $(x_{n})_{n=1}^{\infty}$ be a sequence of real numbers $x_{1},x_{2},...$ and let $L\in \mathbb{R}$. We say that $x_n$ {\it converges to $L$}, and we write $\lim_{n\rightarrow\infty} x_{n}=L$, or $x_{n}\rightarrow L$ as $n \rightarrow \infty$, if for all $\epsilon>0$, there is a real number $N$ so that for $n> N$, we have 
\[
|x_{n}-L|<\epsilon
\]
or equivalently,\footnote{See Exercise~\ref{one}.}
\[
L-\epsilon<x_n<L+\epsilon.
\]
\end{definition}
\begin{definition}We say that a sequence $(a_n)$ is {\it convergent} if for some $L \in \R$, it converges to $L$. 
\end{definition}
{\bf Exercise.} Give an example of a convergent sequence. Give an example of a non-convergent sequence.

Notice that here we are working with the notion of a {\bf sequence} $(x_n)_{n=1}^\infty$ -- this is just an {\em ordered} list of real numbers $x_n$, indexed by the natural numbers $\mathbb{N}$. Or, if you prefer, you can think of a sequence as a function $x:\mathbb{N}\to \mathbb{R}$ with domain $\mathbb{N}$ and codomain $\mathbb{R}$.

\medskip
Note that in Liebeck, the $N$ in Definition 6.1 is required to be an integer, but it actually makes no difference since we can just round $N$ up to an integer if we like, using the Archimedean principle. It is important to recognise that, in this definition, we expect $N$ to depend on $\epsilon$.

\newcommand{\floor}[1]{\left\lfloor #1 \right\rfloor}
\newcommand{\ceil}[1]{\left\lceil #1 \right\rceil}

%The following notation will be convenient below: for a number $x$, we let $\floor{x}$ be $x$ rounded down, i.e. the largest integer at most $x$, so that
%\[
%\floor{x} \leq x<\floor{x}+1\;\;\; \mbox{ and }\;\;\; \floor{x}\in\mathbb{Z}.
%\]

\begin{example}
Show that $\lim_{n\rightarrow \infty} \frac{1}{n}=0$. \\

We need to show that for all $\epsilon>0$ there is $N$ so that $n> N$ implies $|\frac{1}{n}-0|<\epsilon$, which is the same as $\frac{1}{n}<\epsilon$. This is equivalent to finding $N$ so that $n>N$ implies $n>\frac{1}{\epsilon}$. Thus, if we pick $N=\frac{1}{\epsilon}$, we see that if $n> N$,
\[
\left|\frac{1}{n} -0\right|=\frac{1}{n} < \frac{1}{N}= \frac{1}{\frac{1}{\epsilon}}=\epsilon.
\]
\end{example}

\begin{protip}
{\bf Note on establishing limits:} Don't expect a nice looking proof to fall into your lap immediately when trying to establish a certain limit. It usually requires some rough work first, which you can then use to help you write your proof.
\end{protip}


\begin{example}
Show that $\lim_{n\rightarrow \infty} \frac{5n^2+2n}{n^2+4}=5$. \\
%
%{\bf Intuition:} For guessing the limit, note that in the numerator and denominator are polynomials in $n$ and the terms that grow fastest in $n$ are $5n^2$ and $n^2$ respectively. So as $n$ gets large, the other terms ($2n$ and $4$ respectively) will be insignificant in comparison to these terms. Thus, as $n$ gets larger, this should look approximately like $\frac{5n^2}{n^2}=5$. So my guess is that the limit is $5$.\\
%
%{\bf Note:} Intuition is not a proof! The above is just how we might go about hypothesizing an answer, but now we must justify our hypothesis. \\

{\bf Rough work:} We are going to need to show that $ \frac{5n^2+2n}{n^2+4}-5$ is small when $n$ is large, so it first makes sense to simplify this expression algebraically to see if we can identify when it might be small. Doing this,
\[\frac{5n^2+2n}{n^2+4}-5 =\frac{2n-20}{n^2+4},
\]
and therefore, using the triangle inequality and the fact that $4\geq0$,
\[\left|\frac{5n^2+2n}{n^2+4}-5\right|
\leq \left| \frac{2n-20}{n^2+4} \right| 
\leq \frac{2n}{n^2 + 4} + \frac{20}{n^2 + 4}
\leq \frac{2}{n} + \frac{20}{n^2}.
\]
This is looking good because all we need to do to make this small is to make both $2/n$ and $20/n^2$ small, and this is very reasonable. Now we bring in the $\epsilon > 0$. If $n$ is such that both $2/n < \epsilon/2$ and $20/n^2 < \epsilon/2$, we can deduce that 
\[\left|\frac{5n^2+2n}{n^2+4}-5\right| < \frac{\epsilon}{2}+ \frac{\epsilon}{2} = \epsilon,
\]
which is exactly what we need. What do we need $n$ to satisfy for this to be true? Well, for $\frac{2}{n} < \frac{\epsilon}{2}$ we need $n > \frac{4}{\epsilon}$, and for $\frac{20}{n^2} < \frac{\epsilon}{2}$ we need $n > \sqrt{\frac{40}{\epsilon}}$. To ensure both of these hold, we need $n$ to satisfy $n > \max\left\{\frac{4}{\epsilon}, \sqrt{\frac{40}{\epsilon}}\right\}.$ So we take $N = \max\left\{\frac{4}{\epsilon}, \sqrt{\frac{40}{\epsilon}}\right\}.$ 

(All the above was just rough work to help me formulate and write my formal proof. Below is what you want to do when writing your work for marking (and not the rough work). It's fine not to show how you chose your $N$, so long as it works. And there is no "single answer" for $N$ -- if $N$ works, then so does $N'$ for any $N'>N$.)\\

{\bf Now for the actual proof:} \\

{\bf Claim:} $\lim_{n\rightarrow\infty}\frac{5n^2+2n}{n^2+4}=5$. 

\begin{proof}
Let $\epsilon$ be any positive real number. We need to show that there is $N\in\mathbb{R}$ so that $n> N$ implies $\left|\frac{5n^2+2n}{n^2+4}-5\right|<\epsilon$. We assert that $N=\max\left\{\frac{4}{\epsilon}, \sqrt{\frac{40}{\epsilon}}\right\} $ works. Indeed, if $n> N$, then
\[\left|\frac{5n^2+2n}{n^2+4}-5\right|
\leq \left| \frac{2n-20}{n^2+4} \right| 
\leq \frac{2n}{n^2 + 4} + \frac{20}{n^2 + 4}
\leq \frac{2}{n} + \frac{20}{n^2} < \frac{2}{N} + \frac{20}{N^2} \leq \frac{\epsilon}{2}+ \frac{\epsilon}{2} = \epsilon.
\]
 Since we have shown we can find such an $N$ no matter what $\epsilon>0$ we started with, this proves the claim.

\end{proof}
\end{example}


\begin{example}\label{sqrootxn}
Suppose $x_{n}\geq 0$ and $x_{n}\rightarrow x>0$, then $\sqrt{x_{n}}\rightarrow \sqrt{x}$. \\

{\bf Discussion:} This is a little different from the previous example because now $x_n$ isn't given in a concrete form, and all we know about it is that $x_n \rightarrow x$, or, more informally, $|x_n - x|$ is small for all large $n$. Nevertheless the conclusion, $\sqrt{x_{n}}\rightarrow \sqrt{x}$, is similar to what we had before, and the tactic of manipulating the expression $\sqrt{x_{n}}-\sqrt{x}$ algebraically, and using some inequalities on it until it's apparent why it should also be small, is equally useful. Thus, we want to manipulate $|\sqrt{x_{n}}-\sqrt{x}|$ so that $|x_{n}-x|$ appears in the expression, since we hope that if we can make $|x_{n}-x|$ as small as we want, then we can make $|\sqrt{x_{n}}-\sqrt{x}|$ small as well.

A standard trick when working with roots is to multiply and divide by the conjugate, and we have
\[
|\sqrt{x_{n}}-\sqrt{x}|
=\left|(\sqrt{x_{n}}-\sqrt{x})\cdot \frac{\sqrt{x_{n}}+\sqrt{x}}{\sqrt{x_{n}}+\sqrt{x}}\right|
=\left|\frac{x_{n}-x}{\sqrt{x_{n}}+\sqrt{x}}\right|
=\frac{|x_{n}-x|}{\sqrt{x_{n}}+\sqrt{x}} \leq \frac{|x_{n}-x|}{\sqrt{x}}
\]
where in the last step we used the facts that $\sqrt{x_n} \geq 0$ and $x > 0$. This is looking good because we should be able to make 
$\frac{|x_{n}-x|}{\sqrt{x}}$ small by taking $n$ large enough. Indeed, now let's bring in $\epsilon > 0$. Since $x_n \rightarrow x$, we know that there is some $N \in \mathbb{R}$ such that whenever $n > N$, 
we have 
\[ |x_n - x| <\epsilon\sqrt{x}.\]
(This maybe requires a pause for thought. The definition of $x_n \rightarrow x$ talks about "$|x_n - x| < \epsilon$" for all large $n$, it doesn't say anything about "$ |x_n - x| <\epsilon\sqrt{x}$". But the definition of $x_n \rightarrow x$ applies to {\em every} positive $\epsilon$ -- and thus it applies just equally well to $\epsilon\sqrt{x}$
in place of $\epsilon$. Remember that $x$ is fixed in this discussion. The $N$ coming out depends on $x$ as well as $\epsilon$, but this is not a problem, as $x$ is fixed.)
So if $n > N$, we have
\[|x_n - x| \leq \frac{|x_n - x|}{\sqrt{x}} < \frac{\epsilon\sqrt{x}}{\sqrt{x}} = \epsilon
\]

\begin{proof}
We need to show that for any $\epsilon>0$ there is $N \in \R$ so that $n> N$ implies $|\sqrt{x_{n}}-\sqrt{x}|<\epsilon$. 

Let $\epsilon>0$. Since $x_n\rightarrow x$, there is $N$ so that $n> N$ implies $|x_{n}-x|<\epsilon\sqrt{x}$, and we take this $N$. We then have, for $n > N$,
\[
|\sqrt{x_{n}}-\sqrt{x}|
=\left|(\sqrt{x_{n}}-\sqrt{x})\cdot \frac{\sqrt{x_{n}}+\sqrt{x}}{\sqrt{x_{n}}+\sqrt{x}}\right|
=\left|\frac{x_{n}-x}{\sqrt{x_{n}}+\sqrt{x}}\right|\]
\[=\frac{|x_{n}-x|}{\sqrt{x_{n}}+\sqrt{x}}
\leq \frac{|x_{n}-x|}{\sqrt{x}} 
< \frac{\epsilon \sqrt{x}}{\sqrt{x}} = \epsilon.
\]
\end{proof}
\end{example}
{\bf Point to ponder:} What happens if $x = 0$ in this example?


\section{Rules for Limits}
\begin{definition} We say that a sequence $(a_n)$ is {\it bounded} if the set of values $\{a_{1},a_{2},...\}$ is a bounded set, i.e. there are $m,M$ so that $m\leq a_n\leq M$ for all $n$. 
\end{definition}

\begin{proposition}\label{cgtgivesbdd}
Suppose the sequence $(a_{n})$ converges. Then it is bounded. 
\end{proposition}

\begin{proof}
Let $L=\lim_{n\rightarrow\infty}a_{n}$. By the definition of a limit (with $\epsilon=1$), there is $N$ so that $n> N$ implies $|a_{n}-L|<1$ (there was nothing special about our choice of $\epsilon=1$ here, any positive number will work in this argument). Thus, for $n> N$,
\[
|a_{n}|=|a_{n}-L+L|
\leq |a_{n}-L|+|L|
<1+|L|.
\]
For $n<N$, we have $|a_{n}|\leq \max\{|a_i| : |\; i=1,...,N-1\}$. Thus, for all $n$,
\[
|a_{n}|\leq \max\{1+|L|,|a_{1}|,...,|a_{n}|\}.
\]
By Exercise \ref{ex:bounded}, the sequence is bounded.
\end{proof}
\noindent
{\bf Point to ponder:} Does the converse to Proposition~\ref{cgtgivesbdd} hold?

\begin{proposition}
\label{p:limit-rules}
Suppose that $(a_{n})$ and $(b_{n})$ converge to $a$ and $b$ respectively. Then

\begin{enumerate}[label=(\alph*)]
\item $a_{n}+b_{n}\rightarrow a+b$
\item $ a_{n}b_{n} \rightarrow ab$ 
\item If $c\in\mathbb{R}$, then $ca_{n}\rightarrow ca$.
\item If $b\neq 0$ and $b_n\neq 0$ for all $n$, then $\frac{a_{n}}{b_{n}}\rightarrow \frac{a}{b}$. 
\end{enumerate}
\end{proposition}



\begin{proof}
\begin{enumerate}[label=(\alph*)]
\item Let $\epsilon>0$. We need to show there is $N$ so that $n > N$ implies $|a_{n}+b_{n}-(a+b)|<\epsilon$. Note that by the triangle inequality
\[
|a_{n}+b_{n}-(a+b)|
=|(a_n-a)+(b_n-b)|
\leq |a_{n}-a|+|b_{n}-b|
\]
We can make $|a_{n}+b_{n}-(a+b)|<\epsilon$ by making $|a_{n}-a|$ and $|b_{n}-b|$ both less than $\frac{\epsilon}{2}$. Since $a_n\rightarrow a$, we know that there is $N_{1}$ so that $n > N_1$ implies $|a_{n}-a|<\frac{\epsilon}{2}$. Similarly, since $b_n\rightarrow b$, we know that there is $N_{2}$ so that $n > N_2$ implies $|b_{n}-b|<\frac{\epsilon}{2}$. If we set $N=\max\{N_1,N_2\}$, then for all $n > N$,
\[
|a_{n}+b_{n}-(a+b)|
\leq |a_{n}-a|+|b_{n}-b|
<\frac{\epsilon}{2}+\frac{\epsilon}{2}=\epsilon. 
\]
\item Let $\epsilon>0$.  Note that
\begin{align}
|a_{n}b_{n}-ab|
& =|a_{n}b_{n}-ab_{n}+ab_{n}-ab|
=|(a_{n}-a)b_{n} + a(b_n-b)| \notag \\
& \leq |(a_{n}-a)b_{n}| + |a(b_n-b)|  
=|a_{n}-a|\cdot |b_{n}|+ |a|\cdot |b_{n}-b|.
\label{e:anbn-ab}
\end{align}
Thus, if we can pick $N$ large enough so that $n > N$ implies  $|a_{n}-a|\cdot |b_{n}|<\frac{\epsilon}{2}$ and $|a|\cdot |b_{n}-b|<\frac{\epsilon}{2}$, then the above will imply $|a_{n}b_{n}-ab|<\epsilon$ and we'll be done. So let's focus on proving these two things.

Let's first find $N_1$ so that $n > N_1$ implies $|a_{n}-a|\cdot |b_{n}|<\frac{\epsilon}{2}$. Since $b_n$ converges, it is bounded, and so there is $M$ so that $|b_{n}|\leq M$ for all $n$, thus
\[
|a_{n}-a|\cdot |b_{n}|\leq M|a_{n}-a|.
\]
Since $a_n\rightarrow a$, there is $N_1$ so that $n > N_1$ implies $|a_{n}-a|<\frac{\epsilon}{2M}$ and hence
\[
|a_{n}-a|\leq M|a_{n}-a|<M\cdot \frac{\epsilon}{2M}=\frac{\epsilon}{2}.
\]
Now let's find $N_2$ so that $n > N_2$ implies $|a|\cdot |b_{n}-b|<\frac{\epsilon}{2}$. If $a=0$, then this is always true and we can set $N_2=1$. If $a\neq 0$, then since $b_n\rightarrow b$, there is $N_2$ so that $n > N_2$ implies $|b_{n}-b|<\frac{\epsilon}{2|a|}$, and so that
\[
|a|\cdot |b_{n}-b|<|a|\cdot \frac{\epsilon}{2|a|}=\frac{\epsilon}{2}. 
\]
Thus, if we set $N=\max\{N_{1},N_{2}\}$, then $n > N$ and  \eqref{e:anbn-ab} imply
\[
|a_{n}b_{n}-ab|
\leq |a_{n}-a|\cdot |b_{n}|+ |a|\cdot |b_{n}-b|
<\frac{\epsilon}{2}+\frac{\epsilon}{2}=\epsilon.
\]
\item We leave this one as an exercise. 
\item We claim that it suffices to show that if $b_n\rightarrow b$ and $b_n,b\neq 0$, then $b_{n}^{-1}\rightarrow b^{-1}$. Indeed, if this is the case, then  by part (b),
\[
\frac{a_{n}}{b_{n}}=a_{n}\cdot b_{n}^{-1} \rightarrow ab^{-1}=\frac{a}{b}.
\]
So now let's prove that if $b_n\rightarrow b$ and $b_n,b\neq 0$, then $b_{n}^{-1}\rightarrow b^{-1}$.  \\

Let $\epsilon>0$. We need to find $N$ so that $n > N$ implies $|b_{n}^{-1}-b^{-1}|<\epsilon$. Observe that 
\[
\left|\frac{1}{b_{n}}-\frac{1}{b}\right|
=\left|\frac{b-b_{n}}{bb_{n}}\right|
=\frac{|b-b_{n}|}{|b|\cdot |b_{n}|}.
\]
Suppose we can show that for some number $s>0$ we have
\begin{equation}
\label{e:bn>m}
|b_{n}|\geq s \;\;\; \mbox{ for all }n\in\mathbb{N}.
\end{equation}
 Then since $b_{n}\rightarrow b$, we know that for any $\epsilon_1>0$ there is $N = N(\epsilon_1)$ so that $n > N$ implies $|b_{n}-b|<\epsilon_1$, and then we would have
\[
\left|\frac{1}{b_{n}}-\frac{1}{b}\right|\leq \frac{|b-b_{n}|}{|b|\cdot |b_{n}|}
\leq \frac{\epsilon_1}{|b| s}.
\]
Hence, if we pick $\epsilon_1=|b|s\epsilon$, then all this implies that for $n > N(|b|s\epsilon)$,
\[
\left|\frac{1}{b_{n}}-\frac{1}{b}\right| 
\leq \frac{\epsilon_1}{|b| s}= \frac{|b|s\epsilon}{|b| s}=\epsilon.
\]
So now we focus on proving \eqref{e:bn>m}. Since $b_n\rightarrow b$, there is $N'$ so that $n > N'$ implies $|b_{n}-b|<\frac{|b|}{2}$. Thus, by the reverse triangle inequality,
\[
|b_{n}|
=|b_{n}-b+b|
\geq |b|-|b_{n}-b|
>|b|-\frac{|b|}{2} = \frac{|b|}{2}.
\]
For $n \leq N'$, we simply have $|b_{n}|\geq \min\{|b_{1}|,\dots,|b_{N'}|\}>0$. Thus, for all $n$, 
 \[
|b_{n}|\geq \min\left\{\frac{|b|}{2},|b_{1}|,\dots ,|b_{N'}|\right\}.
\]
This finishes the proof of  \eqref{e:bn>m}, and thus of (d). 



\end{enumerate}
\end{proof}


From these rules, we can get many more useful results:

\begin{proposition}
\label{p:power-lim}
Suppose that $x_n\rightarrow L$ as $n \to \infty$ and that $k \in \mathbb{N}$. Then $x_{n}^{k}\rightarrow L^{k}$ as $n \to \infty$.
\end{proposition}

\begin{proof}
We prove this by induction on $k$. The base case $k=1$ is a tautology. For the induction step, assume we have shown that whenever $x_n\rightarrow L$, then for some particular $k \geq 1$, $x_{n}^{k}\rightarrow L^{k}$. Now apply Proposition \ref{p:limit-rules}(b) with $a_n=x_{n}^{k}$ and $b_{n}=x_{n}$, to obtain
\[
\limn x_{n}^{k+1}=\limn x_{n}^{k}x_{n}
=\left(\limn x_{n}^{k}\right)\cdot\left(\limn x_{n}\right)
=L^{k}\cdot L = L^{k+1}
\] Thus, the proposition follows by induction. 
\end{proof}




\begin{proposition}
\label{p:x_n<y_n}
Let $(x_{n})$ and $(y_{n})$ be sequences.
\begin{enumerate}[label=(\alph*)]
\item  If $x_{n}\rightarrow x$, $y_{n}\rightarrow y$, and $x_{n}\leq y_{n}$ for all $n\in\mathbb{N}$, then $x\leq y$. 
\item If $x_n\rightarrow x$ and $x_{n}\leq y$ for all $n$, then $x\leq y$. 
\end{enumerate}
\end{proposition}


\begin{proof}
\begin{enumerate}[label=(\alph*)]
\item  Suppose for a contradiction that $x>y$. Take $\epsilon = (x-y)/2 > 0$. Since $x_{n}\rightarrow x$, there is $N_{1}$ so that $n > N_{1}$ implies 
\[
x-\frac{x-y}{2}<x_{n}<x+\frac{x-y}{2}
\]
and in particular
\[
\frac{x+y}{2}<x_{n}.
\]
Similarly, since $y_{n}\rightarrow y$, there is $N_{2}$ so that $n > N_{2}$ implies 
\[
y-\frac{x-y}{2}<y_{n}<y+\frac{x-y}{2}
\]
and in particular
\[
y_{n}<\frac{x+y}{2}.
\]
Therefore, for all $n>  \max\{N_{1},N_{2}\}$,
\[
\frac{x+y}{2}<x_{n} \leq y_n < \frac{x+y}{2}.
\]
which gives the desired contradiction.

\item This follows by (a) by letting $y_{n}=y$ for all $n$.
\end{enumerate}
\end{proof}

There is another important rule for limits, often called the "squeeze theorem":

\begin{proposition}\label{squeeze}
Suppose that $x_n \leq y_n \leq z_n$ for all $n$, and that $x_n \to L$ and $z_n \to L$. Then $y_n \to L$.
\end{proposition}

\begin{proof}
Let $\epsilon > 0$. There exist $N_1$ and $N_2$ such that for $n > N_1$ we have
\[ |x_n - L| < \epsilon\]
and for $n > N_2$ we have
\[ |z_n - L| < \epsilon.\]
Take $N = \max\{N_1, N_2\}$. For $n > N$ we have 
\[ - \epsilon < x_n - L \leq y_n - L \leq z_n - L < \epsilon.\]
Therefore for $n > N$ we have $|y_n - L| < \epsilon$.
\end{proof}


\section{Application: the existence of roots.}

Recall that in Week 1 we asserted the existence of roots; now we can now prove it. 


\begin{theorem}
\label{t:proof-roots-exist}
Let $x>0$ and $k\in\mathbb{N}$. Then there is a unique $y>0$ so that $y^{k}=x$. 
\end{theorem}

\begin{proof}
Let $S=\{s \geq 0 \; | \; s^{k}<x\}$. Then $S$ is nonempty (since $0 \in S$) and bounded above since $\max \{x, 1\}$ is an upper bound for $S$. Therefore
\[
y=\LUB(S)= \LUB(\{s \geq0 \; | \; s^{k}<x\})
\]
exists by the Completeness Axiom. We claim that $y^{k}=x$. 

Since $y$ is the least upper bound of $S$, this means that for all $n\in\mathbb{N}$, $y-\frac{1}{n}$ is not an upper bound, i.e. there is $y_{n}\in S$ so that 
\[
y-\frac{1}{n}<y_{n}\leq y.
\]
This implies $y_n\rightarrow y$: indeed, for any $\epsilon>0$, if  $N=\epsilon^{-1}$, then for $n> N$, $\frac{1}{n}<\frac{1}{N}=\epsilon$ and so
\[
y-\epsilon < y-\frac{1}{n}<y_{n}\leq y<y+\epsilon,
\]
Thus, we can find for each $\epsilon$ an $N$ so that $n> N$ implies $y-\epsilon<y_n<y+\epsilon$, hence $y_n\rightarrow y$.  

Since $y_n\in S$, $y_{n}^{k}<x$ for all $n$, so Proposition \ref{p:power-lim} and Proposition \ref{p:x_n<y_n} imply 
\[
y^{k}=\limn y_{n}^{k}\leq x.
\]
Now we will show the reverse inequality, i.e. that $x\leq y^{k}$, and this will imply $y^k=x$. Since $y=\LUB(S)$, $y$ is an upper bound for $S$, and so for all $n$, $y+\frac{1}{n}\not\in S$. By the definition of $S$, this means 
\[
\left(y+\frac{1}{n}\right)^{k}\geq x.
\]
Again, Proposition \ref{p:power-lim} and Proposition \ref{p:x_n<y_n}, together with the fact that $y + \frac{1}{n}\rightarrow y$ imply that
\[
y^{k} = \limn\left(y+\frac{1}{n}\right)^{k}\geq x.
\]
Thus, $y^k=x$. Finally, if $0 < y_1 < y_2$, then $y_1^k < y_2^k$ (see Proposition 4.3) and so there is a {\em unique} $y >0$ such that $y^k = x$.


\end{proof}



\section{Infinite limits}

\begin{definition}
Let $(x_{n})$ be a sequence of real numbers.
\begin{enumerate}[label=(\alph*)]
\item We say {\it $(x_{n})$ tends to $\infty$} or {\it diverges to} $\infty$ or $x_{n}\rightarrow \infty$ as $n \to \infty$ if, for all $M>0$, there is $N$ so that $n> N$ implies $x_{n}\geq M$. 
\item Similarly, we say {\it $x_{n}$ tends or diverges to $-\infty$} or $x_{n}\rightarrow -\infty$ as $n \to \infty$ if, for all $M<0$, there is $N$ so that $n>N$ implies $x_{n}\leq M$. 
\end{enumerate}
\end{definition}

\begin{example}
$\sqrt{n}\rightarrow\infty$: let $M>0$, we need to find $N$ so that $n> N$ implies $\sqrt{n}> M$. If we pick $N=M^2$, then for $n> N$,
\[
\sqrt{n}> \sqrt{N}=\sqrt{M^2}=M.
\]
Thus, we have shown that for any $M>0$ we can find $N$ so that $n> N$ implies $\sqrt{n}\geq M$, hence $\sqrt{n}\rightarrow\infty$.
\end{example}



\begin{example}
The sequence $x_{n} = n^2-n-1$ tends to infinity. To prove this, we must show that there is $N$ so that $n> N$ implies 
\[
n^2-n-1>M, 
\]
%We could use the quadratic formula to solve for such an $N$, but here is another way that could be used for other problems: it involves choosing $N$ large enough so that you can bound your sequence below by something simpler that you can then show is bigger than $M$. 

Note that  for $n>1$,
\[
n^2-n -1= n(n-1)-1\geq n\cdot 1-1=n-1
\]
Thus, if $N=M+1$, then $n> N$ implies 
\[
n^2-n-1 \geq n-1> N-1=M+1-1=M.
\]
\end{example}

%{\bf Note:} The symbol $\infty$ is {\it not} a number, so in particular, the limit rules we saw earlier do not hold for limits tending to infinity. If you ever catch yourself writing something like $\infty\cdot 3$, $\infty-2$, or $\infty^2$, you are writing something invalid, even if you may be getting the correct answer.


%
%\begin{protip}
%{\bf Simplifying sequences tending to infinity:} In the above exercise, we showed that $n^{2}-n+1$ tended to infinity by showing it was at least $\frac{n^{2}}{2}$ for $n$ large enough, and then it was easier to show that we could make $\frac{n^{2}}{2}$ as large as we wanted. In general, if you have a sequence you think tends to infinity of the form $a_n+b_n$ and you think that $a_n$ goes to infinity faster than $b_n$, try and first prove that there is $N_0$ so that $n\geq N_{0}$ implies $|b_{n}|<\frac{a_{n}}{2}$, because then 
%\[
%a_{n}+b_{n}\geq a_{n}-|b_{n}|>a_{n}-\frac{a_{n}}{2}=\frac{a_{n}}{2}
%\]
%and then try and show that there is $N\geq N_{0}$ so that $n\geq N$ implies $\frac{a_{n}}{2}\geq M$. 
%\end{protip}




\section{Exercises}




The exercises in Liebeck's book relevant to this section are in Chapter 23.

\begin{exercise}\label{one}
Show that $|x- L| < \epsilon$ if and only if $L - \epsilon < x < L+ \epsilon$.
\begin{solution}
 Suppose that $|x-L|<\epsilon$. Then $
x-L\leq |x-L|<\epsilon, \mbox{ and so } x < L + \epsilon;
$
moreover 
$
L-x \leq |L-x|=|x-L|<\epsilon
\mbox{ and so } x>L-\epsilon. 
$
Conversely, suppose that $L-\epsilon<x<L+\epsilon$. Then 
$ - \epsilon < x -L < \epsilon$
and so $
|x-L|<\epsilon.$
 \end{solution}
 \end{exercise}
 
 \begin{exercise}
 Using the $\epsilon-N$ definition of a limit, prove the following:
 
\begin{enumerate}[label=(\alph*)]
\item $\limn \frac{1}{n^{3}}=0$.
\begin{solution}
Let $\epsilon>0$. If $N=\epsilon^{-1/3}$, then for $n> N$,
\[
\left|\frac{1}{n^3}-0\right|=\frac{1}{n^3}< \frac{1}{N^3}=\frac{1}{ (\epsilon^{-1/3})^3}=\frac{1}{\epsilon^{-1}}=\epsilon.
\]
\end{solution}
\item $\limn \frac{n^{4}}{n^{4}+1}=1$
\begin{solution}
Let $\epsilon>0$. We want to find $N$ so that $n> N$ implies
\[
\left|\frac{n^{4}}{n^{4}+1}-1\right|=\frac{1}{n^4+1}<\epsilon.
\]
If $\epsilon\geq 1$, then this always holds so we can just pick $N=1$. If $\epsilon<1$, this is equivalent to saying $n^4+1>\epsilon^{-1}$. This will certainly hold if $n^4>\epsilon^{-1}$, i.e. when $n>\epsilon^{-\frac{1}{4}}$. Thus, the above inequality holds if $n>N$ where $N=\epsilon^{-\frac{1}{4}}$.
\end{solution}
\item $\limn \frac{n^{2}+3}{n^{2}+n}=1$
\begin{solution}
Let $\epsilon>0$, we want to find $N$ so that $n> N$ implies 
\[
\left|\frac{n^{2}+3}{n^{2}+n}-1\right|=\left|\frac{3-n}{n^2+n}\right|<\epsilon.
\]
Note that for $n> 3$, $n-3> 0$ and so 
\[
\left|\frac{3-n}{n^2+n}\right|=\frac{n-3}{n^2+n}<\frac{n}{n^2+n}=\frac{1}{n+1}.
\]
Thus, if we pick $N=\max\{3,\frac{1}{\epsilon}\}$, we see that $n> N$ implies 
\[
\left|\frac{n^{2}+3}{n^{2}+n}-1\right|=\frac{n-3}{n^2+n}<\frac{1}{n+1}< \frac{1}{N+1}\leq \frac{1}{\frac{1}{\epsilon}+1}<\frac{1}{1/\epsilon}=\epsilon.
\]
\end{solution}
\item $\limn \frac{2n^{2}-n}{n^{2}+n-1}=2$. 

\begin{solution}
Let $\epsilon>0$, we want to find $N$ so that $n> N$ implies
\[
\left|\frac{2n^{2}-n}{n^{2}+n-1}-2\right|
=\left|\frac{-3n+2}{n^2+n-1}\right|<\epsilon.
\] 
Since $n\geq 1$, $3n-2\geq 0$, so 
\[
\left|\frac{-3n+2}{n^2+n-1}\right|=\frac{3n-2}{n^2+n-1}
<\frac{3n}{n^2-1}=\frac{3n}{(n-1)(n+1)}<\frac{3n}{(n-1)n}=\frac{3}{n-1}.
\]
We want $N$ so that $n> N$ implies $\frac{3}{n-1}<\epsilon$, i.e. that $n>3\epsilon^{-1}+1$. Thus, if we pick $N=3\epsilon^{-1}+1$, then $n> N$ implies 
\[
\left|\frac{2n^{2}-n}{n^{2}+n-1}-2\right|=\frac{3n-2}{n^2+n-1}<\frac{3}{n-1}< \frac{3}{N-1} =\frac{3}{3\epsilon^{-1}+1-1}=\epsilon.
\]

\end{solution}

\end{enumerate}
 
 \end{exercise}
 
 
 \begin{exercise}
 For the following sequences, show that they tend to infinity. 
 
\begin{enumerate}[label=(\alph*)]
\item $5n-2$
\begin{solution}
Let $M>0$. Let $N>\frac{M+2}{5}$, then for $n> N$, 
\[
5n-2> 5N-2\geq 5\frac{M+2}{5}-2=M.
\]
\end{solution}
\item $n^{3}+1$
\begin{solution}
Note that $n^3+1>n^3$, so if $M>0$, we can pick $N=M^{\frac{1}{3}}$ and then $n> N$ will imply
\[
n^3+1>n^3> N^3=M.
\]
\end{solution}
\item $n^{4}-n+1$
\begin{solution}
Note that for $n>1$,
\[
n^4-n+1=n(n^3-1)+1\geq n+1
\]
and so if $N=M-1$, then for $n>N-1$, we have $n^4-n+1\geq n+1\geq N+1\geq M$.
\end{solution}
\item $n!$
\begin{solution}
Let $M>0$. If we set $N=M$, then $n> N$ implies
\[
n!\geq n> N=M.
\]
\end{solution}
\item $\sqrt{n}$
\begin{solution}
Let $M>0$ and $N=M^2$, then $n> N$ implies
\[
\sqrt{n}> \sqrt{N}=\sqrt{M^2}=M.
\]
\end{solution}
\end{enumerate}
\end{exercise}
 
\begin{exercise}
Show that the limit of a sequence is unique, that is, if $x_{n}\rightarrow L$ and $x_{n}\rightarrow M$, then $L=M$.
\begin{solution}
Suppose $L\neq M$. Let $\epsilon=\frac{|L-M|}{2}$. Since $x_{n}\rightarrow L$, there is $N_1$ so that $n> N_1$ implies $|x_{n}-L|<\epsilon$. Similarly, $x_{n}\rightarrow M$, there is $N_2$ so that $n> N_2$ implies $|x_{n}-L|<\epsilon$. Thus, for $n>N= \max\{N_1,N_2\}$,
\[
|L-M|=|L-x_{n}+x_{n}-M|\leq |L-x_{n}|+|x_{n}-M|<\frac{\epsilon}{2}+\frac{\epsilon}{2} = \epsilon = |L-M|
\]
which is a contradiction.
\end{solution}
\end{exercise}


 \begin{exercise}
For each $L$ below, find sequences $(x_{n})$ and $(y_{n})$ so that $x_{n}\rightarrow 0$, $y_{n}\rightarrow 0$ and $x_{n}\neq 0, y_n \neq 0$ for all $n$, and such that $\frac{x_{n}}{y_{n}}\rightarrow L$. 
\begin{enumerate}[label=(\alph*)]
\item $1$
\item $0$
\item $\infty$
\end{enumerate}
\begin{solution}
\begin{enumerate}[label=(\alph*)]
\item $x_n=y_n=\frac{1}{n}$.
\item $x_n=\frac{1}{n^2}, y_n=\frac{1}{n}$.
\item $x_n=\frac{1}{n}, y_n=\frac{1}{n^2}$.
\end{enumerate}
(Many other solutions are possible.)
\end{solution}
\end{exercise}

 \begin{exercise}
For each $L$ below, find sequences $(x_{n})$ and $(y_{n})$ so that $x_{n}\rightarrow 0$, $y_{n}\rightarrow \infty$ and $x_{n}y_{n}\rightarrow L$. 
\begin{enumerate}[label=(\alph*)]
\item $1$
\item $0$
\item $\infty$
\end{enumerate}
\begin{solution}
\begin{enumerate}[label=(\alph*)]
\item $x_n=y_n=n$.
\item $x_n=n, y_n=n^2$.
\item $x_n=n^2, y_n=n$.
\end{enumerate}
\end{solution}
\end{exercise}
 

 
 
\begin{exercise}
Show that if $x_n\rightarrow x$, then $|x_n|\rightarrow |x|$. 

\begin{solution}
Let $\epsilon>0$. Since $x_n\rightarrow x$, there is $N$ so that $n> N$ implies $|x_{n}-x|<\epsilon$. By the reverse triangle inequality, for $n> N$,
\[
||x_{n}|-|x||
\leq |x_{n}-x|<\epsilon.
\]
Thus, $|x_n|\rightarrow |x|$.
\end{solution}

\end{exercise}

\begin{exercise}\label{68}
Suppose that $(x_{n})$ converges. Show that  for any $k\in\mathbb{N}$, $\limn x_{n}= \limn x_{n+k}$. 

\begin{solution}
Let $L=\limn x_{n}$. We want to show $\limn x_{n+k}=L$, that is, for all $\epsilon>0$ there is $N$ so that $n> N$ implies $|x_{n+k}-L|<\epsilon$. Let $\epsilon>0$. Since $L=\limn x_{n}$, we know there is $N$ so that $n> N$ implies $|x_{n}-L|<\epsilon$. If $n> N$, then $n+k>N$, and so we also have $|x_{n+k}-L|<\epsilon$ for all $n> N$. This proves the claim.
\end{solution}
\end{exercise}




\begin{exercise}
Suppose that $(x_{n})$ converges. Show that $x_{n}-x_{n+1}\rightarrow 0$. 

\begin{solution}
Using the limit rules and the previous exercise, if $\limn =L$, then
\[
\limn(x_{n}-x_{n+1})
=\limn x_n - \limn x_{n+1} = L-L=0.
\]
\end{solution}
\end{exercise}
%
%\begin{exercise}
%Suppose $x_n\in \mathbb{N}$ for all $n$ and $x_n$ converges. Show that there is an integer $x,N\in\mathbb{N}$ so that $x_n=x$  for all $n\geq N$.
%
%
%\begin{solution}
%Let $x=\limn x_{n}$. Using the $\epsilon-N$ definition of a limit with $\epsilon=\frac{1}{2}$, we know that there is $N$ so that $n\geq N$ implies 
%\[
%|x_{n}-x|<\frac{1}{2}.
%\]
%We claim that $x_{n}=x_{N}$ for all $n\geq N$. If there was $n\geq N$ so that $x_{n}\neq x_{N}$, then since $x_{n},x_{N}\in \mathbb{N}$, $|x_{n}-x_{N}|\geq 1$, but then
%\[
%1\leq |x_{n}-x_{N}| |x_{n}-x+x-x_{N}|\leq |x_{n}-x|+|x-x_{N}|
%< \frac{1}{2}+\frac{1}{2}=1,
%\]
%which is a contradiction.
%\end{solution}
%\end{exercise}


\begin{exercise} Using the $\epsilon-N$ definition of a limit, prove that if $x_n>0$ and $x_n\rightarrow x>0$ and $x_n\neq x$ for all $n$, then $x_{n}^{\frac{1}{3}}\rightarrow x^{\frac{1}{3}}$. 

\begin{solution}
Note that
\[
|x^{\frac{1}{3}}-x_{n}^{\frac{1}{3}}|
=\left| \frac{ (x^{\frac{1}{3}}-x_{n}^{\frac{1}{3}})(x^{\frac{2}{3}}+x^{\frac{1}{3}}x_{n}^{\frac{1}{3}}+x_{n}^{\frac{2}{3}})}{x^{\frac{2}{3}}+x^{\frac{1}{3}}x_{n}^{\frac{1}{3}}+x_{n}^{\frac{2}{3}}} \right|
 = \frac{|x-x_{n}|}{x^{\frac{2}{3}}+x^{\frac{1}{3}}x_{n}^{\frac{1}{3}}+x_{n}^{\frac{2}{3}}}
 \leq \frac{|x-x_{n}|}{x^{\frac{2}{3}}}.\\
\]
Let $\epsilon>0$. Pick $N$ so that $n> N$ implies $|x-x_{n}|< \epsilon x^{\frac{2}{3}}$. Then the above implies that for $n> N$,
\[
|x^{\frac{1}{3}}-x_{n}^{\frac{1}{3}}|
\leq  \frac{|x-x_{n}|}{x^{\frac{2}{3}}}
<\frac{\epsilon x^{\frac{2}{3}}}{x^{\frac{2}{3}}}=\epsilon
\]
which thus proves $x_{n}^{\frac{1}{3}}\rightarrow x^{\frac{1}{3}}$. 
\end{solution}

\end{exercise}

\begin{exercise}
Let $a_n = \left( 1- \frac{1}{n^2}\right)^n$. Show that $a_n \to 1$ as $n \to \infty$. What happens if we consider $\left(1-\frac{1}{n^s}\right)^n$ for $s > 1$? ({\bf Hint:} Bernoulli's inequality.)
\begin{solution}
By Bernoulli's inequality with $ h= -1/n^2$ we have
\[1 \geq \left( 1- \frac{1}{n^2}\right)^n \geq 1 - \frac{n}{n^2}
= 1 - \frac{1}{n}.\] 
By the squeeze theorem we deduce that 
\[ \left( 1- \frac{1}{n^2}\right)^n \to 1\]
as $n \to \infty$. A similar  argument works for $\left(1-\frac{1}{n^s}\right)^n$ 
whenever $s > 1$. 
\end{solution}
\end{exercise}

\begin{exercise}
Let $a_n = \left( 1+ \frac{1}{n^2}\right)^n$. Show that $a_n \to 1$ as $n \to \infty$.
\end{exercise}
\begin{solution}
We have
\[ \left( 1+ \frac{1}{n^2}\right)^n\left( 1- \frac{1}{n^2}\right)^n=\left( 1- \frac{1}{n^4}\right)^n.\] 
Let $b_n = \left( 1- \frac{1}{n^2}\right)^n$ and $c_n = \left( 1- \frac{1}{n^4}\right)^n$ so that 
$a_n b_n = c_n$. But by the previous exercise, $b_n \to 1$ and $c_n \to 1$. Therefore by the rules for limits, $a_n \to 1$ too.

\end{solution}

\begin{exercise}
Let $a_n = \left( 1+ \frac{1}{n}\right)^n$. Given that $a_n \to L$ as $n \to \infty$, what can you say about
the sequence $(b_n)$ where
$b_n = \left( 1- \frac{1}{n}\right)^n$?
\begin{solution}
By the argument for the previous exercise, $b_n \to 1/L$ (since $L \geq 1$). 
\end{solution}
\end{exercise}


%
%
%\begin{exercise}
% Show that $|x_{n}+y_{n}|-|x_{n}-y_{n}|\rightarrow \infty$ if and only if $\lim_{n\rightarrow\infty} |x_{n}|=\lim_{n\rightarrow\infty} |y_{n}|=\lim_{n\rightarrow\infty} x_{n}y_{n} = \infty$. 
% \end{exercise}


\end{document}