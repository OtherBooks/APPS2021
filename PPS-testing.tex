%%%%%%%%%%%%%%%%%%%%%%%%%%%%%%%%%%%%%%%%%
% The Legrand Orange Book
% LaTeX Template
% Version 2.0 (9/2/15)
%
% This template has been downloaded from:
% http://www.LaTeXTemplates.com
%
% Mathias Legrand (legrand.mathias@gmail.com) with modifications by:
% Vel (vel@latextemplates.com)
%
% License:
% CC BY-NC-SA 3.0 (http://creativecommons.org/licenses/by-nc-sa/3.0/)
%
% Compiling this template:
% This template uses biber for its bibliography and makeindex for its index.
% When you first open the template, compile it from the command line with the 
% commands below to make sure your LaTeX distribution is configured correctly:
%
% 1) pdflatex main
% 2) makeindex main.idx -s StyleInd.ist
% 3) biber main
% 4) pdflatex main x 2
%
% After this, when you wish to update the bibliography/index use the appropriate
% command above and make sure to compile with pdflatex several times 
% afterwards to propagate your changes to the document.
%
% This template also uses a number of packages which may need to be
% updated to the newest versions for the template to compile. It is strongly
% recommended you update your LaTeX distribution if you have any
% compilation errors.
%
% Important note:
% Chapter heading images should have a 2:1 width:height ratio,
% e.g. 920px width and 460px height.
%
%%%%%%%%%%%%%%%%%%%%%%%%%%%%%%%%%%%%%%%%%

%----------------------------------------------------------------------------------------
%	PACKAGES AND OTHER DOCUMENT CONFIGURATIONS
%----------------------------------------------------------------------------------------

%\documentclass[11pt,fleqn,dvipsnames]{book} % Default font size and left-justified equations
\documentclass[11pt,dvipsnames]{book} 

%----------------------------------------------------------------------------------------

\input{structure} % Insert the commands.tex file which contains the majority of the structure behind the template



%%agregué




%%%My stuff


%\usepackage[utf8x]{inputenc}
\usepackage[T1]{fontenc}
\usepackage{tgpagella}
%\usepackage{due-dates}
\usepackage[small]{eulervm}
\usepackage{amsmath,amssymb,amstext,amsthm,amscd,mathrsfs,eucal,bm,xcolor}
\usepackage{multicol}
\usepackage{array,color,graphicx}
\usepackage{enumerate}


\usepackage{epigraph}
%\usepackage[colorlinks,citecolor=red,linkcolor=blue,pagebackref,hypertexnames=false]{hyperref}

%\theoremstyle{remark} 
%\newtheorem{definition}[theorem]{Definition}
%\newtheorem{example}[theorem]{\bf Example}
%\newtheorem*{solution}{Solution:}


\usepackage{centernot}


\usepackage{filecontents}

\begin{filecontents*}{MyPackage.sty}
\NeedsTeXFormat{LaTeX2e}
\ProvidesPackage{MyPackage}
\RequirePackage{environ}
\newif\if@hidden% \@hiddenfalse
\DeclareOption{hide}{\global\@hiddentrue}
\DeclareOption{unhide}{\global\@hiddenfalse}
\ProcessOptions\relax
\NewEnviron{solution}
  {\if@hidden\else {\bf Solution: }\BODY\fi}
\end{filecontents*}

%\usepackage[unhide]{MyPackage}
\usepackage[hide]{MyPackage}





\def\putgrid{\put(0,0){0}
\put(0,25){25}
\put(0,50){50}
\put(0,75){75}
\put(0,100){100}
\put(0,125){125}
\put(0,150){150}
\put(0,175){175}
\put(0,200){200}
\put(25,0){25}
\put(50,0){50}
\put(75,0){75}
\put(100,0){100}
\put(125,0){125}
\put(150,0){150}
\put(175,0){175}
\put(200,0){200}
\put(225,0){225}
\put(250,0){250}
\put(275,0){275}
\put(300,0){300}
\put(325,0){325}
\put(350,0){350}
\put(375,0){375}
\put(400,0){400}
{\color{gray}\multiput(0,0)(25,0){16}{\line(0,1){200}}}
{\color{gray}\multiput(0,0)(0,25){8}{\line(1,0){400}}}
}



%\usepackage{tikz}

%\pagestyle{headandfoot}
%\firstpageheader{\textbf{Proofs \& Problem Solving}}{\textbf{Homework 1}}{\textbf{\PSYear}}
%\runningheader{}{}{}
%\firstpagefooter{}{}{}
%\runningfooter{}{}{}

%\marksnotpoints
%\pointsinrightmargin
%\pointsdroppedatright
%\bracketedpoints
%\marginpointname{ \points}
%\totalformat{[\totalpoints~\points]}

\def\R{\mathbb{R}}
\def\Z{\mathbb{Z}}
\def\N{{\mathbb{N}}}
\def\Q{{\mathbb{Q}}}
\def\C{{\mathbb{C}}}
\def\hcf{{\rm hcf}}


%%end of my stuff


\usepackage[hang, small,labelfont=bf,up,textfont=it,up]{caption} % Custom captions under/above floats in tables or figures
\usepackage{booktabs} % Horizontal rules in tables
\usepackage{float} % Required for tables and figures in the multi-column environment - they




\usepackage{graphicx} % paquete que permite introducir imágenes

\usepackage{booktabs} % Horizontal rules in tables
\usepackage{float} % Required for tables and figures in the multi-column environment - they

\numberwithin{equation}{section} % Number equations within sections (i.e. 1.1, 1.2, 2.1, 2.2 instead of 1, 2, 3, 4)
\numberwithin{figure}{section} % Number figures within sections (i.e. 1.1, 1.2, 2.1, 2.2 instead of 1, 2, 3, 4)
\numberwithin{table}{section} % Number tables within sections (i.e. 1.1, 1.2, 2.1, 2.2 instead of 1, 2, 3, 4)


\setlength\parindent{0pt} % Removes all indentation from paragraphs - comment this line for an assignment with lots of text

%%hasta aquí


\begin{document}

%----------------------------------------------------------------------------------------
%	TITLE PAGE
%----------------------------------------------------------------------------------------

\begingroup
\thispagestyle{empty}
\begin{tikzpicture}[remember picture,overlay]
\coordinate [below=12cm] (midpoint) at (current page.north);
\node at (current page.north west)
{\begin{tikzpicture}[remember picture,overlay]
\node[anchor=north west,inner sep=0pt] at (0,0) {\includegraphics[width=\paperwidth]{Figures/penrose.pdf}}; % Background image
\draw[anchor=north] (midpoint) node [fill=ocre!30!white,fill opacity=0.6,text opacity=1,inner sep=1cm]{\Huge\centering\bfseries\sffamily\parbox[c][][t]{\paperwidth}{\centering Proofs and Problem Solving \\[15pt] % Book title
{\Large School of Mathematics}\\[20pt] % Subtitle
{\huge Jonas Azzam \\  based on Martin Liebeck's \\ \textit{A Concise Introduction to \\ Pure Mathematics}}}}; % Author name
\end{tikzpicture}};
\end{tikzpicture}
\vfill
\endgroup


%----------------------------------------------------------------------------------------
%	COPYRIGHT PAGE
%----------------------------------------------------------------------------------------

%\newpage
%~\vfill
%\thispagestyle{empty}

%\noindent Copyright \copyright\ 2013 John Smith\\ % Copyright notice

%\noindent \textsc{Published by Publisher}\\ % Publisher

%\noindent \textsc{book-website.com}\\ % URL

%\noindent Licensed under the Creative Commons Attribution-NonCommercial 3.0 Unported License (the ``License''). You may not use this file except in compliance with the License. You may obtain a copy of the License at \url{http://creativecommons.org/licenses/by-nc/3.0}. Unless required by applicable law or agreed to in writing, software distributed under the License is distributed on an \textsc{``as is'' basis, without warranties or conditions of any kind}, either express or implied. See the License for the specific language governing permissions and limitations under the License.\\ % License information

%\noindent \textit{First printing, March 2013} % Printing/edition date

%----------------------------------------------------------------------------------------
%	TABLE OF CONTENTS
%----------------------------------------------------------------------------------------

\chapterimage{Figures/dominoes.jpg} % Table of contents heading image

%\chapterimage{chapter_head_1.pdf} % Table of contents heading image

\pagestyle{empty} % No headers

 \tableofcontents % Print the table of contents itself

\cleardoublepage % Forces the first chapter to start on an odd page so it's on the right

\pagestyle{fancy} % Print headers again

%----------------------------------------------------------------------------------------
%	PART
%----------------------------------------------------------------------------------------



\part{Week 1: Logic and the Reals}


\chapterimage{Figures/blank.png} 



\chapter{Logic}

\setcounter{page}{1}






\part{Week 2: Induction}


\chapterimage{Figures/dominoes.jpg} 



\chapter{Induction}

\setcounter{page}{1}
The main topic of this chapter is the following axiom of mathematics:

\begin{theorem}[The Principle of mathematical induction]
Given a list of statements $P(k),P(k+1),...$, we know that $P(n)$ is true for every integer $n\geqslant k$ if

\begin{itemize}
\item we know that $P(k)$ is \textcolor[rgb]{1.00,0.00,0.50}{true}, (``base case")

\item and we can \textcolor[rgb]{0.50,0.00,0.50}{prove} that $P(n)\Rightarrow P(n+1)$ for any integer $n\geqslant k$. (``induction step")
\end{itemize}
\end{theorem}

\vspace{10pt}
By saying it is an axiom, this means that we take it as a rule (so it requires no proof). 

\begin{multicols}{2}
\begin{center}
 \includegraphics[height=.25\textwidth]{Figures/dominoes.jpg}
\end{center}
Induction is like a line of dominoes: To knock them down, make sure each one topples the next one, and then just topple the first! \\

To use induction in a proof, it is also a two step process. 
\begin{itemize}
\item {\bf Base  Case:} If the first statement in your list is $P(k)$, then prove that first.
\item {\bf Induction Step:} Prove $P(n)\Rightarrow P(n+1)$, that is, assuming $P(n)$ (we call this the {\it inductive hypothesis}, show that $P(n+1)$ must be true.
\end{itemize}

\end{multicols}

\begin{definition}
Something
\end{definition}




{\bf Proof Technique:} Finding $P(n)$ in $P(n+1)$
To prove the induction step $P(n)\Rightarrow P(n+1)$ in an induction proof, try and work backwards: Start looking at  $P(n+1)$ and try to find an occurrence where you can use the assumption of $P(n)$, or look for a copy of the statement $P(n)$ in the statement $P(n+1)$. 



\begin{example}
For $n\in\mathbb{N}$, $\sum_{k=1}^{n}k=\frac{n(n+1)}{2}$. 
\end{example}

\begin{proof}
Let $P(n)$ be the statement that $\sum_{k=1}^{n}k=\frac{n(n+1)}{2}$. 

\noindent {\bf Base Case:} First let's prove $P(1)$. This is just
\[
\sum_{k=1}^{1}k=1=\frac{1(1+1)}{2}. 
\]
Thus, the base case holds. 

\noindent {\bf Induction Step:} Assume $P(n)$ holds, we will show that $P(n+1)$ holds. Note that $\sum_{k=1}^{n+1}k$ contains a copy of $\sum_{k=1}^{n}k$ in it, so we can use our induction hypothesis (that is, we can use our assumption $P(n)$) on it:

\[
\sum_{k=1}^{n+1}k =n+1+ \sum_{k=1}^{n} k  = n+1 \frac{n(n+1)}{2} = \frac{(n+1)(n+2)}{2}.
\]
Since we have shown the base case and induction step, $P(n)$ is true for all $n$ by induction.
\end{proof}


\fbox{%
    \parbox{\textwidth}{{\bf Sidenote:} 
The above formula $\sum_{k=1}^{n}k=\frac{n(n+1)}{2}$ has been known for ages. It was shown, for example, by Abu Bakr Muhammad ibn al Hasan al-Karaji (c. 953-c. 1029), and he also found other formulae for $\sum_{k=1}^{n}k^2$ and $\sum_{k=1}^{n}k^3$. However, his proofs were not induction as we know today: he proved his formulas for the first 5 values of $n$ and then said the other statements could be proven similarly. 

There is a famous story of Carl Friedrich Gauss (1777-1855) giving another proof of this formula when he was 10: His teacher one day asked students to sum up all integers from $1$ to $100$, and when they were done they were to bring their slate up to the teacher. In a few seconds, Gauss took his slate up to the desk. He explained to his teacher that he figured it out so quickly because he realized that 
\[
1+2+3+\cdots + 100 = (1+100)+(2+99)+(3+98)+\cdots (50+51) = 50\cdot 101.\]

}}


Some statements may not be true for all $n\geq 1$ but maybe for $n\geq k$ for some $k$, and depending on the problem you will need to figure out what this $k$ is. FIX HERE!

\vspace{10pt}
When should you use induction? There are a couple of things to look out for:

\begin{multicols}{2}
{\bf  Induction is very useful when:}

\begin{itemize}
\item A statement has a nested structure: each statement builds off the last.
\item A statement involves a clear multi-step algorithm.
\item A statement involves an identity for a complicated sum.
\end{itemize}

{\bf Induction is difficult when:}

\begin{itemize}
\item The statement for $n$ is not so clearly related to the statement for $n-1$.
\item There is no clear base case.
\item For sums, when you don't have a guess for a general answer.
\end{itemize}
\end{multicols}

So you should:

%\begin{itemize}
%\item See if it is a nested or recursive type statement.
%\item Try and work backwards.  Start from $P(n+1)$ and try to find an occurrence where you can use the assumption of $P(n)$.
%\item If working out an identity for sums, you have to guess at the general form.
%\end{itemize}




% 
% 
%\frametitle{Tower of Hanoi}
%\begin{center}
% \includegraphics[height=1.25in]{Tower-of-Hanoi.jpg}
%\end{center}
%{\bf Rules:} 
%\begin{itemize}
%\item We start $n$ disks with holes in the center, stacked on one rod from largest to smallest (with largest on bottom).
%\item The objective is to move the disks from the first rod to the third.
%\item You can only move one disk at a time.
%\item You can place a disk on top of any other larger disk (i.e. no disk can be under a larger disk). 
%\end{itemize}
%
%
%
%





\section{Strong Mathematical Induction}
\index{Strong Induction}
Sometimes it's not enough to know that $P(n)$ is true to prove $P(n+1)$. 
\ 
\begin{theorem}[Principle of Strong Mathematical Induction]
  Let $k\in \mathbb{N}$ and let $P(k),P(k+1),...$ are statements. Suppose that\ 
  \begin{itemize}
  \item $P(k)$ is true,  (``base case")\ 
  \item for any integer
$n\geqslant k$:   $$ P(k),P(k+1),P(k+2),\ldots,P(n)\Longrightarrow
P(n+1).$$ (``induction step"). \ 

\end{itemize} 

  Then $P(n)$ is true for every positive integer $n\geqslant
k$.
\end{theorem}


You need to use strong induction when you need several previous statements to prove the next one. For example:

\begin{theorem}
Every integer $n\geq 2$ can be written as a product of primes $n=p_{1}\cdots p_{k}$. 
\end{theorem}

\begin{remark}
If $n$ is prime, this statement still makes sense: we just interpret $n$ as being the product of just one number, $n$ itself. 
\end{remark}

\begin{proof}
We prove by strong induction on $n$. The case $n=2$ immediately holds (taking into account the previous remark). For the induction step, suppose the theorem holds for all integers $n<N$.  If $N$ is prime, there is nothing to prove; otherwise, if $N$ is not prime, then $N=a\cdot b$ for some positive integers $a$ and $b$ greater than $1$. Both $a$ and $b$ can be  decomposed by the strong induction hypothesis, thus so can $n=ab$.
\end{proof}



\section{Recurrence Relations}


Another case when you need strong induction is when studying recurrence relations. A {\it recurrence} relation is a sequence $a_{1},a_{2},...$ of numbers  where each $a_{n}$ is defined in terms of previous terms in the sequence. The most famous recurrence relation is the {\it Fibonacci sequence}, which is defined  as $f_{1}=1$, $f_{2}=1$ and for $n>2$, $f_{n}=f_{n-1}+f_{n-2}$. 

For a recurrence relation $a_{n}$, if we would like to prove a statement about it for all $n$, then of course we will need induction. However, we need to take care. 

Let's consider a simple example with the Fibonacci sequence.

\begin{example}
$\frac{f_{n+1}}{f_{n}}$ is increasing.
\end{example}



\begin{example}
Let $\alpha=\frac{1+\sqrt{5}}{2}$ and $\beta=\frac{1-\sqrt{5}}{2}$. Then 
\[
f_{n} = \alpha^{n}+\beta^{n} \;\; \mbox{ for } n\geq 1.
\]
\begin{proof}
Note that 
\end{proof}
\end{example}



\section{Exercises}





\begin{exercise}
 Find a formula for the number of diagonals in a convex polygon with $n\geq 3$ vertices. 
 \end{exercise}

\begin{solution}
Let $f(n)$ denote the number of diagonals in a convex polygon of $n$ vertices. We will find a recurrence relation that $f$ satisfies. 

Suppose we have a convex  polygon with $n+1$ vertices. Draw an edge between the second vertex $a$ and last vertex $c$, so splits the polygon into a triangle between $a$, $c$, and the first vertex $c$ and a polygon with $n$ vertices. The number of diagonals in this polygon (which are diagonals for the original polygon) are $f(n)$ many. The ones that are diagonals for the original polygon that weren't counted are the ones that have $b$ as an endpoint (of which there are $n+1-3=n-2$ many) and the edge that we first drew. Thus the total is
\begin{align*}
f(n+1)
& =f(n)+n-2+1=f(n)+n-1  = f(n-1) + n-2+n-1=\cdots \\
& = f(3) + \sum_{k=2}^{n-1}k
=0+\sum_{k=1}^{n-1}k-1 = \frac{n(n-1)}{2}-1. 
\end{align*}

\end{solution}


\begin{exercise}
Suppose two players play a game, where they are given $N$ stones and each player takes turns removing up to 4 stones from the pile (so they can remove 1,2,3, or 4 stones, and they have to take at least 1). The winner is the person who removes the last stone. Conjecture and prove a rule, depending on $N$, that determines whether the first player or second player has a winning strategy.
\end{exercise}

\begin{solution}
If $N=5n$ for some integer $n\geq 1$, then player 2 always wins. We prove this by induction. If $n=1$, then regardless of how many stones Player 1 takes, Player 2 can take the rest and win. Now suppose we know that player 2 always wins if $N=5n$ for some integer $n$. If $N=5(n+1)$, and player 1 takes away 1,2,3, or 4 stones, then player 2 can take away 4,3,2, or 1 stones so that there are only 5N left. The induction hypothesis says that, now that it is player 1's turn, player 2 has a winning strategy. 

If $n=5n+j$ for some $j\in \{1,2,3,4\}$, then player 1 has a winning strategy. Indeed, player 1 needs to remove j stones so that there are $5n$, now the first part of the problem says that player 1 (who now has second move since it is player 2's turn) has a winning strategy. 

\end{solution}


\begin{exercise}
Let $p_{1},...$ be the primes in order, so $p_1=2$, $p_2=3$,.... Show that for all $n\geq 1$,
\[
\sum_{k=1}^{n}\frac{1}{p_{k}}
\]
is not an integer. (Hint: try working out a few values. What do you notice about the fractions you get?)
\end{exercise}

\begin{solution}
We claim that $\sum_{k=1}^{n}\frac{1}{p_{k}}$ is always an odd number divided by an even number. Clearly, the $n=1$ case is true since the sum is just $1$. Now suppose the statement is true for some integer $n$, so $\sum_{k=1}^{n}\frac{1}{p_{k}}=\frac{a}{b}$ where $a$ is odd and $b$ is even. Then
\[
\sum_{k=1}^{n+1}\frac{1}{p_{k}} = \frac{1}{p_{n+1}} +\sum_{k=1}^{n}\frac{1}{p_{k}} = \frac{1}{p_{n+1}}+\frac{a}{b} = \frac{p_{n+1}a+b}{p_{n+1}b}.
\]
Since $b$ is even, so is $p_{n+1}b$, and since $p_{n+1}$ and $a$ are odd, $p_{n+1}a+b$ is odd. This proves the claim. 
\end{solution}




\begin{exercise}
Suppose $x_{1}=1$ and $x_{n+1} = \sqrt{1+2x_{n}}$ for all $n\geq 1$. Show $x_{n}\leq 4$ for all $n$. \\
\end{exercise}


\begin{solution}
This clearly holds for $n=1$. Assume $x_{n}\leq 4$, then 
\[
x_{n+1}= \sqrt{1+2\cdot x_{n}}\leq \sqrt{1+2\cdot 4}=\sqrt{9}=3<4\]
and thus the induction step also holds. 
\end{solution}







\begin{exercise}  (Aug 2018 Exam) Show that, for all $n\geq 0$, $\sum_{k=1}^{n} k\cdot k! = (n+1)!-1$. 
\end{exercise}
\begin{solution}
The base case is $\sum_{k=1}^{1} k\cdot k!=1=2!-1$. For the induction step, assuming the $(n-1)$st case is true,
\[
\sum_{k=1}^{n} k\cdot k! 
=\sum_{k=1}^{n-1} k\cdot k! + n\cdot n!
=n!-1+ n\cdot n!
=(n+1)\cdot n! -1 = (n+1)!-1.
\]
\end{solution}


\begin{exercise} Show that $n!\leq n^{n}$ for all $n\geq 1$
\end{exercise}

\begin{solution}
The base holds immediately. Assume $n\geq 1$ is so that $n^{n}\geq n!$. Then
\[
(n+1)!=n! (n+1)\leq n^{n} (n+1)<(n+1)^{n}(n+1)=(n+1)^{n+1}.
\]
\end{solution}



\begin{exercise} Show that for all $n\in\mathbb{N}$, there are {\it distinct} integers $x_{1},...,x_{n}$ so that 
\[
\sum_{k=1}^{n} \frac{1}{x_{k}}=1.
\]
***It is an open question whether you can always do this with $x_{n}$ distinct {\it odd} numbers.

\end{exercise}


\begin{solution}
We prove this by induction. The base case $n=1$ is clear. Suppose the claim is true for some integer $n$. Let $x_{1},...,x_{n}$ be so that 
\[
\sum_{k=1}^{n} \frac{1}{x_{k}}=1.
\]
Note that we must have $x_{k}>1$ for all $k$. Then 
\[
\sum_{k=1}^{n} \frac{1}{2x_{k}}=\frac{1}{2}.
\]
Hence,
\[
\frac{1}{2} + \frac{1}{2x_{1}}+\cdots + \frac{1}{2x_{n}}=\frac{1}{2}+\frac{1}{2} =1,
\]
Note that as $x_{k}\geq 2$ for all $k$, $2x_{k}\neq 2$ for all $k$, thus the integers $2,2x_{1},...,2x_{n}$ are $n+1$ distinct integers whose recipricals add up to one, which proves the induction step and hence the claim. 

\end{solution}

\begin{exercise}  Let $f_{1}=f_{2}=1$ and $f_{n}=f_{n-1}+f_{n-2}$ for $n>2$. 
\begin{itemize}
\item Show that $3|f_{4n}$ for all $n\geq 1$. 
\item Show that $1\leq f_{n+1}/f_{n}\leq 2$ for all $n\geq 1$. 
\end{itemize}
\end{exercise}

\begin{solution}
\begin{itemize}
\item We prove by induction. We can compute that $f_{4}=3$, which establishes the base case $n=1$. Now assume $3|f_{4n}$ for some $n\geq 1$, we'll show $3|f_{4(n+1)}$:

\[
f_{4(n+1)}=f_{4n+3}+f_{4n+2}
=2f_{4n+2}+f_{4n+1}
=2(f_{4n+1}+f_{4n})+f_{4n+1}
=3f_{4n+1}+2f_{4n}.
\]
By assumption, $3|f_{4n}$, and clearly $3|3f_{4n+1}$, and so $3|f_{4(n+1)}$, which proves the induction step and hence the claim. 

\item It is clear the inequalities hold for $n=1$. Assume we have shown it to hold for some $n$. Then
\[
\frac{f_{n+2}}{f_{n+1}}=1+\frac{f_{n}}{f_{n}}\leq 1+1=2
\]
and 
\[
\frac{f_{n+2}}{f_{n+1}}=1+\frac{f_{n}}{f_{n}}\geq 1+0=1.\]
This proves the induction step and hence the claim. 
\end{itemize}
\end{solution}


%Exam 2019
%\item Let $u_{n}$ be the sequence defined by $u_{1}=u_{2}=1$ and for $n>2$, $u_{n+1}=u_{n}+u_{n-1}$ (these are the Fibonacci numbers). Show that $u_{n}^2=u_{n-1}u_{n+1}+(-1)^{n-1}$ for all integers $n\geq 2$. 
%
%\begin{solution}
%We prove by induction. The base cases $n=2$ and $n=3$ can easily be established, so let $n>2$ and assume we have verified that $u_{k}^2=u_{k-1}u_{k+1}+(-1)^{n-1}$ for all $k\leq n$. Then
%\begin{align*}
%u_{n+1}^2 
%& = (u_{n+2}-u_{n})(u_{n}+u_{n-1})\\
%& =u_{n+2}u_{n}+u_{n+2}u_{n-1}-u_{n}^2-u_{n}u_{n-1}\\
%& = u_{n+2}u_{n}+u_{n+2}u_{n-1}-(u_{n+1}u_{n}+(-1)^{n-1})-u_{n}u_{n-1}\\
%& = u_{n+2}u_{n}+(u_{n+1}+u_{n})u_{n-1}-u_{n+1}u_{n}+(-1)^{n}-u_{n}u_{n-1}\\
%& =  u_{n+2}u_{n}+u_{n+1}u_{n-1}+u_{n}u_{n-1}-u_{n+1}u_{n}+(-1)^{n}-u_{n}u_{n-1}\\
%& = u_{n+2}u_{n}+(-1)^{n}.
%\end{align*}
%This proves the induction step and hence the claim.
%\end{solution}


\begin{exercise} Joseph Bertrand conjectured in 1845 (and it was proven by Chebychev in 1852) that, if $p_{1},p_{2},...$ are the primes in ascending order, then
\[
p_{n+1}<2p_{n}.
\]
This is called {\bf Bertrand's postulate}. Using this result, show that for all $n\geq 4$ we have 
\[
p_{n} <\sum_{k=1}^{n-1} p_{k}.
\]

\end{exercise}

\begin{solution}

We prove this by induction. For $n=4$, we have 
\[
p_{4}=7 < 2+3+5 = \sum_{k=1}^{3} p_{k}.
\]
This establishes the base case. Now assume the statement is true for some integer $n$. Then
\[
p_{n+1}<2p_{n} = p_{n}+p_{n} < p_{n}+\sum_{k=1}^{n-1} p_{k} = \sum_{k=1}^{n} p_{k}.
\]
This proves the induction step and hence the claim. 



\end{solution}


\begin{exercise} Let $F_{n}=2^{2^{n}}+1$ for $n\geq 0$. Prove that for $n>0$,
\[
F_{n} = F_{n-1}\cdots F_{0}+2.
\]
\end{exercise}

\begin{solution}
We prove by induction. The base case $n=1$ is immediate. Suppose the statement is true for some $n\geq 1$. Then
\[
F_{n}\cdot F_{n-1}\cdots F_{0}+2
= F_{n}(F_{n}-2)+2 
=F_{n}^{2}-2F_{n}+2
=(F_{n}-1)^2+1
=(2^{2^{n}})^2+1 = 2^{2^{n+1}}+1 =F_{n+1}.
\]

\end{solution}







\part{Week 4: Complex Numbers and Polynomials}

\chapterimage{Figures/mandelbrot.jpg} 




\chapter{Complex Numbers}




\begin{multicols}{2}
\epigraph{\it     To divide 10 in two parts, the product of which is 40....It is clear that this case is impossible. Nevertheless, we shall work thus...
%: We divide 10 into two equal parts, making each 5. These we square, making 25. Subtract 40, if you will, from the 25 thus produced, as I showed you in the chapter on operations in the sixth book leaving a remainder of -15, the square root of which added to or subtracted from 5 gives parts the product of which is 40. These will be $5 + \sqrt{- 15}$ and $5 - \sqrt{-15}$. 
Dismissing mental tortures, and multiplying $5 + \sqrt{- 15}$ by $5 - \sqrt{-15}$, we obtain $25 - (-15)$. Therefore the product is $40$...and thus far does arithmetical subtlety go, of which this, the extreme, is, as I have said, so subtle that it is useless.}{Cardan, {\it Ars Magna}, 1545} 

\epigraph{\it For any equation one can imagine as many roots [as its degree would suggest], but in many cases no quantity exists which corresponds  to  what  one  imagines.}{Rene Descartes, {\it Discours de la M\'{e}thode Pour bien conduire sa raison, et chercher la v\'{e}rit\'{e} dans les sciences}, 1637} 
\end{multicols}


The first of the above quotes comes from the inception of complex numbers. Here, Cardan is trying to find two numbers $x$ and $y$ so that $x+y=10$ but $xy=40$. This ends up being possible if you allow for taking square roots of negative numbers, since then $x=5+\sqrt{-15}$ and $y=5-\sqrt{-15}$ solves these equations. Mathematicians like Cardan considered these clever sophisms, but it was in this way did complex numbers come about: mathematicians felt like there {\it should} be a solution to some equation; such a solution was technically impossible, but if they just assumed for the sake of argument that $\sqrt{-1}$ existed, then you {\it could} find a solution. Descartes called these numbers {\it imaginary}. It turns out that they aren't just sophisms, but they are fundamental to mathematics and physics. \\

%At this time, Italian mathematicians like Cardan and Tartaglia were researching how to solve polynomial equations. (In this day, there were even contests for solving cubic equations, for which Tartaglia achieved his fame). 




%Given $z=a+bi \in \mathbb{C}$, with $a,b \in \mathbb{R}$ we define
%\[ \mathrm{Re}(z)=a \qquad \mathrm{Im}(z)=b \]
%to be the real and imaginary parts of $z$ respectively. A complex number written in this way (as $z=x+iy$) is said to be in {\em Cartesian form}.
%
%We can think of $(\mathrm{Re}(z),\mathrm{Im}(z))$ as a point on a plane using Cartesian coordinates. 



\section{Complex Numbers}


\begin{definition}[Complex Numbers]
\begin{multicols}{2}
Define $i$ to be a number such that $i^2=-1$. The {\it complex numbers} are any number of the form 
\[
z=x+iy\mbox{ where }x,y \in \mathbb{R}.\]
The set of all complex numbers is denoted $\mathbb{C}$. We define $\mathrm{Re}(z)=x$ and $\mathrm{Im}(z)=y$ to be the {\it real} and {\it imaginary parts} of $z$. 

\includegraphics[width=150pt]{Figures/cartesian.pdf}
\begin{picture}(0,0)(150,0)
\put(-5,60){$iy$}
\put(80,5){$x$}
\put(85,65){$z=x+iy$}
\end{picture}

\end{multicols}

\end{definition}


Multiplication and addition is exactly how you think it would work. For example:
\[
(5+3i) + (3+4i)=8+7i\]
and
\[
(1+3i)(2+i)=1\cdot 2 + 1\cdot i + 3i\cdot 2 + 3i\cdot i = 2+i+6i-3=-1+7i.
\]
For an integer $n$, we write  $z^{n}=z\cdot z\cdots z$ as we did with real numbers and similarly $z^{-1}=\frac{1}{z}$.\\

What if we want to do something more complicating, like $\frac{3+i}{1+i}$? How do we write it as $a+ib$? Before we tackle this question, there are a couple of important quantities related to a complex number that will make this task easier. \\


The first is the {\it modulus}:

\begin{definition}[Modulus]
\begin{multicols}{2}
Given a complex number $z=x+iy$, the {\it modulus} of $z$ is defined to be
\[
|z|=|x+iy|=\sqrt{x^{2}+y^{2}}.
\]

Geometrically, the modulus of $z$ is its distance from the origin, which is the length of the hypotenuse of length $x$ and height $y$, computed using the Pythagorean theorem.

\includegraphics[width=150pt]{Figures/modulus.pdf}
\begin{picture}(0,0)(150,0)
\put(-5,60){$iy$}
\put(80,5){$x$}
\put(85,65){$z=x+iy$}
\put(25,35){$|z|$}
\end{picture}

\end{multicols}

\end{definition}

This should remind you of the vector norm that you learned in ILA. Recall that for a vector $(x,y)$, its norm was $||(x,y)|| = \sqrt{x^2+y^2}$ as well. 

A second important quantity is the conjugate. 

\begin{definition}[Complex Conjugate] 
\begin{multicols}{2}
Given a complex number $z=x+iy$, its {\em complex conjugate} is defined to be 

\[\overline{z}=x-yi.
\]
Geometrically, this is the complex number obtained by reflecting $z$  across the `$x$-axis'. \\

\includegraphics[width=.40\textwidth]{Figures/conjugate.png}

\end{multicols}

\end{definition}



One useful thing about a conjugate is that it is a number we can multiply a complex number by to make it real:

$$z\overline{z} = (a+bi)(a-bi) = a^2 + b^2 = |z|^2.$$



Let's look at $\frac{3+i}{1+i}$ again. We can make the denominator real by mutliplying and dividing by the conjugate of $1+i$ (which is $\overline{1+i}=1-i$):

$$\frac{3+i}{1+i}=\frac{3+i}{1+i}\frac{1-i}{1-i} = \frac{(3+i)(1-i)}{(1+i)(1-i)} = \frac{5-2i}{2}=\frac{5}{2}-i.
$$

In general, we have

\begin{equation}
\label{e:1/z}
 \frac{z}{w} = \frac{z}{w}\frac{\overline w}{\overline w} = \frac{z\overline w}{w\overline w} = \frac{z\overline w}{|w|^2}, \;\; \frac{1}{z} = \frac{\overline{z}}{|z|^2}
 \end{equation}


Here are some other properties that we will use throughout the chapter.

\begin{lemma}
For all $u,v\in \mathbb{C}$:
\begin{itemize}
\item $\overline{u+v} = \overline{u}+\overline{v}$.
\item $\overline{uv} = \overline{u}\cdot\overline{v}$.
\item $|uv|=|u|\cdot|v|$.
\end{itemize}
\end{lemma}

We leave the proof as an exercise.\\


\begin{solution}

\begin{proof}
These each follow by direct computation.  Let us denote $u=a+bi$, $v=c+di$, for $a,b,c,d\in\mathbb{R}$.
\begin{itemize}
\item We have:
$$\overline{u+v} = \overline{(a+c) + (b+d)i} = a+c - (b+d)i = (a-bi) + (c-di) = \overline{u} + \overline{v}.$$
\item We have:
\begin{align*}
\overline{u\cdot v} 
& = \overline{(a+bi)(c+di)}\\
&  = \overline{(ac-bd) + (bc+ad)i} \\
& = ac-bd - (bc+ad)i \\
& = (a-bi)(c-di) \\
& = \overline{u}\cdot\overline{v}.
\end{align*}

\item We have:
$$|uv|^2 = (uv)(\overline{u}\overline{v}) = u\overline{u}v\overline{v} = |u|^2\cdot|v|^2,$$
where we have used (b) and (c) above.  Since both sides of the final equality are non-negatives, the inequality holds if, and only if, it holds after taking square roots (as we showed in Chapter 5, on inequalities)
\end{itemize}
Hence, we have established each of the four equalities.\end{proof}
\end{solution}









\section{Cartesian and Polar form}

There are a couple of ways of representing complex numbers, and you can think of each representation as a set of directions to give someone in order to reach a certain complex number. When we write $z$ as $z=x+iy$, we say $x+iy$ is $z$ in {\it Cartesian form}. The reason for calling this Cartesian form is that we are essentially representing a complex number $z=x+iy$ as a vector whose Cartesian coordinates are $(\Re(z),\Im(z))=(x,y)$. This gives you  a set of directions for reaching $z$ from $0$: we travel $x$ distance to the right and $y$ distance upward.

Another useful representation is {\it polar form}, where instead we represent a complex number by specifying what direction we want to travel in, given by an angle $\theta$ from the $x$ axis, and then a distance we want to travel in that direction, namely, the distance from $0$ to $z$ which, recall, is $|z|$. 



Now we define the polar form of $z=a+ib$ explicitly: 

\begin{definition}
\begin{multicols}{2}
Let $z\neq 0$ be a complex number. Let $r = |z|$ and let {\em argument} of $z$ is the angle $\theta\in [0,2\pi)$ between the line from $0$ to $z$ and the positive $x$-axis. 
We can then write
$$z= r(\cos\theta+i\sin\theta).$$

This is the {\it polar form} of $z$.

\includegraphics[width=150pt]{Figures/polar.pdf}
\begin{picture}(0,0)(150,0)
\put(87,65){$z$}
\put(35,40){$r$}
\put(35,20){$\theta$}
\put(85,30){$r\sin \theta$}
\put(25,0){$r\cos\theta$}
\end{picture}


\end{multicols}
\end{definition}

This representation can be done using geometry: if we look at the triangle in the above figure, the hypotenuse is $r=|z|$, and so the base and height of the triangle are $r\cos \theta$ and $r\sin\theta$ respectively, and so these give the real and imaginary parts (i.e. the $x$ and $y$ coordinates) of $z$. \\

\begin{remark}
Note that while we specify that the argument $\theta$ of a complex number $z$ to be in $[0,2\pi)$, there are infinitely many numbers $\phi$ so that $z=re^{i\phi}$ (where $r=|z|$): we can just take $\phi = \theta+ 2n\pi$ for $n\in\mathbb{Z}$. However, we define the argument in this way since there is {\it exactly} one $\theta\in [0,2\pi)$ for which $z=re^{i\theta}$.
\end{remark}


We will often write $z$ in {\em exponential form}
\[z= re^{i\theta} = r(\cos \theta + i \sin \theta).\]



This may seem odd, what does the exponential have to do with anything? We won't go too much into this, but if you are studying Calculus and Taylor Series, you can show that $e^{i\theta} = \cos \theta + i \sin \theta$ by plugging $i\theta$ into the Taylor series for $e^{x}$ and then you can split the series into $\cos \theta + i\sin\theta$. However, we will not need this fact. As far as we are concerned in this course, $e^{i\theta}$ is just notation for $\cos \theta + i \sin \theta$. However, we will see in the theorem below that $e^{i\theta}$ still behaves how you would expect it to when performing multiplication:





\begin{theorem}[De Moivre's Theorem (Multiplication in polar form)]
Let $z=re^{i\theta}$ and $w=se^{i\phi}$ then $zw = rs e^{i(\theta + \phi)}$.
\end{theorem}

\begin{proof}
In short, multiply the moduli and use the angle formulas for sine and cosine:

\begin{align*}
zw & = r(\cos \theta + i \sin \theta)\times  s(\cos \phi + i \sin \phi) \\
& = rs ( \cos \theta \cos \phi - \sin \theta \sin \phi)+ rsi (\cos \theta \sin \phi + \sin \theta \cos \phi) \\ 
                                                                                               & = rs (\cos(\theta + \phi) + i\sin(\theta + \phi))= rs e^{i(\theta + \phi)}
\end{align*}
\end{proof}



\begin{multicols}{2}

Geometrically, this says that when we multiply a complex number by $z$, it scales the distance from $0$ by a factor of $|z|$ and rotates anti-clockwise by $\mathrm{arg}(z)$.

{\bf Example:} Since $i=e^{i\frac{\pi}{2}}$, mutliplying $z$ by $i$ corresponds to rotating counter clockwise by $\frac{\pi}{2}$.

\begin{center}
\includegraphics[width=0.5\textwidth]{Figures/mult}
\end{center}
\end{multicols}
 
\vspace{10pt}


\begin{theorem}[A very special case]
If we let $z = r(\cos \theta + i \sin \theta)$, and $n \in \mathbb{N}$ then 
\begin{align*}
z^n = &r^{n} (\cos n\theta + i \sin n\theta) \\
z^{-n} = &r^{-n} (\cos (-n\theta) + i \sin (-n\theta)) = \frac{1}{r^n} (\cos (n\theta) - i \sin (n\theta)) 
\end{align*}
\end{theorem}

\begin{proof}
We prove the first statement by induction: For $n=2$, by De Moivre's Theorem, $z^2=r^2e^{2i\theta} = r^2(\cos 2\theta+i\sin 2\theta)$. For the induction step, assume the theorem holds for some integer $n\geq 1$. Then by the induction hypothesis and then using De Moivre's Theorem,
\[
z^{n+1} = z^{n}z=r^{n}e^{in\theta} re^{in\theta} = r^{n+1}e^{i(n+1)\theta},
\]
This proves the first equation. 

For the second part, we just note that by \eqref{e:1/z},
\[
\frac{1}{\cos n \theta +i\sin n \theta} =\cos n \theta - i\sin n \theta,\]
and so
\[
z^{-n}=(z^{n})^{-1} = (r^{n}(\cos n\theta + i\sin n\theta))^{-1} = r^{-n} (\cos n \theta - i\sin n \theta) = r^{-n} e^{-i n\theta}.\]

\end{proof}

This makes taking a power of a complex number much easier if we know it's polar form. 

\begin{example}
What is $(1+i)^{6}$?
\end{example}

We could multiply this out by hand, but we'll use polar coordinates instead: To write $1+i$ in polar form $re^{i\theta}$, we first find $r$:
\[
r=|1+i|=\sqrt{1^2+1^2}=\sqrt{2}.
\]
Then 
\[
e^{i\theta} = \frac{1+i}{r} =\frac{1+i}{\sqrt{2}} = \frac{1}{\sqrt{2}}+i\frac{1}{\sqrt{2}}=\cos \frac{\pi}{4} + i\sin\frac{\pi}{4}.
\]
Finally,
\[
(1+i)^{6}= (\sqrt{2})^{6} e^{i\frac{5\pi}{4}} = 8\left(\cos \frac{5\pi}{4}+i\sin \frac{5\pi}{4}\right) = -8-8i.
\]



\section{Roots of Unity}

\begin{center}
{\it How do we find all solutions to $z^{n}=1$?
}
\end{center}
We now have enough tools in place to answer this. We find them all in a few steps:

\begin{itemize}
\item First, using polar coordinates, we can find one root rather easily: $w=e^{\frac{2\pi i}{n}}$, since then 
\[
w^{n} = e^{\frac{2\pi i}{n}\cdot n}=e^{2\pi i}=1.\]
\item In particular, this means that any power of $w$ is also a root, since if $j\in\mathbb{N}$, 
\[
(w^{j} )^{n} = (w^{n})^{j}=1.
\]
\item Finally, we will show later that for any degree $n$ polynomial there are at most $n$ distinct roots. So if we show that the numbers
\[
1,w,w^{2},...,w^{n-1}
\]
are all distinct, then we will have all the roots. \\

\item Suppose for the sake of a contradiction that there are $0\leq j,k<n$ so that $w^{j}=w^{k}$. Then
\[
1=w^{j-k}=e^{(j-k)\frac{2\pi i}{n}} = \cos \left((j-k)\frac{2\pi }{n}\right)+i\sin \left((j-k)\frac{2\pi }{n}\right).\]
The only way this can be $1$ is if the cosine is 1 and the sine is zero, so which only happens if their arguments are multiples of $2\pi$, that is, we must have 
\[
(j-k)\frac{2\pi i}{n} = 2\pi k \mbox{ for some integer }k\]
which implies $j-k=nk$ for some $k$, but this is impossible since $0\leq j,k<n$, and so $-n<i-j<n$. Thus, the numbers $1,w,...,w^{n-1}$ are distinct and form all the roots.
\end{itemize}

We have thus shown the following.

\begin{theorem}[Roots of Unity]
The solutions to $z^{n}=1$ are $1,w,\cdots w^{n-1}$ where $w=e^{\frac{2\pi i}{n}}$. That is, they are $e^{\frac{2\pi k i}{n}}$ for $k=0,1,...,n-1$.
\end{theorem}







\begin{example}[Third roots of unity]

The third roots of unity are
\begin{multicols}{2}
\vspace{-10pt}
\begin{align*}
z_1=&1\\
z_2=& e^{2\pi i/3} =  \cos \frac{2\pi}{3} + i\sin \frac{2\pi}{3} \\
& = -\frac{1}{2}+i\frac{\sqrt{3}}{2} \\
z_3= & e^{4\pi i/3} = \cos \frac{4\pi}{3} + i\sin \frac{4\pi}{3} \\ 
& = -\frac{1}{2}-i\frac{\sqrt{3}}{2} 
\end{align*}
\begin{center}
\includegraphics[width=.5\textwidth]{Figures/cube}
\end{center}
\end{multicols}
\end{example}
 


\begin{center}
{\it How do we find all solutions to $
z^{n}=a$ where $a\in\mathbb{C}$?}
\end{center}

Again, we can use $n$th roots of unity. We showcase the method in an example:



\begin{example}

Suppose we set $a=16i$ and wish to find the fourth roots of $a$ (i.e., solve $z^4=16i$). We can find one quick root using polar coordinates: note that $a=16e^{i\pi/4}$, so we can spot one root as
\[
z=16^{1/4}=e^{i\frac{\pi}{4}\frac{1}{4}} = 2e^{i\frac{\pi}{8}},\]
since then $z^{4} = a$. 

Recall that there are at most $4$ distinct roots for a degree $4$ polynomial. Note that since $z$ is a solution, if $1,w,w^2,w^3$ (that is, $1,e^{i2pi/4}=e^{i\pi/2}, e^{i4\pi/4}=e^{i\pi}$, and $e^{i6\pi/4}=e^{i3\pi/2}$) are the $4$th roots of unity, then for $k=0,1,2,3$,
\[
(zw^{k})^{4}=z^{4}w^{4k}=a\cdot 1=a.
\]
Thus, the other solutions are $z,zw,zw^2,zw^3$. For $k \in \{0,1,2,3\}$ this gives the following (in exponential form)
\vspace{-1mm}
\[ 2e^{i\frac{\pi}{8}},2e^{i\frac{5\pi}{8}},2e^{i\frac{9\pi}{8}},2e^{i\frac{13\pi}{8}}\]
\vspace{-1mm} 
or equivalently
\vspace{-1mm}
\[ 2e^{i\frac{\pi}{8}},2e^{i\frac{5\pi}{8}},2e^{i\frac{-7\pi}{8}},2e^{i\frac{-3\pi}{8}}\]
\end{example}

 


\chapter{Polynomials}


\epigraph{\it  The mathematicians have been very much absorbed with finding the general solution of algebraic equations, and several of them have tried to prove the impossibility of it. However, if I am not mistaken, they have not as yet succeeded. I therefore dare hope that the mathematicians will receive this memoir with good will, for its purpose is to fill this gap in the theory of algebraic equations.}{Niels Henrik Abel, 1824, having shown there are no formulas for roots to polynomials degree $5$ and higher.}





\section{Introduction}



\begin{definition}
For $n\in\mathbb{N}$, an {\it $n$-degree complex polynomial} is a function of the form 
\[
p(z)=a_nz^n + a_{n-1}z^{n-1} + \cdots + a_0
\]
where $a_n \neq 0$ and $a_i \in \mathbb{C}$ for all $i$. We say that $\alpha$ is a {\it root} of $p(z)$ if $p(\alpha)=0$, in other words, if $\alpha$ is a solution to the polynomial equation $p(z)=0$.
\end{definition}



Any quadratic $az^2+bz+c$ has  two roots using the familiar {\it quadratic formula}:

\[\frac{-b\pm \sqrt{b^2-4ac}}{2a}.\]

{\bf Note:} One needs to be careful here, since we are allowing $a,b,c\in\mathbb{C}$, so  $b^2-4ac$ could be complex. In this case, $\sqrt{b^2-4ac}$ is interpreted to mean any number $z$ so that $z^2=b^2-4ac$, and then the solutions are $\frac{-b\pm z}{2a}$.\\


 There are also formulae for the roots of cubic or quartic (i.e. degree 3 or 4) polynomials, although these are much less convenient to write down. It is natural to ask whether there are convenient formulae for higher order polynomials, but this is not the case:


\begin{theorem}[Abel-Ruffini Theorem] There is no formula (like the quadratic formula) for the roots of a polynomial of degree $\geq 5$.
\end{theorem}


This doesn't mean it is impossible to solve higher order polynomials for their roots, it just means we have to be more clever, so below we fill focus on learning a few tricks to help us out.
 

\subsection{Factorizing Polynomials}

We will require the following theorem, which we will state without proof (although by year 3 you will have the background to understand its proof):

\begin{theorem}[Fundamental Theorem of Algebra]
Any complex polynomial has at least one root in $\mathbb{C}$
\end{theorem}

Using this, we know that any polynomial can be factored as follows:

\begin{theorem}[Factorization Theorem]
If $p$ is a degree $n$ polynomial, then there are $n$ roots $z_{1},...,z_{n}\in\mathbb{C}$ so that 
\[
p(z) = a(z-r_{1})(z-r_{2})\cdots (z-r_{n}).\]
Some roots may repeat. If a root appears $m$ times in $r_{1},...,r_{n}$, it has {\it multiplicity} $m$. 
\end{theorem}



\begin{proof}
Since knowledge of a root gives us a factor, we can write $p(z)=(z-\alpha)q(z)$ where $q(z)$ is a degree $(n-1)$ polynomial. Indeed, if $z_1$ is a root, then consider the polynomial $r(z)=p(z+z_1)$. Then this has a root at $0$,  and if we write it out as a sum 
\[
r(z)= r_{0}+r_{1}z+\cdots + r_{n}z^{n},
\]
we see that $r_{0}=r(0)=p(z_1)=0$, thus,
\[
p(z+z_{1})=r_{z} = r_{1}z+\cdots + r_{n}z^{n} = z\underbrace{(r_{1}+r_{2}z+\cdots + r_{n-1}z^{n})}_{=q(z+z_{1})}.
\]
Thus, $p(z)=(z-z_{1})q(z)$. 

Now we prove the theorem by induction on $n$. The base case $n=1$ is immediate, so now suppose the theorem holds true for some integer $n\geq 1$. Let $p$ be a degree $n+1$ polynomial. By the Fundamental Theorem of Algebra, there is a root $z_{n+1}$ of $p$, and so by the earlier discussion, we may factor
\[
p(z) = (z-z_{n+1})q(z)\]
for some degree $n$ polynomial $q$. By our induction hypothesis, there are $a,z_{1},...,z_{n}\in\mathbb{C}$ so that 
\[
q(z) = a(z-r_{1})(z-r_{2})\cdots (z-r_{n})\]
and so
\[
p(z) = a(z-r_{1})(z-r_{2})\cdots (z-r_{n}) (z-z_{n+1}).
\]
This proves the induction step and hence the theorem.



\end{proof}

 
This might seem useless since this factorization theorem doesn't {\it tell} us what the roots are, but because it tells us about the structure of a polynomial, it can help us find roots if we are, say, given partial information about other roots or its coefficients. This is particularly helpful if we are given a {\it real polynomial}, by which we mean a polynomial $p(z)=a_{0}+\cdots + a_{n}z^{n}$ such that each $a_i \in \mathbb{R}$.


\begin{theorem}[Real Polynomials have conjugate roots] If $p(x)$ has {\it real} coefficients and $r$ is a root, so is $\bar{r}$. 
\end{theorem}

\begin{proof}
If $p(x)=a_0+a_{1}x+\cdots +a_{n}x^n$ with $a_{i}$ real, and $p(r)=0$, then
\begin{align*}
0 =\overline{p(r)}
 & =\overline{a_0+a_{1}r+\cdots +a_{n}r^n} \\
{\color{magenta} (\overline{z+w}=\bar{z}+\bar{w})} & = \overline{a_0}+\overline{a_1 r}+\cdots + \overline{ a_n r^n} \\ 
{\color{magenta} (\overline{zw}=\bar{z}\bar{w})} & =\overline{a_0}+\bar{a_1} \bar{ r}+\cdots + \bar{ a_n}\bar{ r^n} \\ 
{\color{magenta} (a_{i} \;\; \mbox{are real})} &  ={a_0}+{a_1} \bar{ r}+\cdots + { a_n}\bar{ r^n} = p(\bar{r}).
\end{align*}
 
\end{proof}
 

%Here we've used that conjugation commutes with both sums and products, and that real numbers (the $a_i$) are invariant under conjugation. 


\noindent {\bf Note:} If any of the coeffients is not a real number then all bets are off! That is, we won't be able to factor into conjugate terms. \\




\begin{example}
Find the roots of $x^4+2x^3-7x^2+2x-8$ given that one of them is $i$. \\


Recall that the roots come in conjugate pairs, and so $-i$ is also a root, hence $(x-i)(x+i)=x^2+1$ is a factor in the above polynomial, so we can do polynomial long division to see how it factors: first, we subtract $x^2(x^2+1)$ from the polynomial to get
\[
2x^3-8x^2+2x-8\]
Next, we remove $2x(x^2+1)$ from this to get
\[
-8x^2-8=-8(x^2+1).
\]
Thus, we see that 
\[
x^4+2x^3-7x^2+2x-8=(x^2+1)(x^2-2x-8)
\]
Now we just need to solve $x^2-2x-8=0$. Using the quadratic formula, we find that the other two roots are $-2$ and $4$. Thus, all the roots are $\pm i, -2,$ and $4$. 
\end{example}

Notice how in that example, we started off just knowing one root and from that the polynomial collapsed and we could find the other 3, thus, even with partial information about the roots of a polynomial, we can use tricks like these to solve for them all. 



Another useful tool is the following theorem which shows how the coefficients of a polynomial relate to the roots.

\begin{theorem}[Root-Coefficient Theorem] If $p(x)=x^{n}+a_{n-1}x^{n-1}+\cdots + a_{1}x+a_{0}$, has roots $r_{1},...,r_{n}$ (counting multiplicities), then
\[
r_{1}+\cdots + r_{n} = -a_{n-1} \]
\[
r_{1}\cdots r_{n} = (-1)^{n}a_{0}.\]
In general, if $s_{j}$ denotes the sum of all products of $j$-tuples of the roots (e.g. $s_{2} = r_{1}r_{2}+r_{1}r_{3}+r_{2}r_{3}+\cdots $), then
\[
s_{j} = (-1)^{j}a_{n-j}.
\]

\end{theorem}

\begin{proof}
%We prove by induction. The case when $n=1$ can easily be verified. Suppose now that any degree $n$ polynomial satisfies the conclusions of the above theorem. Let 
%\[
%p(x)=x^{n+1}+a_{n}x^{n}+\cdots + a_{1}x+a_{0}.
%\]
%Let $r_{n+1}$ be a root of $p(x)$, so we can factor
%\[
%p(x) = (x-r_{n+1})q(x)\]
%where 
%\[
%q(x) = x^{n}+(a_n-r_{n+1})x^{n-1}+\cdots + 

First, factorize
\[
p(x)=x^{n}+a_{n-1}x^{n-1}+\cdots + a_{1}x+a_{0}=a(x-r_{1})\cdots (x-r_{n}).\]
Note that as the coefficient of $x^{n}=1$, we know $a=1$ (since otherwise the right side, when multiplied out, wouldn't equal the left). We can establish the formulas in the theorem now by multiplying out the product on the right.
\end{proof}







\begin{example}
Suppose $x^{3}+ax^{2}+bx+c$ has roots $\alpha,\beta,$ and $\gamma$. Find a polynomial with roots $\alpha\beta$, $\beta\gamma$, and $\gamma\alpha$ in terms of $a,b,c$ (that is, the coefficients of your polynomial should only be described using $a,b,$ and $c$, not $\alpha,\beta$, and $\gamma$). 

By the Root Coefficient Theorem,
\[
\alpha+\beta+\gamma = -a,
\]
\[
\alpha\beta+\beta\gamma+\gamma\alpha = b\]
and 
\[
\alpha\beta\gamma = -c.\]
Let $x^{3}+Ax^{2}+Bx+C$ be a polynomial with roots $\alpha\beta$, $\beta\gamma$ and $\gamma\alpha$. Then we know
\[
-A=\alpha\beta+\beta\gamma+\gamma\alpha = b\]
\[
B=\alpha\beta^{2}\gamma+\alpha\beta\gamma^{2}+\alpha^{2}\beta\gamma=-c(\alpha+\beta+\gamma)=ac
\]
and 
\[
-C=\alpha\beta \cdot \beta\gamma\cdot \gamma\alpha 
 = (\alpha\beta\gamma)^{2}=c^{2}
 \]
 Hence, the polynomial is 
 \[
 x^{3}-bx^{2}+acx-c^{2}.
 \]
\end{example}
 






\section{Exercises}



\begin{exercise} Show that $\frac{z\overline{w}+\overline{z}w}{2} = \Re(z\overline{w})$. 

\end{exercise}

\begin{exercise} Show that 
\[
\Re(zw)\leq \frac{|z|^2+|w|^2}{2}.
\]


\end{exercise}



\begin{exercise} Find the (complex) roots of the following polynomials:\\

(a) $x^2-5x+7-i=0$. 

\begin{solution}

Recall that the roots of this polynomial are 
\[
x= \frac{5\pm z}{2}
\]
where $z$ are the solutions to 
\[
z^2 = 5^2-4\cdot (7-i)\cdot 1  = 25-28+4i = -3+4i.
\]
If we set $z=a+ib$, this gives
\[
a^2-b^2+2abi = -3+4i.
\]
So in particular, $2ai = 4i$ and $a^2-b^2 = -3$. Moreover, 
\[
|z^2|= |-3 + 4i| = 5
\]
Thus, 
\[
5=|z^2|=|z|^2 = \sqrt{a^2+b^2}^2 = a^2+b^2
\]
Adding this to  $a^2-b^2=-3$ implies $2a^2 = 2$, so $a^2=1$, and so $a\pm 1$. Similarly, subtracting $a^2-b^2=-3$ from the above equation gives $2b^2 = 8$, so $b=\pm 2$. Finally, $2ab = 4$ implies that $a<0<b$ or $b<0<a$, so we must have $a+ib$ is $1-2i$ or $-1+2i$, so these are our two solutions for $z$. Hence,
\[
x=\frac{5 \pm (1-2i)}{2} \]
so $x$ is either $3-i$ or $2+i$. 

\end{solution}




(b)  $ x^4 + x^2 + 1 = 0$.

\begin{solution}
Let $y=x^2$. Then $y^2+y+1=0$, and the solutions to this are
\[
x^2=y=\frac{1\pm \sqrt{5}}{2} 
\]
Thus, we see that $x=\pm \sqrt{\frac{1+\sqrt{5}}{2}}$ are two solutions, we just need to find the other two. They will be solutions to $x^2 = \frac{1-\sqrt{5}}{2} = - \frac{\sqrt{5}-1}{2}$, which are $\pm i \sqrt{\frac{\sqrt{5}-1}{2}}$. Thus, all 4 solutions are 

\[
\pm \sqrt{\frac{1+\sqrt{5}}{2}}, \;\; \pm i \sqrt{\frac{\sqrt{5}-1}{2}}.
\]



\end{solution}


(c) $2x^4-4x^3+3x^2+2x-2$, given that one of the roots is $1+i$.


\end{exercise}


\begin{exercise}  Below you see a graph of the complex plane and some points representing some complex numbers (the outer and inner circles have radii $1$ and $1/2$ for scale). They lie on lines through the origin making a 45 degree angle with the x-axis.


\begin{multicols}{2}
Draw the points in the plane if we plug these points into the functions
\begin{itemize}
\item $\left(\sqrt{2}+\sqrt{2}i\right)z$
\item $z^2$
\item $\frac{1}{z}$. 
\end{itemize}

\begin{center}
\includegraphics[width=100pt]{Figures/complex-diagram.pdf}
\end{center}
\end{multicols}


\begin{solution}
The portraits of (a), (b) and (c) are as follows:

\begin{center}
\includegraphics[width=300pt]{Figures/complex-soln.pdf}
\end{center}
\end{solution}
%\vspace{1cm}


\end{exercise}


\begin{exercise} Find all solutions to $(z+1)^4=z^4$. 

\begin{solution}
First, we write this equation as $(1+1/z)^4=1$, so this implies that $1+1/z$ is one of the $4$th roots of unity $\pm1,\pm i$. However, there are no solutions to $1+1/z=1$, so the only solutions are when $1+1/z$ is $-1$ or $\pm i$, in which case 
\[
z=-\frac{1}{2}, \frac{1}{1\pm i}.
\]
\end{solution}


\end{exercise}



\begin{exercise} Factor the following polynomials into products of real polynomials that are linear and/or quadratic:

\begin{itemize}
\item $x^3-1$.
\begin{solution}
We first need to find the roots of unity of $x^3$, which are $1, w=e^{2\pi i/3}=-\frac{1}{2}+\sqrt{3}{2}i$ and $w^2=e^{4\pi i/3} = -\frac{1}{2}-\sqrt{3}{2}i=\overline{w}$.  Thus,
\begin{align*}
x^3-1 & = (x-1)(x-w)(x-\overline{w}) = (x-1)(x^2-wx-\overline{w}x+\overline{w}w) \\
& = (x-1)(x^2+x+|w|^2) = (x-1)(x^2+x+1).
\end{align*}
\end{solution}
\item $x^3+1$.

\begin{solution}
We need to find the roots of $x^3+1=0$, which are solutions to $x^3=-1=e^{\pi i}$. One solution is clearly $1$. Another we can get by taking 1/3 the exponent of $e^{\pi i}$, which is $w=e^{\pi i/3} = \frac{1}{2}+\frac{\sqrt{3}}{2}$. We know that roots of real polynomials come in conjugate pairs, and so the other root is $\overline{w}$. Thus,
\begin{align*}
x^3+1 & = (x-(-1))(x-w)(x-\overline{w}) = (x+1)(x^2-wx-\overline{w}x+\overline{w}w) \\
& = (x+1)(x^2-x+|w|^2) = (x+1)(x^2-x+1).
\end{align*}
\end{solution}

\item $x^4-1$.

\begin{solution}
There is no need to find complex roots here:
\[
x^4-1 = (x^2)^2-1 = (x^2-1)(x^2+1).
\]
\end{solution}

\item $x^4+1$. 

\begin{solution}
To find the roots of $x^4+1$, we need to find solutions to $x^4=-1=e^{\pi i}$. One solution is $z=e^{\pi i/4}=\frac{1}{\sqrt{2}}+\frac{1}{\sqrt{2}}i$. To find all 4 roots, we just multiply this by the 4th roots of unity, which are $\pm 1, \pm 1$, thus all solutions which will be
\[
\pm \frac{1}{\sqrt{2}} \pm \frac{1}{\sqrt{2}}i
\]
where we consider all 4 possible combinations of $+$'s and $-$'s. Let $z=\frac{1}{\sqrt{2}}+i\frac{1}{sqrt{2}}$ and $w=-\frac{1}{\sqrt{2}}+i\frac{1}{\sqrt{2}}$. Then the roots are $z,w,\overline{z},\overline{w}$. Thus,
\begin{align*}
x^4+1 & 
= (x-z)(z-\overline{z})(x-w)(x-\overline{w})\\
& =(x^2 -zx-\overline{z}x+z\overline{z})(x^2 -wx-\overline{w}x+w\overline{w})\\
& = (x^2-\sqrt{2}x+|z|^2)(x^2+\sqrt{2}x+|w|^2)\\
& = (x^2-\sqrt{2}x+1)(x^2+\sqrt{2}x+1).
\end{align*}
\end{solution}

\item $x^5+1$. {\it Hint: $\cos \frac{2\pi}{5}= \frac{-1+\sqrt{5}}{4}$ and $\cos \frac{4\pi}{5}= \frac{-1-\sqrt{5}}{4}$.}

\begin{solution}


We note that this is easily completely factorized over the complex numbers.  Let $\xi = e^{\frac{2\pi i}{10}}$, then the $5$th roots of unity are $1,\xi,\xi^2, \xi^3,\xi^4$. By examining these numbers, we can see that $\xi^{4}=\overline{\xi}$ and $\xi^{3} = \overline{\xi}^{2}$.  Thus, 

$$x^5+1 = (x+1)\underbrace{(x-\xi)(x-\bar{\xi})}_{\textrm{conjugate}}\underbrace{(x-\xi^2)(x-\bar{\xi}^2)}_{\textrm{conjugate}}.$$

If we collect into pairs the terms which involve a root and its conjugate, we obtain real polynomials.

We compute:
$$(x-\xi)(x-\bar{\xi}) = x^2 - (\xi+\bar{\xi})x + \xi\bar{\xi} = x^2 - 2\cos(\frac{2\pi}{5})x+1 
= x^2 +\frac{1-\sqrt{5}}{2}x+1.
$$ 
and 

$$(x-\xi^2)(x-\bar{\xi}^2) = x^2 -(\xi^2 + \bar{\xi}^2)x + \xi^2\bar{\xi}^2 = x^2 - 2\cos(\frac{4\pi}{5})x+1
= x^2 +\frac{1+\sqrt{5}}{2}x+1.
,$$
Hence,
\[
x^5 = \left(x^2 +\frac{1-\sqrt{5}}{2}x+1\right) \left(x^2 +\frac{1+\sqrt{5}}{2}x+1\right).
\]

\end{solution}


\item $x^6+1$. 

\begin{solution}

The roots of $x^6+1$ are $e^{i\pi/6}\omega^{j}$ where $\omega= e^{i2\pi/6}=e^{i\pi/3}$ and $j=0,1,2,3,4,5$. Hence,
\[x^{6}+1
 =(x-e^{i\pi/6})(x-e^{i\pi/6}w)(x-e^{i\pi/6}w^2)(x-e^{i\pi/6}w^3)(x-e^{i\pi/6}w^4)(x-e^{i\pi/6}w^5).\]
We need to group the terms into conjugate pairs, so that when we multiply the pairs out they become real numbers. 
Note that 
\[
e^{i\pi/6}w^{5}=e^{i\pi/6+i5\pi/3}=e^{i11\pi/6}=e^{-i\pi/6}=\overline{e^{i\pi/6}},
\]
\[
e^{i\pi/6}w^{4}=e^{i\pi/6+4\pi/3}=e^{i3\pi/2}=-i=\overline{i}=\overline{e^{i\pi/6}\omega}
\]
\[
e^{i\pi/6}w^{3}=e^{i\pi/6+i\pi}=e^{i7\pi/6}=\overline{e^{5\pi/6}}
=\overline{e^{i\pi/6}w^{2}}
\]
Thus,
\begin{align*}
x^{6}+1
&  =(x-e^{i\pi/6})(x-e^{i\pi/6}w)(x-e^{i\pi/6}w^2)(x-e^{i\pi/6}w^3)(x-e^{i\pi/6}w^4)(x-e^{i\pi/6}w^5)\\
& = (x-e^{i\pi/6})(x-e^{i\pi/6}w)(x-e^{i\pi/6}w^2)(x-e^{i\pi/6}w^3)(x-e^{i\pi/6}w^4)(x-e^{i\pi/6}w^5)
\end{align*}
Hence, rearranging the terms in our product for $x^6+1$, we get
\begin{align*}
 x^6+1
&  =(x-e^{i\pi/6})(x-e^{i\pi/6}w^5)(x-e^{i\pi/6}w^2)(x-e^{i\pi/6}w^3)(x-e^{i\pi/6}w^4)(x-e^{i\pi/6}w)\\
& =(x-e^{i\pi/6})(x-e^{-i\pi/6})(x-e^{i\pi/6}w^2)(x-e^{-i\pi/6}w^{-2})(x-e^{-i\pi/6}w^{-1})(x-e^{i\pi/6}w)\\
& =(x^2-2xRe(e^{i\pi/6})+1)(x^2-2xRe(e^{i\pi/6}w^2)+1)(x^2-2xRe(e^{i\pi/6}w)+1)\\
& = (x^2-2x\cos\pi/6+1)(x^2-2x\cos(7\pi/6)+1)(x^2-2x\cos \pi/2+1)\\
& = (x^2-\sqrt{3}x+1)(x^2+\sqrt{3}x+1)(x^2+1).
\end{align*}


\end{solution}

\end{itemize}


\end{exercise}



\begin{exercise} Solve $z^2=i\overline{z}$. 

\begin{solution}
Notice that if this equation holds, then
\[
|z|^2=|i\overline{z}|=|i|\cdot |\overline{z}| = |z|,\]
so either $|z|=0$ or $|z|=1$. In the latter case, this means that $\overline{z} = \frac{1}{z}$, and so
\[
z^2=i\overline{z} = \frac{i}{z}\]
which implies
\[
z^3=i.
\]
Hence, we just need to solve this equation now. Since $i=e^{\frac{\pi}{2}i}$, then $e^{\frac{\pi}{6}i}$ is one solution. Thus, to get all 3 solutions, we multiply this by all the 3rd roots of unity, so we get
\[
e^{\frac{\pi}{6}i}, \;\; e^{\frac{5\pi}{6}i}, \;\; e^{\frac{3\pi}{2}i}.
\]
Thus, all solutions to the original equation are 
\[
0, \;\; \pm \frac{\sqrt{3}}{2}+\frac{i}{2}, \;\; -i.
\]

\end{solution}

\end{exercise}



\begin{exercise} Solve $|z|^2 - z|z| + z = 0$. 

\begin{solution}
Note that by rearranging the equations so that the $z's$ are on one side and $|z|'s$ are on the other, we get if $|z|\neq 1$ that 
\[
z  =\frac{|z|^{2}}{|z|-1},\]
and so $z\in\R$. If $z\geq 0$, then our original equation becomes $z^2-z^2+z=0$, hence $z=0$. If $z<0$, then $|z|=-z$, and our original equation is 
\[
z^2+z^2+z=0\]
So $0=z(2z+1)$, hence $z=0$ or $z=-\frac{1}{2}$. 

If $|z|=1$, then the original equation is
\[
1-z+z=0,\]
which is impossible, so there are no solutions in this case. Thus, $z=0,-\frac{1}{2}$ are the only solutions.
\end{solution}


\end{exercise}




\begin{exercise} Show that if $|z| = 1$, then $\Re\frac{1}{1-z} =
\frac{1}{2}$.

\begin{solution}
\[ \frac{1}{1-z} = \frac{1}{1-z}\frac{1+\bar{z}}{1+\bar{z}} =
\frac{1+\bar{z}}{1-z+\bar{z} + z\bar{z}}\]
\[ =
\frac{1+\bar{z}}{1-2iy-|z|^{2}}=\frac{1+\bar{z}}{-2iy}=i\frac{1+\bar{z}}{2y}\]
\[\Re \frac{1}{1-z} = \Re i\frac{1+\bar{z}}{2y} = \Re
i\frac{1+x-iy}{2y} = \Re\left(\frac{y}{2y} + i\frac{1+x}{2y}\right) =
\frac{y}{2y}\]
\[=\frac{1}{2}\]
\end{solution}


\end{exercise}



%2019/20 Exam problem
%\begin{exercise} Prove that for complex numbers $z$ and $w$
%\[
%|z+w|^2+|z-w|^2=2|z|^2+2|w|^2.
%\]
%
%
%\end{exercise}





\begin{exercise} If $x^3+15x^2+74x+120$ has roots of the form $a,a+1,a+2$, find $a$.

\begin{solution}
We see that by the Root-Coefficient Theorem,
\[
-15=a+a+1+a+2=3a+3\]
and so $a=-3$.
\end{solution}

\end{exercise}




\begin{exercise} Describe geometrically the points $z\in\mathbb{C}$ so that $|z-1|=|z+i|$. 



\end{exercise}





\begin{exercise} 
 Let $a\in \mathbb{R}$. If $x^{3}-x+a$ has three integer roots, solve for $a$. 

\begin{solution}


Note that if the integer roots are $r_{1},r_{2}$, and $r_{3}$, then by Proposition 7.1,
\[
r_{1}+r_{2}+r_{3}=0\]
and 
\[
r_{1}r_{2}+r_{2}r_{3}+r_{3}r_{1}=-1
\]
Hence,
\[
r_{1}^{2}+r_{2}^{2}+r_{3}^{2} = (r_{1}+r_{2}+r_{3})^{2}-2(r_{1}r_{2}+r_{2}r_{3}+r_{3}r_{1})=2.
\]
Since the $r_{i}$ are integers, the $r_{i}^{2}$ are nonnegative integers, and so one of them has to be zero. Thus, $0=0^{2}-0+a$, hence $a=0$. 

\end{solution}


\end{exercise}





\begin{exercise} Show that if $z_1,z_2,z_3\in \C$ are so that $z_1+z_2+z_3=0$, and $z_1^2+z_2^2+z_3^2=0$ then $|z_1|=|z_2|=|z_3|$


\begin{solution}
Let $f(z)=(z-z_1)(z-z_2)(z-z_3)$. Then $f(z)=z^3-az^2+bz-c$ where 
\begin{align*}
a&=z_1+z_2+z_3\\[4pt]
b&=z_1z_2+z_2z_3+z_3z_1\\[4pt]
c&=z_1z_2z_3\\[4pt]
\end{align*}
By assumption, we get that $a=z_1+z_2+z_3=0$.  Also, since $z_{1}^2+z_{2}^2+z_{3}^2=0$, we have 
\[
(z_1+z_2+z_3)^2=z_1^2+z_2^2+z_3^2+2(z_1z_2+z_2z_3+z_3z_1)
\]
and so $b=0$. Thus, $f(z)=z^3-c$, so in particular, since $z_{1},z_{2}$ and $z_{3}$ are roots, we have $z_{1}^3=z_{2}^3=z_{3}^{3}=c$. Thus
\[
|c|=|z_{i}^3|=|z_{i}|^3
\]
so $|z_{i}|=|c|^{1/3}$ for $i=1,2,3$. 
\end{solution}

\end{exercise}



\begin{exercise} Show that the solutions of $z^3=c$ where $|c|=1$ are the corners of an equilateral triangle.

\end{exercise}



\begin{exercise} Suppose $|z_{1}|=|z_{2}|=|z_{3}|=1$ and $z_{1}+z_{2}+z_{3}=0$. Show that $z_{1}^3=z_{2}^3=z_{3}^3$. {\it Hint: First think about when $z_{1}=1$, then use that to prove the general case.}

\begin{solution}
First let's assume $z_{1}=1$. Then $z_{2}+z_{3}=-1$. In particular, this means that the imaginary parts of $z_{2}$ and $z_{3}$ cancel, that is, if $z_{j}=x_{j}+iy_{j}$, then $y_{2}=-y_{3}$. Also, $x_{2}+x_{3}=-1$, and we can't have $x_{2}>0$, since then $x_{2}+x_{3}>x_{3}\geq -1$. Similarly, we can't have $x_{3}>0$, thus $x_{2},x_{3}\leq 0$. Also, 
\[
-x_{2} = \sqrt{1-y_{2}^2} = \sqrt{1-y_{3}^2}= -x_{3},\]
hence $z_{3} = -x_{2} -iy_{2} = \overline{z_{2}}$, and $-1=z_{2}+z_{3}=-x_{2}-x_{2}=-2x_{2}$, thus $x_{2}=-\frac{1}{2}$. Thus, we must have that $z_{3} = -\frac{1}{2}\pm i\frac{\sqrt{3}}{2}$. 

For the general case, let $w_{i}=\overline{z_{1}}z_{i}$, then $w_{1}=1$ and $w_{1}+w_{2}+w_{3}=0$. We now apply the previous case to conclude that $w_{i} = w^{i}$ where $w=e^{2\pi i/3}$. Thus,
\[
z_{j}^3 = (z_{1} w_{j})^3 = z_{1}^3 w_{j}^3 = z_{1} (w^{3j})=z_{1}.
\]


\end{solution}

\end{exercise}




\begin{exercise} Suppose $|z_{1}|=|z_{2}|=|z_{3}|=1$ and $z_{1}+z_{2}+z_{3}=0$. Show that $z_{1}^{2^{n}}+z_{2}^{2^{n}}+z_{3}^{2^{n}}=0$ for all $n\in \mathbb{N}$. 

\begin{solution}
By the previous problem, $z_1,z_2,z_3$ are roots of $z^3-c$ for some complex number $c$ with $|c|=1$. In particular, by the Root-Coefficient Theorem,
\[
z_{1}z_{2}+z_{1}z_{2}+z_{2}z_{3}=0.
\]
Thus, 
\[
0=(z_{1}+z_{2}+z_{3})^2 = z_{1}^2+z_{2}^2+z_{3}^2 + 2(z_{1}z_{2}+z_{1}z_{2}+z_{2}z_{3})=z_{1}^2+z_{2}^2+z_{3}^2 .
\]
Now we can repeat the process by induction using $z_{1}^2,z_{2}^2,z_{3}^2$ instead of $z_1,z_2,z_3$.
\end{solution} 


\end{exercise}


\begin{exercise} Let $\mathbb{D}=\{z\in \mathbb{C}: |z|<1\}$, that is, the set of complex numbers with modulus strictly less than $1$. For a complex number $z\in \mathbb{D}$ and a real number $-1< r<1$, define
\[
f_{r}(z) = \frac{z-r}{1-zr}.\]
 Show $f_{r}(z)\in \mathbb{D}$ for all $z\in \mathbb{D}$ and $-1<r<1$.


***The above function $f_{r}$ is called a {\it M\"obius transformation} and is usually defined for $r\in \mathbb{D}$ as well, not just $-1<r<1$. They are the only functions with the property that, if $A$ is a circular arc or straight line in $\mathbb{D}$ (imagine the intersection of a circle or line with $\mathbb{D}$), then $f_{r}(A)$ is also a line or circlular arc (this is not part of the problem, it's just cool).







\begin{solution}
If $-1<r<1$, then
\begin{align*}
\left|\frac{z-r}{1-rz}\right|<1
& \Leftrightarrow \;\;|z-r|<|1-rz|\\
&  \Leftrightarrow \;\; |z-r|^{2}<|1-rz|^{2}\\
& \Leftrightarrow \;\; |z|^{2}+|r|^{2}-2Re(zr)<1+|rz|^{2}-2Re(zr)\\
& \Leftrightarrow \;\; |z|^{2}+|r|^{2}-|rz|^{2}<1\\
& \Leftrightarrow \;\; |z|^{2}(1-|r|^{2})+|r|^{2}<1\\
\end{align*}

But this inequality is true since $|z|<1$, so
\[
|z|^{2}(1-|r|^{2})+|r|^{2}<1\cdot (1-|r|^{2})+|r|^{2}=1. 
\]

\end{solution}


\end{exercise}

\begin{exercise} We all know what $\cos \theta$ and $\sin\theta$ are when $\theta$ is a multiple of $\pi/6$, $\pi/4$, $\pi/3$, or $\pi/2$, but what about $\pi/5$, $\pi/7$ and $\pi/8$?  In the following problems, we will use complex numbers to find other values of cosine and sine, and other interesting facts about these trigonometric functions. 


\begin{itemize}

\item (Liebeck 6.7)  Here we will find $\cos \pi/5$.
\begin{itemize}
\item Let $w=e^{2\pi i/5}$. Show that
\[
1+w+w^2+w^3+w^4=0.
\]
\begin{solution}
Note that 
\[
1+w+w^2+w^3+w^4 = \frac{1-w^5}{1-w}=\frac{1-1}{1-w}=0.
\]
\end{solution}

\item Let $\alpha = 2\cos 2\pi/5$ and $\beta = \cos 4\pi/5$. Show that $\alpha = w+w^4$ and $\beta = w^2+w^3$. 

\item Find a polynomial whose roots are $\alpha$ and $\beta$, solve it to find $\cos 2\pi /5$. 

\begin{solution}
Let 
\begin{align*}
p(x) 
& = (x-\alpha)(x-\beta)\\
& =x^2 - (\alpha + \beta)x+\alpha \beta \\
& =x^2 -(w+w^2+w^3+w^4)x+(w^3+w^4+w^6+w^7) \\
& = x^2 +x + w^3 (1+w+w^3+w^4)\\
& = x^2 + x - w^3w^2 = x^2 + x -1 .
\end{align*}
The roots of this polynomial are 
\[
\frac{-1 \pm \sqrt{5}}{2}.
\]
Since $\alpha>0>\beta$, we see that 
\[
2\cos \frac{2\pi}{5} = \alpha  = \frac{-1+\sqrt{5}}{2}.
\]
Thus,
\[
\cos \frac{2\pi}{5} = \frac{-1+\sqrt{5}}{4}.
\]
Finally, since 
\[
\cos \frac{2\pi}{5} = 2\cos^2 \frac{\pi}{5} -1,\]
we see that 
\[
\cos \frac{\pi}{5} = \sqrt{\frac{3 + \sqrt{5}}{8}}.
\]



\end{solution}
\end{itemize}

%
\item Repeat the argument using $7$th roots of unity to find a polynomial with integer coefficients whose roots are $2\cos 2\pi/7$,  $2\cos 4\pi/7$, and $2\cos 6\pi/7$.

\begin{solution}
Again, if $w=e^{2\pi i/7}$, we have 
\[
1+w+w^2+\cdots + w^6=0.
\]
Moreover, notice that we can match these terms as conjugate pairs: $\overline{w} = w^6$, $\overline{w^2}=w^5$, $\overline{ w^3} = w^{4}$. Thus, when we add these together, we get 
\[
\alpha = w+w^6 = 2\cos \frac{2\pi}{7}, \;\; \beta = w^2+w^5 = 2\cos \frac{4\pi }{7}, \;\; \gamma = w^{3} + w^{4} = 2\cos\frac{6\pi }{7} .
\]
Let's look at the polynomial that has these numbers as roots:

\[
p(x) 
 = (x-\alpha)(x-\beta)(x-\gamma) 
= x^3+ax^2+bx+c
\]
where
\[
-a=\alpha +\beta + \gamma=1+w+\cdots + w^6=0,
\]
\begin{align*}
b & = \alpha \beta + \beta \gamma + \gamma\alpha\\
& = w^3+w^6+w^8+w^{11} + w^5+w^6+w^8+w^9 + w^4+w^5+w^9+w^{10}\\
& =w^3(1+2w^3+2w^5+w^8+2w^2+2w^6+w+w^7) \\
& w^3(1+2w^3+2w^5+w+2w^2+2w^6+w+1)  \\
& = 2w^3(1+w^3+w^5+w^2+w^6+w)\\
& =2w^3\cdot 2^4 = 2.
\end{align*}
and finally,
\begin{align*}
-\alpha\beta\gamma 
& = (w+w^6)(w^2+w^5)(w^3+w^4)\\
& =w(1+w^5)w^2(1+w^3)w^3(1+w) \\
& =w^6(1+w^5)(1+w^3)(1+w) = w^6(1+w+w^3+w^4+w^5+w^6+w^8+w^9)\\
& = w^{6}(-w^2+w^8+w^9) =w^6(-w^2+w+w^2) = w^7=1.
\end{align*}
Hence,
\[
p(x) = x^3+2x-1.
\]


\end{solution}


\item Find $\cos \pi/8$. 

\begin{solution}
We can just do this via the double angle formula:
\[
\frac{1}{\sqrt{2}} = \cos \pi/4 = 2\cos^2 \pi/8-2
\]
and so 
\[
\cos \frac{\pi}{8} = \sqrt{\frac{2\sqrt{2}+1}{2\sqrt{2}}}.
\]

\end{solution}



%
%\begin{align*}
%\cos \frac{\pi}{3} 
%& =\cos \frac{\pi}{9}\cos\frac{2\pi}{9}
%-\sin \frac{2\pi}{9}\sin\frac{\pi}{9}\\
%& =\cos\frac{\pi}{9}(2\cos^2\frac{\pi}{9}-1)-2\sin^2\frac{\pi}{9}\cos\frac{\pi}{9}\\
%& =2\cos^3\frac{\pi}{9} - \cos\frac{\pi}{9} -2\cos\frac{\pi}{9} +2\cos^2\frac{\pi}{9}
%=4\cos^3\frac{\pi}{9} -3\cos\frac{\pi}{9}
%\end{align*}
%
%Thus, $\cos^3\frac{\pi}{9}$ is a root of $0=4x^3-3x-\cos \frac{\pi}{3}=4x^3-3x-\frac{1}{2}$.


\item Similar to what we did with $7$th roots of unity, use $9$th roots of unity to find a three degree polynomial with integer coefficients whose roots are $ 2 \cos \frac{2\pi}{9}$, $ 2\cos \frac{4\pi}{9}$, and $ 2\cos \frac{8\pi }{9}$.

\begin{solution}
Consider the numbers $\alpha = 2 \cos \frac{2\pi}{9}$, $\beta = 2\cos \frac{4\pi}{9}$, $\gamma = 2\cos \frac{8\pi }{9}$.

\[
\alpha = w+w^8, \;\; \beta = w^2+w^7, \;\; \gamma = w^4+w^5.
\]
Thus,
\[
p(z)
=(z-\alpha)(z-\beta)(z-\gamma)
 = z^3 -(\alpha+\beta+\gamma)z^2 +(\alpha\beta+\beta\gamma+\alpha\gamma)z-\alpha\beta\gamma.
 \]
 Note that if $u=e^{2\pi i/3}$, then $w^3=u$, and so 
 \[
 \alpha+\beta+\gamma=w+w^2+w^4+w^5+w^7+w^8
 =-1-w^3-w^6 = -1-w-w^2
 =0.
 \]
 Also,
 \begin{align*}
 \alpha\beta & +\beta\gamma+\alpha\gamma\\
& =w^3+w^8+w^{10}+w^{15} +w^6+w^7+w^{11}+w^{12} + w^5+w^6+w^{12}+w^{13}\\
& = w^3+w^8+w+w^{6} +w^6+w^7+w^{2}+w^{3} + w^5+w^6+w^{3}+w^{4}\\
& = 3w^3+w^8+w+3w^6+w^7+w^2+w^5+w^4 \\
& = 2w^3+2w^6=2(u+u^2)=-2.
 \end{align*}
 Finally,
 \begin{align*}
 \alpha\beta\gamma 
 & = (w^3+w^8+w^{10}+w^{15})(w^4+w^5)
 =w^7+w^{12}+w^{14}+w^{19}+w^8+w^{13}+w^{15}+w^{20}\\
 & = w^7+w^3+w^5+w+w^8+w^4+w^6+w^2
 -1.
 \end{align*}
 
 Thus, 
 \[
 p(z) = z^3-2z+1.
 \]
 
 
 

\end{solution}
\end{itemize}


\end{exercise}

\begin{exercise} ({\bf Challenge!}) Show that 
\[
\prod_{k=1}^{n-1}\sin\frac{k \pi}{n} = \frac{n}{2^{n-1}}.
\]

\begin{solution}
Let $P$ denote the product above and $w=e^{2i\pi/n}$ be the $n$th root of unity. Then
\begin{align*}
P 
& =\prod_{k=1}^{n-1}\sin(k\pi/n)=(2i)^{1-n}\prod_{k=1}^{n-1}(e^{ik\pi/n}-e^{-ik\pi/n})\\
& =(2i)^{1-n}e^{-i\pi n(n-1)/(2n)}\prod_{k=1}^{n-1}(e^{2ik\pi/n}-1)\\
& =(-2)^{1-n}\prod_{k=1}^{n-1}(w^k-1)=2^{1-n}\prod_{k=1}^{n-1}(1-w^k),
\end{align*}
Now note, that $x^n-1=(x-1)\sum_{k=0}^{n-1}x^k$ and $x^n-1=\prod_{k=0}^{n-1} (x-w^k)$, thus cancelling $x-1$ we have $\prod_{k=1}^{n-1} (x-w^k) =\sum_{k=0}^{n-1}x^k$. Substituting $x=1$ we have $\prod_{k=1}^{n-1} (1-w^k)=n$. Therefore $P=n2^{1-n}$.

\end{solution}

\end{exercise}













\part{Week 5: Integers}




\chapter{Integers}



\epigraph{\it It is impossible to separate a cube into two cubes, or a fourth power into two fourth powers, or in general, any power higher than the second, into two like powers. I have discovered a truly marvelous proof of this, which this margin is too narrow to contain.}{Pierre de Fermat, {\it Arithmetica}, 1637...\\ Proof not found until 1994.} 

%\epigraph{\it God may not play dice with the universe, but something strange is going on with the prime numbers.}{Paul Erd\"{o}s} 


This week we will study the integers, particularly the techniques of studying divisibility, the highest common factor, and prime factorization. 

These techniques are especially useful for solving {\it diophantine equations}, which are polynomial equations where we seek {\it integer} solutions. The most famous diophantine equation is 
\[
x^{n}+y^{n}=z^{n}\]
It was famously claimed by Fermat (without proof) that the only cases where there are integer solutions $x,y,z$ to this equation is if $n=1$ or $n=2$. It wasn't until 1994 when Andrew Wiles actually gave a proof. 

They are also useful when we are trying to show at a diophantine {\it can't} be solved. You have seen one such proof before: that $\sqrt{2}$ is irrational. Let's recall the proof briefly:

Suppose there was a rational number $\frac{m}{n}$ (where $m$ and $n$ are reduced in the sense that they have no common divisors apart from $1$) so that $\left(\frac{m}{n}\right)^2=2$, then $m^2=2n^2$. This means that $m^2$ is even. Then $m$ must be even since, if instead $m=2k+1$ for some integer $k$, then $m^2=(2k+1)^2=4(k^2+k)+1$, which is odd, a contradiction. Hence, $m=2k$ for some integer $k$, and so
\[
2n^2=m^2=(2k)^2=4k^2\]
and so $n^2=2k^2$, and using a similar reasoning, we get that $n$ must also be even, which is a contradiction since we assumed $m$ and $n$ had no common factors, but now we have shown they are both divisible by $2$. 

The punchline of the proof was to use the {\it coprimality} of $m$ and $n$, that is, that they share no common divisors, as well as the fact that if $m^2$ is divisible by $2$ (that is, if $m^2$ was even), then so was $m$. If we would like to generalize this proof, we will need to develop  these tools about divisibility and coprimality. Later, we will in fact show that $\sqrt{n}$ is rational exactly when $n=m^2$ for some integer $m$. 



As motivation for the techniques we develop along the way, we will show how they can be used to solve various diophantine equations. As a special final application of these techniques, we'll also classify all {\it Pythagorean Triples}, that is, integers $x,y,z$ so that 
\[
x^2+y^2=z^2.\]



\section{Division and the Euclidean Algorithm}

\subsection{Remainders, Divisibility, and \hyperref[l:easy-lemma]{\hyperref[l:easy-lemma]{The Easy Lemma} } }


The following theorem is the starting point for our studies on divisibility. 

\begin{theorem}
Let  $a\in\mathbb{N}$ and $b\in\mathbb{Z}$, then there are unique integers $q\in \mathbb{Z}$ and $0\leq r<a$ so that 
\[
b=qa+r.
\]
\end{theorem}


Note that there may be several ways of writing $b=qa+r$ for general integers $q$ and $r$, for example with $b=3$ and $a=2$, $3=1\cdot 2 + 1 = 2\cdot 2 -1$. However, there is only one pair $(q,r)$ with $0\leq r<a$. 

\begin{proof}
We prove existence and uniqueness separately. \\

\noindent {\bf Existence:} Let $q$ be the largest integer for which $qa\leq b$. Let $r=b-qa\geq 0$. Then we must also have $r<a$, since if $r\geq a$, then $b-qa\geq a$, and so $b-(q+1)a\geq 0$, which contradicts our choice of $q$ since $q$ was the largest integer for which $qa\leq b$ (but now $q+1$ also does this and is strictly larger). This shows the existence of a pair $(q,r)$ satisfying the theorem. \\

\noindent {\bf Uniqueness:} Suppose there was another pair $(q',r')$ so that $b=q'a+r'$ and $0\leq r'<a$. Then
\[
0=b-b=q'a+r'-qa-r = (q'-q)a+(r'-r)\]
and so 
\[
r'-r=(q-q')a.\]
Since $q\neq q'$, we know $q-q'\neq 0$, hence $|q-q'|\geq 1$. However, since $0\leq r,r'<a$, we know $|r-r'|<a$, but then
\[
a>|r-r'|=|(q-q')a|=|q-q'|\cdot |a|=1\cdot a=a,\]
which is a contradiction. Thus, there is only one such pair $(q,r)$. 

\end{proof}


If $r=0$, then $b=qa$ for some $q$. This special case gets its own name in the next definition. 


\begin{definition}
For two integers $a$ and $b$ we say {\it $a$ divides $b$}, or write $a|b$, if there is an integer $c$ so that $b=ac$.
\end{definition}

So for example, $2|4$, but $2\not|3$. 


\begin{lemma}
\label{l:abba}
If $a|b$ and $b|a$, then $a=\pm b$. 
\end{lemma}

\begin{proof}
Exercise.
\end{proof}

%\begin{proof}
%Since $a|b$, we know $b=na$ for some integer $n$, and similarly, since $b|a$, we know $a=mb$ for some integer $m$. Hence, $a=mb=mna$, and so $1=mn$. The only way this can happen is if either $m=n=1$ or $m=n=-1$. 
%\end{proof}





We conclude this section with a very easy but very useful lemma. 

\begin{lemma}[The Easy Lemma] 
\label{l:easy-lemma}
Let $a$ and $b$ be integers and suppose $d|a$ and $d|b$. Then $d|ma+nb$ for all $m,n\in\mathbb{N}$. 
\end{lemma}

\begin{proof}
Since $d|a$ and $d|b$, there are integers $s$ and $t$ so that $a=sd$ and $b=td$, so 
\[
ma+nb=msd+ntd=(ms+nt)d,\]
thus, by definition, $d|ma+nb$.
\end{proof}

\def\easylemma{\hyperref[l:easy-lemma]{The Easy Lemma}}

We call it the Easy Lemma since, while the proof is actually very simple, it will be incredibly useful for us later on. 



\subsection{The HCF and the Euclidean Algorithm}


\begin{definition}
Given two integers $a,b\in\Z$ that are not both zero, the {\it highest common factor} of $a$ and $b$ is the largest positive integer $d\in\N$ that divides both $a$ and $b$. We denote this integer $d$ by $\hcf(a,b)$. If $\mathrm{hcf}(a,b)=1$, then we say that $a$ and $b$ are
\textcolor[rgb]{1.00,0.00,0.00}{coprime}.
\end{definition}

\begin{example} $\mathrm{hcf}(15,45)=15$, 
$\mathrm{hcf}(6,15)=3$, and $\mathrm{hcf}(17,91)=1$.
\end{example}

For a prime $p>1$ and any integer $n$, we have:
$$
\mathrm{hcf}(p,n) = \left\{\aligned%
&1\ \text{if $p$ does not divide $n$},\\
&p\ \text{if $p$  divides $n$}.\\
\endaligned
\right.
$$

\hyperref[l:easy-lemma]{The Easy Lemma}  provides a useful way for showing that two numbers are coprime:

\begin{example}
Here we'll show $n$ and $n+1$ are coprime. This might seem like it is common sense, but a proof at first glance might require some thinking. This is now almost immiedate by \hyperref[l:easy-lemma]{The Easy Lemma} : $\hcf(n,n+1)|(n+1)-n=1$, so $\hcf(n,n+1)|1$, which implies $\hcf(n,n+1)=1$, that is, $n$ and $n+1$ are coprime. 
\end{example}

We can also use it to narrow down the hcf of two numbers:

\begin{example}
For $n\in\mathbb{N}$, what is $\hcf(4n^2-2,2n)?$ Observe that by \hyperref[l:easy-lemma]{The Easy Lemma}, $\hcf(4n^2-2,2n)|(2n\cdot (2n)-1\cdot (4n^2-2)=2$, so the highest common factor of these numbers is either $1$ or $2$. However, since $2$ divides both $4n^2-2$ and $2n$, it also divides the $\hcf$, so Lemma \ref{l:abba} implies $\hcf(4n^2-2,2n)=2$.
\end{example}



How do we find the hcf for very large numbers? For this we use the {\it Euclidean Algorithm} which exploits the  Remainder Theorem and \hyperref[l:easy-lemma]{The Easy Lemma} . 

Let $b,a\in\mathbb{Z}$. Notice that $\hcf(a,b)=\hcf(\pm a,\pm b)$, so we can assume that $a$ and $b$ are positive. We can also assume $a<b$. 

The algorithm works inductively as follows. Let $d=\hcf(a,b)$.

\begin{itemize}
\item By the remainder theorem, $$b=aq_{1}+r_{1}$$ for some $q_{1}\in\mathbb{Z}$ and $0\leq r_{1}<a$. 
\item Apply the remainder theorem again to $a$ and $r_{1}$, so we can write $$a=q_{2}r_{1}+r_{2}$$ for some integers $q_{2}$ and $0\leq r_{2}<r_{1}$. 
\item Inductively, assume we have chosen integers $q_{1},...,q_{k}$ and $r_{1},...,r_{k}$ so that 
\[
r_{j-2}=r_{j-1}q_{j}+r_{j}.\] 
We then use the remainder theorem to find $q_{k+1}$ and $0\leq r_{k+1}<r_{k}$ so that $$r_{k-1}=q_{k+1}r_{k}+r_{k+1}.$$ 
\item Eventually, this process has to terminate with $r_{n+1}=0$ for some $n$, since the $r_{j}$ are decreasing by $1$ each step of the algorithm and the $r_{j}$ are nonnegative. If this is the case, then $r_{n}$ is our hcf. 
\end{itemize}

To see why this last item is true, we will show two things:

\begin{itemize}
\item We first show $d|r_{n}$. Note that as $r_{1}=b-aq_{1}$, \hyperref[l:easy-lemma]{The Easy Lemma}  implies $d|r_{1}$ since $d|a$ and $d|b$. Again, since $r_{2}=a-q_{2}r_{1}$ and $d$ divides both $a$ and $r_{1}$, \hyperref[l:easy-lemma]{The Easy Lemma}  implies $d|r_{2}$. Inductively, if we have shown that $d|r_{j}$ for all $j\leq k$, then $r_{k+1}=r_{k}-q_{k}r_{k-1}$, so \hyperref[l:easy-lemma]{The Easy Lemma}  implies $d|r_{k}$. Thus, by induction $d|r_{k}$ for all $k$. In particular, $d|r_{n}$. 
\item Now we show $r_{n}|d$. Note that since $r_{n+1}=0$, we have $r_{n-1}=q_{n+1}r_{n}$, so $r_{n}|r_{n-1}$. Then $r_{n-2}=q_{n}r_{n-1}+r_{n}$, and thus $r_{n}|r_{n-2}$ by \hyperref[l:easy-lemma]{The Easy Lemma} . Working our way backwards, we can get that $r_{n}|a$ and $r_n|b$, so $r_{n}$ is a common factor to both $a$ and $b$. Since $d$ is the highest common factor, we must have that $r_{n}|d$. Hence, $r_{n}=d$. 
\end{itemize}

From the above process, we also get the following:


\begin{theorem}[Bezout’s  Identity]
\label{t:lincombhcf}
 Let $a$ and $b$ be non-zero integers. Then there are integers $s$ and $t$ so that $\hcf(a,b)=sa+tb$. 
\end{theorem}

\begin{proof}
Notice that in the algorithm described above, let $S=\{sa+tb:s,t\in\mathbb{Z}\}$. Then $r_{1}\in S$. Since $r_{2}=a-q_{2}r_{1}$ and $r_{1}\in S$, we have $r_{2}\in S$ as well. In this way, we can show by induction that $r_{j}\in S$ for all $j$, so in particular, $\hcf(a,b)=r_{n}\in S$, which implies the claim.
\end{proof}





If the algorithm was a bit to digest, let's try an example:

\begin{example}
 Let us find
$\textcolor[rgb]{0.98,0.00,0.00}{\mathrm{hcf}}(17,91)$. 
\begin{multicols}{2}
We have:
\begin{align*}{\color{purple} 91}&=5\cdot {\color{RoyalBlue} 17}+{\color{ForestGreen} 6},\\
{\color{RoyalBlue} 17}&=2\cdot {\color{ForestGreen} 6}+{\color{red} 5}\cdot 1,\\
{\color{ForestGreen} 6}&=1\cdot {\color{red} 5}+{\color{magenta} 1}.
\end{align*}

Hence, $\mathrm{hcf}(17,91)=1$.


We can reverse our steps:
\begin{align*}
1&=6-1\cdot 5\\
 &=6-(17-2\cdot 6)\\
 &=-17 +3\cdot 6\\
 &=-17 +3(91-5\cdot 17)\\
 &= 3\cdot 91 -16\cdot 17,
\end{align*}

\end{multicols}
Thus, we have found $s,t$ with $\hcf(17,91) = 17\cdot s+91\cdot t$: $s=-16$ and $t=3$.

\end{example}

Bezout's Identity is incredibly useful, even in examples that seem to have nothing to do with the hcf:

\begin{exercise}
Let $a$ and $b$ be coprime integers. Show that
\begin{description}
\item[(i)] If  $a|c$ and $b|c$ then $ab|c$.
\item[(ii)] If $a|bc$ then $a|c$.
\end{description}
Show that both parts of this result can fail if $a$ and $b$ are not coprime.
\end{exercise}

\begin{solution}
\begin{description}
\item[(i)] Since $a$ and $b$ are coprime, we can find integers $s$ and $t$ so that $sa + tb = 1$. If $a|c$ then we can write $c = ax$, and if $b|c$ then we can write $c = by$ for some $x,y \in \Z$. Then $c = csa + ctb = (by)sa + (ax)tb = (ab)sy + (ab)xt$, so $ab | c$.
\item[(ii)] As in (i), $c = csa + ctb$. If $a |bc$ then write $az = bc$ for some integer $z$, and then we have $c = csa + t(az) = a(cs + tz)$, so $a|c$.
\end{description}
\end{solution}

\begin{exercise}
Show that if $a, b$ are postive integers, and $d=hcf(a,b)$ then there exist positive integers $s, t$ such that $d=sa-tb$.
%\item Find such positive integers $s, t$ in the following examples (from Exercise 10.1):
%\begin{itemize}
%\item $a=17, b=29$.
%\item $a=552, b=713$.
%\item $a=345, b=299$.
%\end{itemize}
%\end{itemize}
\begin{remark} Note that this exercise differs from Bezout’s  Identity, in that we ask $s, t$ to be \underline{positive}.\end{remark}


We can now solve all linear diophantine equations in two variables. 

\begin{exercise}
Find all solutions $(s,t)$ to $sm+tn=c$ for all intgers $m,n,c$. 
\end{exercise}




%
\end{exercise}

\begin{solution}
\begin{claim} If $a,b$ are positive integers and $d=hcf(a,b)$, we can find positive integers $s, t \in\mathbb{Z}$ such that $d=sa-tb$.\end{claim}

\begin{proof}  By the Euclidean algorithm, we can find integers $s,t\in\mathbb{Z}$, such that
\begin{equation}d=sa+tb\label{hcfeqn}\end{equation}. Since $0<d\leq a,b$, at most one of $s$ and $t$ can be positive, and at most one can be non-positive (i.e. zero or negative).  If it happens that $t$ is negative, then we will have produced the desired expression for $d$ (by writing $d=sa - (-t)b$).

In case $t$ is non-negative, we can apply the following observation:
\begin{lemma} The integers $s$ and $t$ satisfy Equation \eqref{hcfeqn} if, and only if, the integers $s+kb$ and $t-ka$ do as well, for any integer $k$.\end{lemma}
\begin{proof} We have $(s+kb)a + (t-ka)b = sa+tb$, simply by expanding the LHS, so one finds the same condition for both pairs.\end{proof}

Choosing $k$ sufficiently large, we can replace $t$ by $t-ka$ so that it is negative.  It follows then that $s+kb$ is positive.
\end{proof}

\end{solution}





%\begin{solution}
%\begin{description}
%\item[(i)] Since $a$ and $b$ are coprime, we can find integers $s$ and $t$ so that $sa + tb = 1$. If $a|c$ then we can write $c = ax$, and if $b|c$ then we can write $c = by$ for some $x,y \in \Z$. Then $c = csa + ctb = (by)sa + (ax)tb = (ab)sy + (ab)xt$, so $ab | c$.
%\item[(ii)] As in (i), $c = csa + ctb$. If $a |bc$ then write $az = bc$ for some integer $z$, and then we have $c = csa + t(az) = a(cs + tz)$, so $a|c$.
%\end{description}
%\end{solution}

\subsection{Corollaries of Euclidean Algorithm}

 \hyperref[t:lincombhcf]{ Bezout's Identity} (Theorem \ref{t:lincombhcf}) and \hyperref[l:easy-lemma]{The Easy Lemma}  are a powerful one-two punch for a lot of problems. Below we prove some useful corollaries using these two results. 

 \begin{corollary} 
 \label{c:c|abc|b}
 Let $a,b,c\in\Z$. Suppose $c\neq
0$, $c|ab$ and $\hcf(a,c)=1$. Then $c|b$. 
\end{corollary}

\begin{proof}
There are integers $s$ and $t$ such that $1=\mathrm{hcf}(c,a)=sa+tc$. This gives $b=sab+tcb$.  Then $c|b$ by \hyperref[l:easy-lemma]{The Easy Lemma} . 
 \end{proof}
 
\begin{corollary} Let $a,c\in\Z$ and let $d\in\mathbb{Z}$ be so that $d|a$ and $d|c$.  Then $d|\hcf(a,c)$. 
\end{corollary}

\begin{proof}
By  \hyperref[t:lincombhcf]{ Bezout's Identity}, there are integers $s$ and $t$ such that
$\hcf(a,c)=sa+tc$.
 Then as $d|a$ and $d|b$, we have $d|\hcf(a,c)$ by \hyperref[l:easy-lemma]{The Easy Lemma}. 
\end{proof}

\begin{corollary} Let $d$ be a common divisor of
 $a$ and $c$, which is divisible by all divisors of $a$ and $c$. Then $d = \pm hcf(a,c)$.
\end{corollary}

\begin{proof}
By the previous corollary, $d|\hcf(a,c)$, and since $d$ is divisible by all divisors, it is divisible by $\hcf(a,b)$, so now we apply Corollary \ref{l:abba}.
\end{proof}


Here are some versions of the above corollaries when some of the numbers involved are primes:

\begin{corollary}
\label{c:p|ab}
If $a,b\in\mathbb{Z}$, $p$ is prime, and $p|ab$, then either $p|a$ or $p|b$ (or both).
\end{corollary}

This just follows from Corollary \ref{c:c|abc|b} by letting $c=p$. 







%\begin{claim}
%If there are integers $m,n$ so that $ma+nb=1$, then $\hcf(a,b)=1$. 
%\end{claim}
%  
%\begin{proof}
%Recall $ma+nb=1$ for some $m,n$.   By Easy Lemma, $\hcf(a,b)|ma+nb=1$, only possible if $\hcf(a,b)=1$.
%\end{proof}
%This is an application of the Euclidean algorithm to get hcf as a linear combination, and then \hyperref[l:easy-lemma]{The Easy Lemma} .
 \begin{corollary}
 \label{c:p|p...p}
If $n=p_1\cdots p_k$ is a product of \emph{primes}, and if $p$ prime divides $n$, then $p=p_i$ for some $i=1,\ldots, k$.
\end{corollary}
 
 \begin{proof}
We prove this by induction. The base case $k=2$ is true from the previous corollary, since if $p|p_{1}p_{2}$, then $p|p_{1}$ or $p|p_{2}$, and the only way this is possible is if $p=p_{1}$ or $p=p_{2}$. For the induction step, suppose $p|p_{1}...p_{k+1}$. Again by the  previous corollary, either $p|p_{k+1}$ or $p$ divides $p_1\cdots p_k$. Thus, either $p=p_{k+1}$, or, by the induction hypothesis, $p=p_i$ for some $i=1,2,...,k$. This proves the corollary.

 \end{proof}
 
 
 Let's use these results to solve a simple diophantine equation. 
 
 \begin{example}
 \label{ex:2x=5y}
 Find all integer solutions to $2x=5y$. 
 \end{example}
 
 Suppose $(x,y)$ are integers solving $2x=5y$. Then Corollary \ref{c:p|ab} implies $2|y$ and $5|x$, so $y=2z$ and $x=5w$ for some integers $z$ and $w$. Inserting these into the original equation, we get a new equation 
 \[
 2(5w)=5(2z) \;\; \Rightarrow 10w=10z.
 \]
 Thus, $w=z$. Hence, any solution $(x,y)$ must be of the form $(x,y)=(5w,2w)$ for some integer $w$. We can also see that each pair of integers of the form $(5w,2w)$ is a solution. Thus, the solutions to $2x=5y$ are exactly all pairs of integers $\{(5w,2w):w\in\mathbb{Z}\}$. 
 



\section{Factoring integers into primes}

\subsection{The Fundamental Theorem of Arithmetic (FTA)}



The fundamental theorem of arithmetic says that all integers have a \emph{unique} factorization as a product of powers of prime numbers. 

\begin{theorem}[The Fundamental Theorem of Arithmetic (FTA)] \label{t:FTA} Let $n\geq 2$ be an integer.
\begin{itemize}
\item (Existence) Then $n$ is equal to a product $n=p_1^{r_{1}}\cdots p_k^{r_{k}}$ of powers of prime numbers, where $p_1< \ldots < p_k$ and $r_{i}>0$ for all $i$.
\item (Uniqueness) The factorization is unique: If we also have
$$ p_1^{r_{1}}\cdots p_k^{r_{k}} = n = q_1^{s_1}\cdots q_\ell^{s_{\ell}}$$
where  $q_{1}<\cdots < q_{\ell}$ are primes and $s_{i}> 0$, then $k=\ell$, $p_i=q_i$, and $r_{i}=s_{i}$ for all $i$.
\end{itemize}
\end{theorem}
%If some of the $p_i$'s are repeated, we can collect them into powers, and write instead,
%$$n=p_1^{a_1}\cdots p_l^{a_l},$$
%with $p_1<\cdots < p_l$ all prime, and $a_i$ positive integers.


%
%\begin{frame}
%\frametitle{Match the claim to its proof (note: they are out order)}
% \begin{claim}[1 - existence of prime decomposition] Every integer $n$ can be written as a product of primes $n=p_1\cdots p_n$.\end{claim}
%
% \begin{claim}[2 -uniqueness of prime decomposition] Every integer $n$ can be written as a product of primes $n=p_1\cdots p_n$ in a \emph{unique} way.
%\end{claim}
%
% \begin{claim}[3]
%The $\operatorname{hcf}(a,b)$ is divisible by any common factor of $a$ and $b$.
%\end{claim}
%
% \begin{claim}[4]
%If $n=p_1\cdots p_n$ is a product of \emph{prime} numbers, and if $p$ prime divides $n$, then $p=p_i$ for some $i=1,\ldots, n$.
%\end{claim}
%
% \begin{claim}[5]
%If $a,b\in\mathbb{Z}$ are coprime, $p$ is prime, and divides $ab$, then either $p$ divides $a$ or $p$ divides $b$ (or both).
%\end{claim}
%
%
%\end{frame}



We will split the proof into three lemmas:


\begin{lemma}[Existence of prime decomposition, Part I]
Every integer $n\geq 2$ can be written as a product of primes $n=p_{1}\cdots p_{k}$. 
\end{lemma}

\begin{remark}
If $n$ is prime, this statement still makes sense: we just interpret $n$ as being the product of just one number, $n$ itself. 
\end{remark}

\begin{proof}
We prove by strong induction on $n$. The case $n=2$ immediately holds (taking into account the previous remark). For the induction step, suppose the theorem holds for all integers $n<N$.  If $N$ is prime, there is nothing to prove; otherwise, if $N$ is not prime, then $N=a\cdot b$ for some positive integers $a$ and $b$ greater than $1$. Both $a$ and $b$ can be  decomposed by the strong induction hypothesis, thus so can $n=ab$.
\end{proof}

\begin{lemma}[Existence of prime decomposition, Part II]
Every integer $n\geq 2$ can be written as a product of powers of primes $n=p_{1}^{r_{1}}\cdots p_{k}^{r_{k}}$ where $p_{1}<\cdots < p_{k}$ and $r_{i}\geq 0$. 
\end{lemma}

\begin{proof}
By the previous lemma, $n=q_{1}\cdots q_\ell$ for some primes $q_{1}\cdots q_{\ell}$ that are not necessarily distinct. If $p_{1}<p_{2}<\cdots < p_{k}$ are the {\it distinct} primes that appear in the list $q_{1},...,q_{\ell}$, let $r_{i}$ denote the number of times that $p_{i}$ appears in the list. Then
\[
n=q_{1}\cdots q_{\ell} = p_{1}^{r_{1}}\cdots p_{k}^{r_{k}}\]
which proves the lemma.
\end{proof}



\begin{lemma}[Uniqueness of prime decomposition] Every integer $n$ can be written as a {\it unique} product of powers of primes $n=p_1^{r_{1}}\cdots p_k^{r_{k}}$.
\end{lemma}
  
  \begin{proof}
  Suppose $p_1^{r_{1}}\cdots p_k^{r_{k}} = n = q_1^{s_{1}}\cdots q_\ell^{s_{\ell}}$ are two decompositions. By cancelling any common factors, we can assume that no $p_i$ equals any $q_j$. If there are  any $p_i's$ and $q_j's$ remaining,  Corollary \ref{c:p|p...p} implies each $p_{i}$ equals some $q_j$, which is a contradiction, thus there can be no terms remaining, so the two factorizations must have been equal.
  \end{proof}



Let's use the FTA to solve another simple-looking diophantine equation. 

\begin{example}
Find all integer solutions to $x^{2}=y^{5}$. \\

We start by finding some easy solutions that will reduce the cases that we have to consider and hopefully make our lives easier. \\

Firstly, we want to use the FTA to $x$ and $y$, but this only applies to positive integers at least $2$, so let's first solve our problem in this case. Hence, assume $x,y\geq 2$. We will try to narrow down what $x$ and $y$ must look like using the FTA: The FTA tells us that we can find primes $p_{1}<\cdots <p_{k}$, $q_{1}<\cdots <q_{\ell}$, and integers $r_{i}\geq 0$ and $s_{j}\geq 0$ for $i=1,...,k$ and $j=1,...,\ell$ so that 
\[
x=p_{1}^{r_{1}}\cdots p_{k}^{r_{k}},\;\; y= q_1^{s_{1}}\cdots q_\ell^{s_{\ell}}.
\]
Plugging this into $x^2=y^5$, we get 
\[
p_{1}^{2r_{1}}\cdots p_{k}^{2r_{k}} = q_1^{5s_{1}}\cdots q_\ell^{5s_{\ell}}.
\]

The FTA says these two factorizations must equal, so $k=\ell$, $p_i=q_i$, and $2r_{i}=5s_{i}$ for $1\leq i\leq k$. By Example \ref{ex:2x=5y}, this means $r_{i}=5w_{i}$ and $s_{i}=2w_{i}$ for some integer $w_{i}$, and since $r_{i}$ and $s_{i}$ are nonnegative, $w_{i}$ is, too. 

Thus,
\[
x=p_{1}^{r_{1}}\cdots p_{k}^{r_{k}}=p_{1}^{5w_{1}}\cdots p_{k}^{5w_{k}}= \left(p_{1}^{w_{1}}\cdots p_{k}^{w_{k}}\right)^{5}
\]
and 
\[
y=
q_1^{s_{1}}\cdots q_\ell^{s_{\ell}}
=p_1^{2w_{1}}\cdots p_\ell^{2w_{\ell}} = \left(p_1^{w_{1}}\cdots p_\ell^{w_{\ell}}\right)^2.
\]
If we let $z=p_1^{w_{1}}\cdots p_\ell^{w_{\ell}}$, we see that $(x,y)=(z^{5},z^{2})$. Thus, we have shown that all integer solutions $x,y\geq 2$ are of the form $(x,y)=(z^{5},z^{2})$ for some $z\in\mathbb{N}$.

Now we need to take care of when $x$ or $y$ is less than $2$, since then we can't use the FTA. Note that if $x=1$, then $y=1$. If $x=0$, then $y=0$. If $x=-1$, then $y=1$. Finally, if $x\leq -2$, then $x^2=y^5$ implies $y$ is a positive integer not equal to $1$, so $y\geq 2$, and moreover $(-x)^2=y^5$, so now $(-x,y)$ is a solution with $-x,y\geq 2$, so our previous work shows that $(-x,y)=(z^5,z^2)$ for some $z\in\mathbb{N}$). Thus, $(x,y)=(-z^5,z^2)$. 

In summary, we have shown that all solutions must be either $(\pm 1, 1)$, $(0,0)$ or of the form $(\pm z^5,z^2)$ for some $z\in\mathbb{N}$. More succinctly, all solutions must be of the form $( z^5,z^2)$ for some $z\in\mathbb{Z}$. One can also easily check that every pair in this set is also a solution to $x^2=y^5$, thus $S$ is {\it exactly} the set of solutions. 

\end{example}



{\bf Pro-tip:} Look for {\it reductions} or ways of simplifying your problem from the start by reducing the number of cases you have to investigate. The next problem is a good example of how to find reductions.

\begin{example}
Find all integer solutions to $x^2-y^2=91$. \\


Let's make a few reductions first:
\begin{itemize}
\item Notice that if $(x,y)$ is a solution, then so is $(\pm x,\pm y)$, and so we can assume that $x,y\geq 0$, since then the other solutions will be of the form $(\pm x,\pm y)$. 
\item Moreover, we can't have $x=y$ since then the equation is not satisfied, so assume $x\neq y$. We can also assume $x>y$, since $x<y$ would imply $x^2<y^2$, so $x^2-y^2<0<51$. 
\item Finally, we can't have either $x$ or $y$ equal to zero, since $91$ is not a perfect square. 
\end{itemize}
Thus, after these reductions, we can assume $x>y>0$.\\

Note that 
\[
91=7\cdot 13 = x^2-y^2=(x-y)(x+y).\] 
We would like to use the FTA to say that the factorizations of $x-y$ and $x+y$ must multiply up to $91$, but we can only use it for positive integers greater than $1$. So let's first consider the case when either $x-y=1$ or $x+y=1$. Since we are assuming $x>y$, this means that we can only have $x-y=1$ (otherwise, since $x>y>0$, $x+y=1$ would imply $x-y<x+y-1=0<x-y$, a contradiction). 

If $x-y=1$, then this means 
\[
x+y=(x+y)(x-y)=x^2-y^2=91
\]
So we have two linear equations $x-y=1$ and $x+y=91$, and solving gives $x=46$ and $y=45$. \\

Now let's consider when neither $x\pm y$ are equal to $1$, so $x+y>x-y\geq 2$. Then by prime factorization, $x-y$ and $x+y$ are integers at least $2$ that are products of primes that multiply up to $91=7\cdot 13$. Thus, the only options are that $x-y=7$ and $x+y=13$. Solving these two equations gives $x=10$ and $y=3$. 

Thus, all the {\it positive} solutions $(x,y)$ are $(46,45)$ and $(10,3)$. Thus, {\it all} solutions are just 
\[
(46,45), \; (-46,45), \; (46,-45), \; (-46,-45), \; (10,3), \; (-10,3), \; (10,-3), \; (-10,-3).
\]



\end{example}



\section{Some consequences of the FTA}
\subsection{Finding divisors via FTA}
If we know the prime decompositions of two integers $m$ and $n$, it is easy to tell whether $m$ divides $n$:

\begin{theorem}
Let $n=p_1^{a_1}\cdots p_k^{a_k}$ be a prime decomposition (i.e. $p_i$s are prime, $p_1<\cdots <p_k$, and $a_i$s are positive). Then $m$ divides $n$ if, and only if:
\begin{equation}
\label{e:m=p1...pk}
m = p_1^{b_1}\cdots p_k^{b_k}, \quad \textrm{with each $0\leq b_i\leq a_i$}.
\end{equation}
\end{theorem}

\begin{proof}
\begin{itemize}
\item  ($\impliedby$): If \eqref{e:m=p1...pk} holds, then
\[
n=m \cdot  p_1^{b_1-a_{1}}\cdots p_k^{b_k-a_{k}}
\]
so $m|n$.
 \item ($\implies$): Suppose $m|n$, then $n=mc$ fr some positive integer $c$.  Then $m$ and $n$ have prime decompositions whose product is the prime decomposition for $n$ as shown below:
$$\underbrace{p_1^{a_1}\cdots p_k^{a_k}}_n = \underbrace{q_1^{c_1}\cdots q_l^{c_l}}_m\underbrace{r_1^{d_1}\cdots r_s^{d_s}}_c.$$
The FTA implies each $q_i$ and $r_i$ equals to some $p_j$. To ease notation, we'll assume $q_{i}=r_{i}=p_{i}$ for all $i$, but that $c_{i}=0$ if $p_{i}$ didn't appear as one of the primes $q_{i}$ originally, and similarly for the $d_{i}$. Then the product above is in fact 
$$\underbrace{p_1^{a_1}\cdots p_k^{a_k}}_n = {q_1^{c_1}\cdots q_l^{c_l}}{r_1^{d_1}\cdots r_s^{d_s}} 
=p_{1}^{c_{1}+d_{1}}\cdots p_{k}^{c_{k}+d_{k}}.
$$

The FTA now implies each power $c_i+d_i$ equals to $a_j$. In particular, this means $m=p_{1}^{c_{1}}\cdots p_{k}^{c_{k}}$ where $0\leq c_{k}\leq a_{k}$. 
  \end{itemize}
\end{proof}


\begin{example}
What are all the divisors of $360$?
\end{example}

First, we factorize $360$: we can see that $36=4\cdot 9=2^2\cdot 3^2$, and so $360 = 10\cdot 2^2\cdot 3^2 = 2^3\cdot 3^2\cdot 5$. To list all the divisors, we just have to look at the values $2^j\cdot 3^{k}5^{\ell}$ where $0\leq j\leq 3$, $0\leq k\leq 2$, and $0\leq \ell\leq 1$. The possible powers of $2$ are $1, \; 2, \; 4, \; 8,$. The possible powers of $3$ are $1,3,9$, so we multiply the powers of two by these to get 

\[
\begin{array}{cccc}
1, &  2, &  4, &  8, \\
3, & 6, &  12, &  24, \\
 9, &  18, &  36, &  72.
 \end{array}
 \]
 
 The possible powers of $5$ are just $1$ and $5$, so we can just multiply these numbers by $1$ and $5$ to get 
 \[
 \begin{array}{cccccccccccc}
1, &  2, &  4, &  8, 
& 3, & 6, &  12, &  24, 
&  9, &  18, &  36, &  72, \\ 
5, &  10, &  20, &  40,
& 15, & 30, & 60,  &  120, 
&  45, &  90, &  180, &  360. 
 \end{array}
 \]


\subsection{LCM and HCF via prime factorizations}


\def\lcm{\rm lcm}
\begin{definition}
The least common multiple $\lcm(a,b)$ of positive integers $a$ and $b$ is the smallest positive integer divisible by both $a$ and $b$.   
\end{definition}

For example, $\lcm(15,12)=60$.

 \begin{theorem}
 \label{t:lcm}
Let $a$ and $b$ have prime factorizations,
$$a=p_1^{r_1}\cdots p_m^{r_m}, \;\; b=p_1^{s_1}\cdots p_m^{s_m}.$$  Here $p_i$'s are distinct, but $r_i$ and $s_i$ are allowed to be zero.  Then:
\begin{itemize}
 \item $\hcf(a,b) = p_1^{min(r_1,s_1)}\cdots p_m^{min(r_m,s_m)}$.
 \item $\lcm(a,b) = p_1^{max(r_1,s_1)}\cdots p_m^{max(r_m,s_m)}$.
 \item $\lcm(a,b) = ab/\hcf(a,b)$.
\end{itemize}
\end{theorem}

We leave the proof of this as an exercise. 

\begin{example}
If $a=120=2^3\cdot 3\cdot 5$ and $b=36=2^2\cdot 3^2$, then 
$\hcf=2^2\cdot 3=12$ and $\lcm=2^3\cdot 3^2 \cdot 5=360=\frac{120\cdot 36}{12}$,
\end{example}

%
%
%\begin{example} What is the smallest positive integer that can be written in the form $375a + 147b$ where $a$ and $b$ are integers?
%\end{example}
%
%This was a question on University Challenge \href{https://www.youtube.com/watch?v=YoVvcMAV2YU#t=703}{(S43E28 Queen's, Belfast vs Southampton)}. As you can see from the video, no one got it right. But now you'll be ready! 
%
%Let's find the prime factorizations of $375$ and $147$. We can just keep dividing by numbers we think divide:
%\[
%375 = 3\cdot 125=3\cdot 5^{3}, \;\; 147 = 7\cdot 21 = 7^2\cdot 3.
%\]
%Now we can see that $\hcf(375,147)=3$. By Bezout’s  Identity, we can find $a,b\in\mathbb{Z}$ so that $375a+147b=3$. We cannot make this number any smaller with different choices of $a,b$ by \easylemma, thus $3$ is the smallest such number I can express as $375a+147b$.
%
%

\subsection{Powers}

In the theorems below we use a useful trick with the FTA. Given two integers $a,b\geq 2$, let $p_{1},...,p_{k}$ be the primes that appear in either of the factorizations for $a$ and $b$. Then we can express $a$ and $b$ as powers of {\it the same primes}
\[
a=p_{1}^{r_{1}}\cdots p_{k}^{r_{k}}, \;\; b=p_{1}^{s_{1}}\cdots p_{k}^{s_{k}}
\]
where $0\leq r_{i}$ and $0\leq s_{i}$ for all $i$, but $r_{i}=0$ if $p_{i}$ does not appear in the prime factorization of $a$ and $s_{i}=0$ if $p_{i}$ does not appear in the factorization of $b_{i}$. 




\begin{theorem}
\label{t:perfectsquare}
 Let $n$ be a positive integer.  Then $\sqrt{n}\in\mathbb{Q}$ $\iff$ $n$ is a perfect square.
 \end{theorem}
 
\begin{proof}
Suppose $\sqrt{n}\in\mathbb{Q}$, then there are positive integers $r,s$ so that $\sqrt{n}=\frac{r}{s}$. Hence, $s^2n=r^2$.  By the prime factorization theorem, we can write each of $s,n,r$ as a product of powers of primes. Let $p_{1},...,p_{k}$ be all the primes that appear in {\it either} of $s,n,r$. Then we can write 
\[
n= p_{1}^{a_{1}}\cdots p_{k}^{a_{k}},
\;\;\;\;
r= p_{1}^{b_{1}}\cdots p_{k}^{b_{k}},\;\;\;\;\mbox{and} 
\;\;\;\;
s= p_{1}^{c_{1}}\cdots p_{k}^{c_{k}},\]
where $0\leq a_{i},b_{i},c_{i}$ (and $a_{i}=0$ if $p_{i}$ does not appear in the prime factorization of $n$, $b_{i}=0$ if $p_{i}$ does not appear in the prime factorization of $r$, and $c_{i}=0$ if $p_{i}$ does not appear in the prime factorization of $s$). Then $s^2n=r^2$ implies
\[
p_{1}^{2c_{1}+a_{1}}\cdots p_{k}^{2c_{k}+a_{k}}=p_{1}^{2b_{1}}\cdots p_{k}^{2b_{k}}.
\]
By the FTA, we must have $2c_{i}+a_{i}=2b_{i}$ for all $i$. In particular, $a_{i}$ is even for all $i$, so $a_{i}=2d_{i}$ for some $d_{i}\geq 0$. This implies
\[
n=p_{1}^{a_{1}}\cdots p_{k}^{a_{k}}=p_{1}^{2d_{1}}\cdots p_{k}^{2d_{k}}
=(p_{1}^{d_{1}}\cdots p_{k}^{d_{k}})^{2},
\]
that is, $n$ is a perfect square. 
\end{proof}

\begin{theorem}
\label{t:abn}
If $a,b\in\mathbb{N}$ are coprime, and $ab$ is an $n$th power, then so are $a$ and $b$.
\end{theorem}

\begin{proof}
 If $a=1$, then we have $ab=b$ is an $n$th power, so the theorem is trivial in this case, and similarly if $b=1$, so we can assume $a,b \geq  2$, so we can apply the FTA.


Let $a=p_{1}^{r_{1}}\cdots p_{k}^{r_{k}}$ and $b=q_{1}^{s_{1}}\cdots q_{\ell}^{s_{\ell}}$ be the prime factorizations of $a$ and $b$. Since $a$ and $b$ are coprime, they share no common prime factors, so $p_{i}\neq q_{j}$ for all $j$.  By assumption $ab=c^n$ for some $c$. Let $w_{1}^{t_{1}}\cdots w_{j}^{t_{j}}$ be the prime facorization for $c$, then 
\[
ab = p_{1}^{r_{1}}\cdots p_{k}^{r_{k}}q_{1}^{s_{1}}\cdots q_{\ell}^{s_{\ell}}
=c^n=w_{1}^{nt_{1}}\cdots w_{j}^{nt_{j}}.
\]
By the FTA,  these prime factorizations are equal, which means for each $i$,  $p_{i}=w_{j}$ for some $j$, and $r_{i}=nt_{j}$. In particular, $n|r_{i}$ for all $i$, and so $a$ is an $n$th power. This holds similarly for $b$. 


\end{proof}


Let's do an example of another diophantine equation. This one is from Liebeck (but is not proven correctly there).

\begin{example}
Find all integer solutions to $4x^2=y^3+1$. \\

Suppose $x,y$ are integer solutions to this equation. Let's try to rearrange and factor:
\[
y^3=4x^2-1=(2x-1)(2x+1).
\]
By the Easy Lemma, the hcf of $2x-1$ and $2x+1$ must divide $2x+1-(2x-1)=2$, so the hcf is either $1$ or $2$. However, since both of these numbers are odd, the hcf must actually be $1$. Hence, $2x\pm 1$ are coprime.  

We claim that $2x\pm 1$ are both cubes. This is immediate from Theorem \ref{t:abn} if they are both positive; if $2x-1<2x+1$, then $-2x+1,2x+1>0$, and so Theorem \ref{t:abn} again implies they are both cubes; if $2x-1<2x+1<0$, then $1-2x,-1-2x>0$, and then these two numbers are again cubes by Theorem \ref{t:abn}. This proves the claim. 

Thus, there are $m,n\in\mathbb{Z}$ so that $m^3=2x+1$ and $n^3=2x-1$. Then 
\[
2=2x+1-(2x-1)=m^3-n^3=(m-n)(m^2+mn+n^2),\]
so $m-n|2$, hence it is either $1$ or $2$. If it is $1$, then $m=n+1$, and so
\[
2=2x+1-(2x-1)=m^3-n^3=(n+1)^3-n^3=3n^2+3n+1\geq 9\]
since $2x-1\neq 0$ implies $n\neq 0$, this is a contradiction. The case that $m-n=1$ can be handled similarly. Thus, there are {\it no integer solutions} to $4x^2=y^3+1$. 

\end{example}



 \section{Application: Pythagorean Triples*}
 
In this section we will classify all Pythagorean Triples, that is, all positive integer solutions to 

\[
x^2+y^2=z^2.
\]


This is a bit more involved than other diophantine problems and harder to figure out on your own on a homework problem (so it is not required reading), but we'll give a proof here since it's a neat application of what we've learned and can test your knowledge of the material in this chapter. \\

We narrow down the solutions in a few steps:

\begin{itemize}
\item If $x=0$, then we must have $y^2=z^2$, and so $y=\pm z$. Thus, we know all solutions if $x=0$, so let's assume $x> 0$. Similarly, we can assume $y,z> 0$.
\item We can assume that $x,y$ and $z$ are coprime (that is, no two of them share a common factor other than $1$). To see this, suppose $d=\hcf(x,y)$. Then $a=x/d$ and $b=y/d$ are coprime, and then
\[
z^2= x^2+y^2 = d^2(a^2+b^2).
\]
By Theorem \ref{t:abn}, $a^2+b^2=c^2$ for some integer $c$, and so $z^2=d^2 c^2=(cd)^2$. Hence, $(x,y,z)=(da,db,dc)$ for some other Pythagorean Triple $(a,b,c)$ where $a$ and $b$ are coprime.  Thus, if we find all solutions $(x,y,z)$ where $x$ and $y$ are coprime, then all other solutions are multiples of these. A similar proof shows that all solutions will be multiples of solutions where $x$ and $z$ are coprime and where $y$ and $z$ are coprime. 
\item Thus, assume $x,y,$ and $z$ are coprime and positive solutions to $x^2+y^2=z^2$. Then either $x$ or $y$ is odd, assume it is $x$. 
\item We now claim $y\pm z$ are coprime. Suppose $z\pm y$ were not coprime. Then there is a prime $q$ that divides them both. But then
\[
q|z+y-(z-y) = 2y,\;\; q| z+y+(z-y) = 2z.
\]
Since $y$ and $z$ are coprime, $q=2$. But then
\[
2|(z-y)(z+y)=z^{2}-y^{2}=x^{2}
\]
which implies $x$ is even, a contradiction.
\item Since $z\pm y$ are coprime and $x^{2}=(z+y)(z-y)$, we know that $z+y=s^{2}$ and $z-y=t^{2}$ for some integers $s$ and $t$ by Theorem \ref{t:abn}. Hence, 
\[
z=\frac{z+y+z-y}{2} = \frac{s^{2}+t^{2}}{2}
\]
Similarly, $y=\frac{s^{2}-t^{2}}{2}$, and finally,
\[
x^{2} =(z-y)(z+y) = s^{2}t^{2}. 
\]
Thus, all positive coprime solutions with $x$ odd are of the form
\[
(x,y,z) = \left( s^2 t^2, \frac{s^2-t^2}{2}, \frac{s^{2}+t^{2}}{2}\right).
\]
\item Finally, we now recall that all solutions are multiples of these soluitions, thus all Pythagorean Triples are of the form
\[
(x,y,z) = \left( as^2 t^2, a\frac{s^2-t^2}{2}, a\frac{s^{2}+t^{2}}{2}\right)
\]
where $a,s,t$ are integers. 
\end{itemize}



\section{Exercises}



\begin{exercise} Show that $n$ and $n^2+n+1$ are coprime. 


\end{exercise}


\begin{exercise} Find all $n\in\mathbb{N}$ so that $n-2|n^2-2$. 

\begin{solution}
Note that $n-2|n^2-4$, so the easy lemma implies 
\[
n-2| (n^2-2-(n^2-4))=2.
\]
Thus $n-2$ must be $\pm 1$ or $\pm 2$, and this is only possible if $n=1,3$ or $4$.
\end{solution}

 
\end{exercise}


\begin{exercise} Find all solutions $x,y\in \Z$ to the following Diophantine equations: 
\begin{itemize}
\item $x^2=16y^2+8y+2$


\begin{solution}
Note that $x^2=16y^2+8y+2$ implies
\[
x^2-1 =16y^2+8y+1=(4y+1)^2
\]
Note that $x^2-1=(x-1)(x+1)$, so their hcf is at most $2$, but $x^2=(4y+1)^2$ which is odd, thus $x\pm 1$ are coprime. Thus, they are perfect squares that differ by $2$, so $x+1=m^2$ and $x-1=n^2$. But then
\[
2= x+1-(x-1) = m^2-n^2 = (m-n)(m+n),
\]
so we must have that $m-n=1$ and $m+n=2$, which has no integer solutions. 
\end{solution}


\item $x^2+2y^2=8z+5$. 


\begin{solution}

\begin{claim}
There are no integer solutions to $x^2+2y^2=8z+5$. 
\end{claim}

\begin{proof}
Suppose $(x,y,z)$ was an integer solution. Notice that $8z+5$ is odd, so $x^2+2y^2$ is odd as well, thus $x^2$ is odd, and so $x$ is odd. Thus, $x=2n+1$ for some integer $n$. Plugging that back into the above, we get 
\[
8z+5 =x^2+2y^2 = (2n+1)^2 +2y^2 = 4n^2 +4n+1 + 2y^2\]
and so
\[
8z+4 = 4n^2+4n+2y^2.\]
Dividing both sides by $2$, we get
\[
4z+2=2n^2+2n+y^3.\]
In particular, $y^3$ must be even, so $y$ is even, hence $y=2m$ for some $m$, so we get
\[
4z+2-2n^2-2n=y^3=8m^3
\]
and dividing through by $2$ gives
\[
2z+1-n^2-n = 4m^3.\]
But notice that $2z+1-n^2-n = 2z-n(n+1)+1$ is odd, because $n(n+1)$ is even and so is $2z$, but $4m^3$ is even, and we get a contradiction. Thus, there are no solutions.
\end{proof}
\end{solution}


\item $x^2 = y^3$.


\begin{solution}

{\bf Claim:} We have the following:
The integer solutions of $x^2=y^3$ are of the form $x=n^3$, $y=n^2$ for some integer $n$.


\begin{proof}
 It is clear to see that letting $x=n^3, y=n^2$ for any integer $n$ solves the equation.  So we need to show that this is in fact necessary, not only sufficient.

Let $(x,y)$ be a solution to $x^2=y^3$. Note that if $(x,y)$ is a solution, then so is $(-x,y)$, and moreover, $y\geq 0$ since $y^3=x^2\geq 0$. Thus, we can assume for the moment that $x,y\geq 0$, since if  $(x,y)$ is a solution with $x<0$, then $(-x,y)$ will be a positive solution. 

Since $x,y\geq 0$, we can look at their prime factorizations. Let us write $x=p_1^{k_1}\cdots p_n^{k_n}$ and $y=q_1^{l_1}\cdots q_m^{l_m}$ for the prime factorizations for $x$ and $y$ respectively, (so we assume $p_i<p_j$ for $i<j$, and $q_r<q_s$ for $r<s$).  Then we have:
$$p_1^{2k_1}\cdots p_n^{2k_n} = x^2=y^3 = q_1^{3l_1}\cdots q_m^{3l_m}.$$
Hence, applying the Fundamental Theorem of Arithmetic to $x^2$, we have $m=n$, $p_i=q_i$, for each $i$, and moreover,
$$2k_i=3l_i,$$
for each $i$.  Since 2 and 3 are coprime, Proposition 10.5 implies that $k_i$ is divisible by 3 and $l_i$ is divisible by 2.  Let $k_i'=\frac{k_i}{3}$.  Let $n:= p_1^{k_1'}\cdots p_n^{k_n'}.$  Then $x=n^3$ and $y=n^2$, as claimed. That is, if $(x,y)$ is a solution with $x,y\geq 0$, then $(x,y)=(n^3,n^2)$ for some $n\geq 0$. Thus, if $(x,y)$ is {\it any } solution, then $(x,y)=(\pm n^3,n^2)$ for some $n\geq 0$, or alternatively, $(x,y)=(n^3,n^2)$ for some $n\in \mathbb{Z}$. 
\end{proof}


\end{solution}



\item $x^2-x=y^3$.


\begin{solution}


{\bf Claim:} The integer solutions of $x^2-x=y^3$ are $x=0,y=0$ and $x=1,y=0$.

\begin{proof}
We may factorize the right hand side as $x(x-1)$.  As $1 = x - (x-1)$ we have $hcf(x,x-1)\leq 1$, hence equals one, and so they are coprime.  Their product is the cube $y^3$ of an integer, and hence by Proposition 11.4, both $x$ and $x-1$ must be cubes.  This happens if and only if $x=0$ or $x=1$; in either case, the equation then implies $y=0$.
\end{proof}
\end{solution}

\item $x^2=y^4-77$.



\begin{solution}

  {\bf Claim:} The integer solutions of $x^2=y^4-77$ are $x=\pm 2, y=\pm 3$ (all four possible combinations are solutions). 


\begin{proof}
 We may rewrite the equation as:
$$77=y^4-x^2 = (y^2-x)(y^2+x).$$
As 77 has a prime factorization $77=7\cdot 11$, we know that exactly one of the following eight situations is true:
$$y^2-x = 7, \textrm{ and } y^2+x=11,$$
$$y^2-x = -7, \textrm{ and } y^2+x=-11,$$
$$y^2-x = 11, \textrm{ and } y^2+x=7,$$
$$y^2-x = -11, \textrm{ and } y^2+x=-7.$$

$$y^2-x = 1, \textrm{ and } y^2+x=77,$$
$$y^2-x = -1, \textrm{ and } y^2+x=-77,$$
$$y^2-x = 77, \textrm{ and } y^2+x=1,$$
$$y^2-x = -77, \textrm{ and } y^2+x=-1.$$

In the second and fourth cases, we can add the two necessary equations to find $y^2=-18$, which clearly has no real, let alone integer solutions.  In the first case, we solve for $x$ to find $x=2$, hence $y=\pm 3$, and in the third case we find $x=-2, y=\pm3$.  Combining these cases gives the claimed result. 

In any of the situations 5-8, we can see that $y^2 = 78$ or $y^2=76$, neither of which have integer solutions in $y$.
\end{proof}

\end{solution}

\item $x^3=4y^2+4y-3$.

\begin{solution}
\begin{claim}
There are no integer solutions of $x^3=4y^2+4y-3$.
\end{claim}

\begin{proof}


 We can factorize the RHS as $(2y+3)(2y-1)$.  As the two factors differ by four, their highest common factor is a divisor of four.  However, they are both odd, so their highest common factor is odd, hence their highest common factor is one, and they are coprime.  Now applying Proposition 11.4, we have that each must be a perfect cube.  However, there are no pairs of perfect cubes which differ by four: 
 
 

\begin{claim}
For any integers $m,n$, $m^3-n^3\neq 4$.
\end{claim}

\begin{proof}
Suppose for the sake of contradiction that $m^3-n^3=1$ for some integers $m$ and $n$. Note that $m>n$, since if $n\geq m$, then $n^3\geq m^3=n^3+1$, which is a contradiction. Then 
\[
4=m^3-n^3= (m-n)(m^2+mn+n^2)
\]
hence $m-n|4$, so $m-n=\pm 1,\pm 2,$ or $\pm 4$. Since $m>n$, this means $m-n= 1,2,$ or $4$. Suppose first that $m-n=1$. Then
\[
4=m^3-n^3=(n+1)^3-n^2 = 3n^2+3n+1.
\]
This implies $3=3n^2+3n=3n(n+1)$, so $n(n+1)=1$, which is impossible. 

If $m-n=2$, then 
\[
4=m^3-n^3=(n+2)^3-n^2 = 6n^2+12n+8.
\]
Then this implies $-4=6n^2+12n$, but this is impossible since $4$ is not divisible by 3. 

Finally, if $m-n=4$, then 
\[
4=m^3-n^3=(n+4)^3-n^2 = 12n^2+48n+64.
\]
so $-60=12n^2+48n=12n(n+4)$, so $-5 = n(n+4)$. For this to be possible, either $n=\pm 1$ or $\pm 5$, but by checking all 4 values, we see that this equation cannot be satisfied,. 
\end{proof}
\end{proof}
\end{solution}

\item $x^3-y^3=7$. 


\begin{solution}

\begin{claim}
The only solutions $(x,y)$ are $(2,1)$ and $(-1,-2)$. 
\end{claim}

\begin{proof}
Suppose $(x,y)$ is a solution. First note that this implies $x^3>y^3$, and so $x>y$. Furthermore, we immediately see that $x=0$ and $y=0$ don't lead to solutions, so we can assume that $x\neq 0 \neq y$ as well. 

Factoring, we see that 
\[
7 = x^3-y^3= (x-y)(x^2+xy+y^2)
\]
Thus, $x-y| 7$, and since $x>y$, this means $x-y>0$, so the only possibilities then are $x-y=1$ or $x-y=7$. We split into two cases:

\begin{itemize}[(a)]
\item $x-y=1$. Then we have that $x=y+1$, and so
\[
7=x^3-y^3= (y+1)^3-y^3 = 3y^2+3y+1,\]
thus $6=3y^2+y$, so $2=y^2+y=y(y+1)$, so we see that $y|2$ and $y+1|2$. The only way this is possible is if either $y=1$ or $y=-2$. By plugging these values into the original equation, we see that the solutions are in this case $(x,y)=(2,1)$ and $(-1,-2)$.
\item Suppose $x-y=7$. Then
\[
7=(x-y)(x^2+xy+y^2)=7(x^2+xy+y^2)
\]
implies
\[
1=(x^2+xy+y^2).
\]
Note that if $|x|\geq |y|$, then $xy\geq -|x|\cdot |y|\geq -|x|^2=-x^2$, and so the above is at least
\[
1\geq x^2-x^2=y^2>0\]
since we are assuming $y\neq 0$. Hence, $y^2=1$, so $y=\pm 1$. Again, we can plug these into our original equation $7=x^3-y^3$ and we find that $y=-1$ leads to no solution and $y=1$ leads to $(x,y)=(2,1)$, which we already found.

Thus, the solutions are just $(x,y)=(2,1)$ and $(-1,-2)$.
\end{itemize}


\end{proof}


\end{solution}

\item $xy= x+y+2y^2+1$.

\begin{solution}

\begin{claim}
The only solutions $(x,y)$ are $(0,-1)$, $(9,2)$, and  $(0,8)$.
\end{claim}

\begin{proof}


First, let's rearrange and factor:
\[
2y^2=xy-x-y-1 = (x-1)(y-1)
\]
and since $\hcf(y,y-1)=1$, we know $y-1|2$, thus $y-1=\pm 1$ or $\pm 2$. We can now just try al these values out to see what $x$ should be:
\begin{itemize}
\item[$y=1$:] In this case, $2=2y^2=(x-1)(y-1)=0$, which is impossible, so there are no solutions.
\item[$y=-1$:] We see that $2=2y^2=(x-1)(y-1)=-2(x-1)$, and so $x=0$, so the solution is $(0,-1)$.
\item[$y=2$:] We see that $8=2y^2 = (x-1)(y-1)=x-1$, and so $x=9$, and the solution in this case is $(x,y)=(9,2)$. 
\item[$y=-2$:] We see that $8=2y^2=(x-1)(y-1)=-3(x-1)$, which holds only if $x=0$, and so $(0,8)$ is the last solution. 
\end{itemize}
\end{proof}
\end{solution}


\item $x^3-x=12y+6$. 

\begin{solution}



\begin{proof}
Suppose $(x,y)$ is a solution, then 
\[
6(2y+1) = x^3-x = (x+1)x(x-1).
\]
Note that $\hcf(x+1,x)=\hcf(x,x-1)=1$, and $\hcf(x-1,x+1)|2$, so it is either $1$ or $2$. Note that since $2\not| 2y+1$, the prime factorization of $(x+1)x(x-1)$ contains exactly one $2$. In particular, $x-1$ and $x+1$ can't both be even, so the only even number is $x$, hence $\hcf(x-1,x+1)=1$. Since $x$ only has one 2 in its prime factorization, $\frac{x}{2}$ is an integer and $x\pm 1$ and $\frac{x}{2}$ are all mutually coprime and
\[
3(2y+1)=(x+1)\frac{x}{2}(x-1).
\]


Thus, by Theorem \ref{t:abn}, 


\end{proof}

\end{solution}




\item$xy+2x+3y=4$.


  
  

\begin{solution}
Write
\[
4=xy+2x+3y =(x+2)(y+3)-6
\]
So now we have 
\[
(x+2)(y+3)=10=2\cdot 5\]
and so the only way this can happen is if either 
\begin{itemize}
\item $x+2=2$, $y+3=5$ (which implies $x=0$ and $y=2$)
\item $x+2=-2$, $y+3=-5$ (so $x=-4$ and $y=-8$)
\item $x+2=5$, $y+3=2$ (so $x=3$ and $y=-1$)
\item $x+2=-5$, $y+3=-2$ (so $x=-7)$ and $y=-5$.
\end{itemize}
Hence, the only solutions are $(x,y)=(0,2),(-4,-8),(3,-1),(-7,-5)$. 
\end{solution}

\item $6x^2=5x^3$.
\end{itemize}



\end{exercise}


\begin{exercise} Find all integer solutions to $x^2-2y^2=1$ given that $y$ is prime. 


\end{exercise}


\begin{exercise} Find all primes $p$ and integers $a,b\in\N$ so that $p^{a}-p^{b}=24$.


\begin{solution}
 If $b=0$, then the above equation implies $p^{a}=25$, so $p=5$ and $a=2$. 
 
 If $b> 0$, then $p|24$, so $p=2$ or $p=3$ and as $a>b$,
 \[
 24 = p^{a}-p^{b} = (p^{a-b}-1)p^{b}
 \]
 so $p^{b}|24$.
 \begin{itemize}[(a)]
 \item If $p=2$, then $b=1$ or $2$, and checking both of these cases turns up no solutions for $a$.
 \item If $p=3$, then $b=1$, and we find that $p^{a} = 24+3^{1}=27=3^3$ so $a=3$. 
 \end{itemize}
 
 Thus, the only solutions $(a,b,p)$ are $(2,0,5)$ and $(3,1,3)$.
 
 \end{solution}

 
 
 
 



\end{exercise}


\begin{exercise} Show that if $f_{n}$ denotes the $n$th Fibonacci number, then $f_{n}$ and $f_{n+1}$ are coprime. 



\begin{solution}
We prove this by induction. Clearly, $f_{1}=f_{2}=1$ are coprime. For the induction step, let $n\geq 2$, we wish to show $d=\hcf(f_{n},f_{n+1})$ are coprime. By the easy lemma,
\[
d|f_{n+1}-f_{n} = f_{n-1},
\]
so $d|f_{n-1}$ and $d|f_{n}$, so $d|\hcf(f_{n},f_{n-1})=1$ by the induction hypothesis. Thus, $d=1$, as desired. 
\end{solution}

\end{exercise}


\begin{exercise} Determine whether the following statement is true or false: if $a,b,k\in\mathbb{N}$ and $a^k|b^k$, then $a|b$. 

\begin{solution}
Our plan is to use Theorem \ref{t:lcm}, so we just need to show that if $p$ is a prime in the decomposition of $a$ and $n$ is its power, then $p^n|b$ (so $p$ appears in the prime decomposition of $b$ and if $m$ is its power, then $n\leq m$).

Note that if $p|a$, then $p|a^k$ so $p|b^k$, hence $p|b$. In particular, any prime appearing in the prime decomposition of $a$ appears in the prime decomposition of $b$. Let $n$ be the power of $p$ in the prime decomposition of $a$ and $m$ the power of $p$ in the decomposition for $b$. Then $p^{kn}|p^{km}$, so $p^n|p^m$. Thus, $n\leq m$. The claim now follows from Theorem \ref{t:lcm}. 
\end{solution}


\end{exercise}


\begin{exercise} Is there a rational number $q$ so that $q^{5}-q^{3}+39=0$?

\end{exercise}


\begin{exercise} Recall that a natural number $n$ is {\it perfect} if it is the sum of all its proper divisors (that is, all positive divisors other than $n$). For example, $1 + 2 + 3 = 6$, so $6$ is perfect. If $p$ is a prime and $n\in\mathbb{N}$, when is $p^{n}$ perfect? 

\begin{solution}
Suppose $p^{n}$ is perfect. Then
\[
p^{n} = 1+p+\cdots + p^{n-1} = \frac{p^{n}-1}{p-1}<p^{n}-1<p^{n},
\]
a contradiction. Thus, $p^{n}$ is {\it never} perfect.
\end{solution}

\end{exercise}


\begin{exercise} Show that $x^{n}+y^{n}=z^{n}$ has integer solutions if and only if it has rational solutions. 

\begin{solution}
 The answer is no: Suppose such rational number $q$ exists. Then $q=\frac{m}{n}$ for some {\it coprime} integers $m$ and $n$. Then
\[
n^5 − n^3m^3 + 39m^5 = 0,\]
which implies that $m|n^5$. Since $m$ and $n$ are coprime, this implies
that $m|n$ by Corollary \ref{c:c|abc|b}. Then $m = 1$, because $m$ and $n$ are coprime. Then
\[
n^5 −n^3 +39=0,
\]
which implies that $n| 39$, so $n\in \{1, −1, 3, −3, 13, −13\}$. Since
$n^5 − n^3 = −39$  is divisible by $n^3$, we see that either $n=1$ or $n=−1$. But
\[1^{5}-1^3+39=39\neq 37 = (-1)^{5}-(-1)^{3}+39\]
which is a contradiction.
\end{solution}


\end{exercise}


\begin{exercise} Find $\hcf(n!+1,(n+1)!)$. 

\begin{solution}
Let $d=\hcf(n!+1,(n+1)!)$. By the Easy Lemma,
\[
d| (n+1)(n!+1)-(n+1)! = n+1
\]
Hence,
\[
d| (n-1)!(n+1)-n! = (n-1)!.
\]
Thus,
\[
d|(n! +1 - n(n-1)!)=1.
\]
Thus, $\hcf(n!+1,(n+1)!)=d=1$.
\end{solution}


\end{exercise}


\begin{exercise} Let 
\[
\Z[\sqrt{-5}]=\{a+b\sqrt{-5}: a,b\in\Z\} = \{a+bi\sqrt{5}: a,b\in\Z\}\subseteq \C.
\]
We say that $p\in \Z[\sqrt{-5}]$ is {\it primo} if the only solutions to $p=ab$ wtih $a,b\in \Z[\sqrt{-5}]$ have either $a$ or $b$ equal to $\pm 1$. 

\begin{itemize}
\item[(a)] Prove that $|a+b\sqrt{-5}|=\sqrt{a^2+5b^2}$. Conclude that if $|z|=1$, then $z=\pm 1$. 
\item[(b)] Assuming (a), prove that every $z\in\Z[\sqrt{-5}]$ can be written as a product of primos. 
\item[(c)] Show that $2,3,1\pm\sqrt{-5}$ are all primo. 

\end{itemize}

\begin{solution}
For part (a), the fact that $|a+b\sqrt{-5}|=\sqrt{a^{2}+5b^{2}}$ juts follows from the definition of the modulus of a complex number. In particular, if $z=a+b\sqrt{-5}$ has modulus 1, then $1=|z|^{2}=a^{2}+5b^{2}$, and the only way this can happen is if $b=0$ and $a=1$ (since a and b are integers).


For part b, we can prove this by strong induction on the modulus of a number z. If $|z|=1$, then z must be $\pm 1$, which is a primo (since if we have xy=z, then 1=|z|=|xy|=|x||y|, so |x|=|y|=1, and hence $x,y=\pm 1$, that is, they are both equal to plus or minus z).


Now suppose we know that claim b holds for all z with $|z|\leq n$ for some integer $n$. Let $z$ now be an element with modulus $n+1$. If it is not primo, then there is a way of writing z as $xy$ where neither x or y are equal to plus or  minus z, so in particular, neither of them are $\pm 1$ either. By part (a), since neither of them are 1, then $|x|,|y|>1$, and $|z|=|xy|=|x||y|$, so $1<|x|=|z|/|y|<|z|=n+1$. Thus, $x$ can be factored into a product of primos by the induction hypothesis, and y can be similarly. This proves the induction step.


For c, note that 2 is primo since otherwise 2 can be written as a product of two numbers with modulus less than 2, but then they must have modulus 1, and so they must be $\pm1$, which is impossible.


3 is primo since if it were the product of two numbers xy with modulus less than 3, then they would have to either be $\pm 1$ or $\pm 2$, which don't multiply up to 3.


$1+\sqrt{-5}$ is primo since, if it were written as a product $xy$ with neither of x or y equal to 1 or $1+\sqrt{-5}$, notice that $|x|\cdot |y|=|xy|=|1+\sqrt{-5}|=\sqrt{1+5}=\sqrt{6}$, so $|x|,|y|\leq \sqrt{6}$, and the only numbers for which this holds are $\pm 1$ and $\pm 2$.
\end{solution}


\end{exercise}


\begin{exercise} Find all integer solutions to the equation $x^3+3y^3+9z^3= 0$.

\begin{solution}
{\bf Claim:} $x^3+3y^3+9z^3= 0$ does not have any solutions in positive integers.

\begin{proof}
Assume that $(x, y, z)$ is a solution in positive integers. Clearly, $x$ is divisible by $3$, so $x= 3x_1$ for some positive integer $x_1$. But then $27x_{1}^{3}+3y^3+9z^3= 0$,hence $9x_1^3+y^3+3z^3= 0$. Now $y= 3y_1$, and we find $3x_1^3+9y^{3}_{1}+z^3= 0$. Finally, $z= 3z_1$ for some positive integer $z_1$, and $x_{1}^{3}+ 3y_{1}^{3}+ 9z_{1}^{3}= 0$.Thus if $(x, y, z)$ is a, integer solution of the equation $x^3+3y^3+9z^3= 0$, then so is $(x/3,y/3,z/3)$. Repeating this argument we find that for every positive solution there is a smaller solution in positive integers: but this is nonsense, thus there is no solution in positive integers.
\end{proof}
\end{solution}


\end{exercise}

%
%\begin{exercise} Suppose $f$ is a degree $n>5$ polynomial so that for some distinct integers $a<b<c<d$ we have $f(a)=f(b)=f(c)=f(d)=0$. Show that there is no integer $k$ so that $f(k)=3$. 
%
%\begin{solution}
%We can factor $f$ as 
%\[
%f(x) = (x-a)(x-b)(x-c)(x-d)g(x)
%\]
%for some polynomial $g$. If $f(x)=3$ for some integer $x$, then
%\[
%3=(x-a)(x-b)(x-c)(x-d)g(x),\]
%so $3$ is a product of at least 4 distinct integers, but this is impossible: the FTA implies that exactly one of these integers must be $\pm 3$, in which case the other integers must be either $\pm 1$, but that means $3$ can only be written as a product of 3 distinct integers.
%\end{solution}
%
%\end{exercise}

%
\begin{exercise} Show that if $n> 4$ is an integer that is {\it not} prime, then $n|(n-1)!$.

\begin{solution}

Since $n$ is not prime, $n=ab$ for some integers $1<a,b<n$. 
\begin{itemize}[(a)]
\item If $a<b$, then
\[
(n-1)!=1\cdot 2 \cdots a\cdots b\cdots (n-1).
\]
so $n=ab|(n-1)!$.
\item if $a=b$, then $n=a^2$ and $a>2$, so $2a\leq a^2-1$, and 
\[
(n-1)!=1\cdot 2 \cdots a\cdots 2a\cdots (a^2-1)
\]
so we see that $a^2|(n-1)!$. 
\end{itemize}
\end{solution}
%
%\end{exercise}


\begin{exercise} Recall that the Fermat numbers $F_{n}=2^{2^{n}}+1$ satisfy the recurrence relation
\[
F_{n} = F_{n-1}\cdots F_{0}+2.
\]
Show that the Fermat numbers are mutually relatively prime, that is, $\hcf(F_{n},F_{m})=1$ whenever $m\neq n$. 








\end{exercise}




{\bf Challenging problem: A special case of Fermat's last theorem}



\begin{exercise}  In this exercise we will show that $x^4 + y^4 = z^4$ (the Fermat quartic) has no {\it nontrivial solutions}, that is, no integer solutions apart from $(x,y,z)=0$. The method we will use is known as ``infinite descent": we suppose there is a solution $(x,y,z)$ with $z>0$, then we assume $(x,y,z)$ has $z$ smallest among all such solutions, then we show that we can find another solution $(u,v,w)$ with $w<z$, contradicting that $z$ was the smallest integer appearing in a solution to $x^4+y^4=z^4$. This is quite difficult and wouldn't be expected of you on a homework or exam, it's just a bit of fun.
\begin{itemize}
\item Explain why it suffices to only find positive solutions?
\item Why does it suffice to show $x^4+y^4=z^2$ has no nontrivial integer solutions?
\item Suppose there is a non-trivial solution positive solution to  $x^4 + y^4 = z^2$. Then there is a solution $(x,y,z)$ with smallest $z>0$ among all solutions. Show that $x,y,z$ have no common factors. 
\item Show that exactly one of $x$ or $y$ is odd. Now assume $x$ is odd and $y$ is even. 
\item Note that $a=x^2$ and $b=y^2$ along with $c=z$ form a Pythagorean triple, $a$ is odd, and $x,y,z$ are coprime.  Hence, by our work on classifying Pythagorean triples, we must have 

\[
(a,b,c) = \left( as^2 t^2, a\frac{s^2-t^2}{2}, a\frac{s^{2}+t^{2}}{2}\right)
\]
for some $s>t$. Show that $4|s-t$. 
\item Using that $(s-t)(s+t)=2b=2y^2$, show that there are integers $u$ and $v$ with $s+t = 2u^2$ and $s-t=4v^2$, and $u$ and $2v$ are relatively prime. solving for $s$ and $t$, verify that $x^2+4v^4 = u^4$.
\item Let $A = x$, $B = 2v^2$ and $C = u^2$. Show that this is another Pythagorean triple of the form described in the parametrization theorem. So they can be expressed as above for $(a,b,c)$, but with parameters $S$ and $T$ instead of $s$ and $t$.
\item Arguing in a similar fashion to (d), show that $S$ and $T$ satisfy $S+T = 2X^2$ and $S-T=2Y^2$ for some positive integers$ X$ and $Y$.
\item Solve for $S$ and $T$ and verify that $u^2 = X^4 + Y^4$. Finally show that $uXY\neq 0$ and that $0 < u < z$, leading to a contradiction.
\end{itemize}  

\end{exercise}












\part{Congruence}


%\chapterimage{} 


\chapter{Modular Arithmetic}


%
%\epigraph{\it And this proposition is generally true for all progressions and for all prime numbers; the proof of which I would send to you, if I were not afraid to be too long.}{Pierre de Fermat, 1640, in a letter to his friend Fr\'{e}nicle de Bessy commenting on his what later was later to be called, ironically, {\it Fermat's Little Theorem}.}





Suppose I asked you to show that $30^{99}+61^{100}$ was divisible by 31. This seems like a daunting task. However, in a couple of pages, you'll learn a trick that will make this and other problems involving divisibility of very large numbers much easier. The method is {\it modular arithmetic}.  

The framework of modular arithmetic as we are familiar with it was first developed by Gauss in his book {\it  Disquisitiones Arithmeticae} in 1801, whereas many fundamental results that are now stated in terms of congruences were in fact proven much earlier; for example, Fermat's Little Theorem below was proven in 1640, though now it is standard to formulate it using congruences. 


\section{Arithmetic $\mod m$}

\begin{definition}
\label{d:mod}
For $m\in \mathbb{N}$ and $a,b\in \mathbb{Z}$, we write  {\color{magenta} $a\equiv b\mod m$} (read {\it $a$ is congruent to $b$ modulo $m$}) if one of the following holds (all are equivalent):
 
\begin{itemize}
\item $m|(a-b)$ 
 
\item $a$ and $b$ have same remainder when divided by $m$ 
\item $b=qm+a$ for some $q\in \mathbb{Z}$ 
\end{itemize}

\end{definition}

\begin{example}
$32\mod 10 = 2$ since $12=3\cdot 10+2$. $82 \mod 3=1$ since $3| 81=82-1$. 
\end{example}
 
\begin{theorem}
The relation $\equiv $ is an equivalence relation on the integers.
\end{theorem}

\begin{proof}
We need to check three things:
\begin{itemize}
\item (Reflexivity) Since $m|(a-a)$, we have that $a\equiv a$.
\item (Symmetry) If $a\equiv b\mod m$, then $m|(a-b)$, so $m|(b-a)$ as well, hence $b\equiv a \mod m$. 
\item (Transitivity) Suppose $a\equiv b \mod m$ and $b\equiv c\mod m$. Then $m|a-b$, so $a-b=km$ for some integer $k$. Moreover, $m|b-c$, so $b-c=jm$ for some integer $j$. Thus,
\[
a-c = a-b+b-c = km+jm = (k+j)m,\]
hence $j|(a-c)$, thus $a\equiv c \mod m$. 
\end{itemize}
\end{proof}

\begin{theorem}[Propositions 13.3 and 13.4]
\label{t:mod-artithmetic}
Suppose $a \equiv x \mod m$ and $b\equiv y \mod m$. Then
\begin{itemize}
\item $a+b\equiv x+y \mod m$
\item $ab \equiv xy \mod m$
\item  $a^{k}\equiv x^{k} \mod m$
\end{itemize}
\end{theorem}

\begin{proof}
Assume $a \equiv x \mod m$ and $b\equiv y \mod m$.
\begin{itemize}
\item Since $a\equiv x\mod m$ implies $m|a-x$, and $b\equiv y \mod m$ implies $m|b-y$, so the Easy Lemma implies
\[
m|(a-x+(b-y))=(a+b-(x-y)),
\]
and so $a+b\equiv x-y\mod m$ by Definition \ref{d:mod} (1).   
\item $a\equiv x \mod m$ implies $a=x+pm$ for some $p\in\mathbb{Z}$ and $b\equiv y \mod m$ implies $b=y+qm$ for some $q\in\mathbb{Z}$. Thus,
\[
ab = (x+pm)(y+qm) = xy+pmy+xqm+ pqm^2 = xy + m(py+xq+pqm),\]
and so $ab\equiv xy$ by Definition \ref{d:mod}(3).
\item This can be proven by induction. For the base case of $k=1$, this is immediate. Now suppose $a^{k}\equiv x^{k}\mod m$ for some $k\geq 1$. Then by part (b) of this theorem with $b=a^{k}$ and $x=y^{k}$ and our induction hypothesis
\[
a^{k+1}=a\cdot a^{k}\equiv a x^{k} \mod m\]
and then applying part (b) again but with $b=y=x^k$,
\[
a x^{k}\equiv x\cdot x^k=x^{k+1}.\]
This proves the induction and hence the theorem.

\end{itemize}
\end{proof}



Now let's revisit that earlier example: why is $30^{99}+61^{100}$ divisible by 31? Note that $30=31-1\equiv -1\mod 31$ and $61= 2\cdot 31-1\equiv -1\mod 31$, thus 
\[
30^{99}+61^{100}\equiv (-1)^{99}+(-1)^{100}=-1+1=0 \mod 31
\]
thus, $31|30^{99}+61^{100}$. Done!\\

{\bf Moral:} When doing arithmetic mod $m$, we can use this theorem to substitute big numbers with smaller ones to make things easier. 






\def\hcf{{\rm hcf}}


Theorem \ref{t:mod-artithmetic} says we can add and multiply modulo $m$. Can we also divide? That is, if $a\equiv x\mod m$, do we also have $\frac{a}{h}\equiv \frac{b}{h}\mod m$? This is not always the case. The following theorem says when we can divide in this way. 

\begin{proposition}
\label{p:xa=ya-x=a}
Let $a$ and $m$ be coprime. If $x,y\in \mathbb{Z}$ and $xa=ya\mod m$, then $x\equiv y\mod m$. In particular, if $p$ is prime,  $p\not|a$, and $xa\equiv ya\mod p$, then $x\equiv y\mod p$. 
\end{proposition}

\begin{proof}
Suppose $a$ and $m$ are coprime and $xa\equiv ya\mod m$. Then $m|xa-ya=(x-y)a$. Since $m$ and $a$ are coprime. Corollary \ref{c:c|abc|b} implies $m|x-y$, and so $x\equiv y\mod m$ by definition. The second part of the proposition follows from the first.
\end{proof}




\begin{example}
Find an integer $x\in \{0,1,..,6\}$ so that $4^{6}\equiv x\mod 7$. \\

Let's look at some powers of $4$ and see what they are $\mod 7$. 

\begin{align*}
4^2 & = 16 =14+2\equiv 2 \mod 7 \\
4^4 & = (4^2)^2 \equiv 2^2 =4 \mod 7.
\end{align*}


Thus,
\[
4^6=4^2\cdot 4^4\equiv 2\cdot 4=8\equiv 1 \mod 7.
\]
This is certainly much quicker than computing $4^6$ by hand and then doing long division!
\end{example}



\section{Solving linear equations modulo $m$}


\subsection{When can we solve $ax\equiv b\mod m$?}

Before trying to solve such an equation, it's good to have a test to see if there is actually a solution.

\begin{theorem}[(Liebeck Proposition 13.6)] The equation $ax\equiv b\mod m$ has a solution if and only if $\hcf(a,m)|b$. 
\end{theorem}


\begin{proof}

Firstly, if $x$ is a solution to $ax\equiv b\mod m$, then $ax=b+qm$ for some integer $q$, and so $b=ax-qm$, so by the Easy Lemma, $\hcf(a,m)|b$. \\

Conversely, suppose $h=hcf(a,m)|b$. Then $x$ is a solution if and only if
\begin{align*}
ax\equiv b \mod m & \;\; \Longleftrightarrow \;\;  \exists q \mbox{ such that } ax=b+mq\\
& \;\; \Longleftrightarrow  \;\;  \exists q \mbox{ such that } b=ax-mq.
\end{align*}

So we just need to find integers $x$ and $q$ for which this equation holds $ b=ax-mq$, and then $x$ will be our solution. 
\begin{itemize}
\item Let's first consider the easier case when {\color{red} $\hcf(a,m)=b=1$}. Then we just need to solve $1=ax-mq$, which we can do using the Euclidean Algorithm (and the first problem from last week's homework) since $hcf(a,m)=1$. 
\item Next let's consider when {\color{red} $\hcf(a,m)=1\neq b$}. If $x$ solves $1=ax-mq$ (which we know exists by the previous case), then multiplying both sides by $b$ modulo $m$ gives $b=a(bx)-mq$, and $bx$ is our desired solution. 
\item Finally, we just consider the general case when $\hcf(a,m)|b$.  Now  $\hcf(a,m)$ could not be $1$, so $a$ and $m$ are not necessarily coprime, but $\hcf(\frac{a}{h},\frac{m}{h})=1$. Hence, the previous case there is $x$ and $q$ s.t.  $\frac{b}{h}=\frac{a}{h}x- \frac{m}{h}q$, and hence $b=ax-mq$, so $x$ is our solution.

%% {\color{magenta} $\frac{a}{h}x\equiv \frac{b}{h} \mod \frac{m}{h}$}.
%
%\begin{align*}
%\frac{a}{h}x\equiv \frac{b}{h} \mod \frac{m}{h}
%& \;\; \Longleftrightarrow \;\; \exists q \mbox{ s.t. }\frac{a}{h}x= \frac{b}{h} + q\frac{m}{h} \\
%& \;\; \Longleftrightarrow \;\; \exists q \mbox{ s.t. }ax= b + qb\\
%& \;\; \Longleftrightarrow \;\;  ax\equiv b\mod m.
%\end{align*}
%



\end{itemize}



\end{proof}


This proof gives an outline for how to solve linear equations. Notice that finding $x$ so that we found a solution to $ax\equiv b\mod m$ by finding $x$ and $q$ so that $\frac{b}{h}=\frac{a}{h}x-\frac{m}{h}q$, that is, by solving $\frac{a}{h}x\equiv \frac{b}{h} \mod \frac{m}{h}$. So we can solve this latter (simpler) equation first.

\begin{example}
Solve $7x\equiv 8\mod 12$. \\

Since $\hcf(7,12)=1$, we can first solve $7x\equiv 1 \mod 12$ and then multiply our solution by $8$. To solve this simpler equation, we use the Euclidean Algorithm:
\begin{multicols}{2}
We have:
\begin{align*}
12 & =7+5 \\
7& =5+2 \\
5& =2\cdot 2 + 1 
\end{align*}
Now we reverse our steps
\begin{align*}
1&=5-2\cdot 2 \\
& = 5-2\cdot (7-5) = 3\cdot 5 -2\cdot 7\\
& = 3\cdot (12-7)-2\cdot 7 = 3\cdot 12-5\cdot 7.
\end{align*}



\end{multicols}

We can verify this is correct because $3\cdot 12 - 5 \cdot 7 = 36-35=1$. Thus,
\[
7\cdot (-5)=1-3\cdot 12 \;\; \Longrightarrow 7(-5) \equiv 1\mod 12.
\]
Recall that when solving $\mod 12$, we want a solution in $\{0,1,...,11\}$, so we replace $-5$ with $7$, and so $7\cdot 7 \equiv 1\mod 12$. Multiplying both sides by $8$, we get 
\[
7\cdot (7\cdot 8) = 8\mod 12\]
And so $x=7\cdot 8=56=48+8\equiv 8\mod 12$ is a solution to our original equation. 

\end{example}

%
%



\begin{example}
Solve $18x\equiv 10\mod 14$. \\

We see that $\hcf(18,14)=2|10$, so there is at least one solution. As mentioned before, we can solve the above equation by first solving the same equation but with $18,10,$ and $14$ divided by the hcf, that is, the equation 
\[
9x\equiv 5\mod 7.\] 
Hence, we need integers $x$ and $q$ so that 
\[
9x=5+7q, \;\; \mbox{ or equivalently, } 5=9x-7q.\] 
To find integers that satisfy this, let's first solve $1=9x-7q$ using the Euclidean Algorithm, we get that $1=-3\cdot 9+4\cdot 7 $. Multiplying both sides by 5 we get $10=-15\cdot 9 +20\cdot 7$, thus $-15\cdot 9\equiv 5\mod 7$, and hence $-15$ also solves $18x\equiv 10\mod 14$. We need a solution in $\{0,1,...,13\}$, but $-15\equiv 13\mod 14$, so $13$ is our desired solution. 

\end{example}

%
%
%\subsection{ How many distinct solutions are there to $ax\equiv b\mod m$?}
%
% 
%\begin{theorem}
%If $\hcf(a,m)|b$, then $ax\equiv b\mod m$ has $\hcf(a,m)$ many distinct solutions in $\{0,1,...,m-1\}$. If $x$ is one solution, the others are of the form
%\[
%x\equiv y+j\frac{m}{h} \mod m\;\; \mbox{ for $j=1,2,..., \hcf(a,b)$}.
%\]
%\end{theorem}
%
%
%
%
%
%
%\begin{proof}
%Suppose $x,y$ are two solutions, so 
%\[
%ax\equiv b\mod m \;\; \mbox{and} \;\; ay\equiv b\mod m.
%\]
%\vspace{-10pt}
% 
%Then $ax\equiv ay$,  hence $a(x-y)\equiv 0 \mod m$,  so $m|a(x-y)$.  Thus,
%\[
%a(x-y)=mq \;\;\; \mbox{ for some integer } q
%\] 
%If $h=\hcf(a,m)$, then 
%\[
%\frac{a}{h}(x-y)=\frac{m}{h}q.
%\]
% 
%Thus, since $\frac{a}{h}$ and $\frac{m}{h}$ are coprime,  we must have $\frac{m}{h}|(x-y)$,   that is,  
%\[
%{\color{magenta} x=y+j\frac{m}{h} \;\; \mbox{ for some $j$}.}
%\] 
%If $x\in \{0,...,m-1\}$ is a solution, all solutions must be of this form,   and there are exactly $\frac{m}{m/h}=h$ such integers in $\{0,...,m-1\}$. 
%
%
%
%\end{proof}
%
%
%
%
%
%\begin{example}
%Find all solutions in $\{0,1,...,13\}$ to $6x=4\mod 14$. 
%\end{example}
% 
%First we need to find one solution. Note that $\hcf(6,14)=2$, so we first solve $3x\equiv 2 \mod 7$. Note that $7=1+3\cdot 2$, and so $3\cdot 2 \equiv -1 \mod 7$, thus $3\cdot (-2)\equiv 1\mod 7$, and so $3\cdot ((-2)\cdot 2)\equiv 2\mod 7$. Thus, $x=-2\cdot 2= -4\equiv 3 \mod 7$ is a solution. Now $x=3$ will also be a solution to $6x=4\mod 14$. 
%
%By previous proposition, there are $\hcf(6,14)=2$ solutions, and since $x=3$ is one solution, the other solution must be of the form
% 
%\[
%3+j\frac{14}{\hcf(6,14)} = 3+j\frac{14}{2}   =3+j7.
%\]
% 
%The only $j$ so that this gives a number in $\{0,1,...,13\}$ is $j=1$,  so $3+7=10$ is the other solution in $\{0,1,...,13\}$. 
%
%
%
%
%
%So let's review the method for solving $ax\equiv b \mod m$:
% 
%\begin{itemize}
%\item Use Euclidean Algorithm to find $s,t\in \mathbb{Z}$ so that $as+tm=\hcf(a,m)$.  
%\item Thus, $as\equiv \hcf(a,m)\mod m$.  
%\item Thus, $as \cdot \frac{b}{\hcf(a,m)}\equiv \hcf(a,m)\cdot \frac{b}{\hcf(a,m)}   =b \mod m$.  
%\item Hence, $y=s \cdot \frac{b}{\hcf(a,m)}$ is a solution. 
%\item Now find $x\in \{0,1,...,m-1\}$ so that $x\equiv y\mod m$.
%\item The other solutions in $\{0,1,...,m-1\}$ will be of the form $x+j\frac{m}{\hcf(a,m)}$, and there are $\hcf(a,m)$ many of them.
%\end{itemize}
%
%\begin{example}
%Solve $17x = 4\mod 11$.  
%\end{example}
%
%\begin{itemize}
%\item $\hcf(17,11)=1$, so there is one solution.  
%\item The Euclidean algorithm gives $2\cdot 17=1+3\cdot 11$.  
%\item Thus, $2\cdot 17 \equiv 1 \mod 11$. 
%\item Hence, $4\cdot 2 \cdot 17\equiv 4\cdot 1 \mod 11$.  
%\item Thus, $x=8$ solves $17x \equiv 4\mod 11$.  
%\end{itemize}
%



\section{$\mathbb{Z}_{m}$}
 
We define 
\[
\mathbb{Z}_{m}=\{\bar{0},\bar{1},...,\overline{m-1}\}\]  
and we define addition and multiplication as: 
\[
\bar{x}\cdot \bar{y} = \bar{z} \mbox{ if $z\in \{0,1,...,m-1\}$ is such that } xy\equiv z\mod m\] 
and
\[
\bar{x}+\bar{y} = \bar {z} \mbox{ if $z\in \{0,1,...,m-1\}$ is such that } x+y\equiv z\mod m.\] 

{\bf Example:}   In $\mathbb{Z}_{5}=\{\bar{0},\bar{1},\bar{2},\bar{3},\bar{4}\}$, 
\[
\bar{2}+\bar{3} = \bar{0} \mbox{ since } 2+3=5\equiv 0 \mod 5.
\] 
\[
\bar{2}\cdot \bar{3} = \bar{1} \mbox{ since } 2\cdot 3=6\equiv 1 \mod 5.
\]
 
You will see $\mathbb{Z}_{m}$ often in future classes, it's a small set with algebraic structure (i.e. addition and multiplication).



{\bf Example:}  In $\mathbb{Z}_{5}$, does $\bar{2}$ have a cubed root?  (That is, does $\bar{2}=\bar{x}^{3}$ for some $\bar{x}\in \mathbb{Z}_{5}$?)\\  
\vspace{10pt}
Let's just test the values: 

\begin{itemize}
\item $\bar{0}^{3}=\bar{0}$. 
\item $\bar{1}^{3}=\bar{1}$. 
\item $\bar{2}^{3}=\bar{3}$ since $2^{3} = 8\equiv 3\mod 5$.  
\item $\bar{3}^{3}=\bar{2}$ since $3^{3} = 27\equiv 2\mod 5$.  
\item $\bar{4}^{3}=\bar{4}$ since $4^{3}\equiv (-1)^{3}=-1\equiv 4 \mod 5$.  
\end{itemize}
Thus, the cubed root of $\bar{2}$ is $\bar{3}$. \\
\vspace{10pt}
 
Not all numbers in every $\mathbb{Z}_{m}$ have roots! Example: Is there $\bar{x}\in \mathbb{Z}_{4}$ so that $\bar{x}^{3}=2$?




{Inverses in $\mathbb{Z}_{m}$}
 
An element $\bar{x}\in \mathbb{Z}_{m}$ is {\it invertible} if there is $\bar{y}\in \mathbb{Z}_{m}$ so that $\bar{x}\cdot \bar{y}=1$. \\

\vspace{10pt}
{\bf Example:}  How many invertible elements are there in $\mathbb{Z}_{81}$? 
\vspace{10pt}

Let $\bar{x}\in \mathbb{Z}_{81}$. By Prop 13.6, 
\[
\mbox{ there is $\bar{y}$ so that $\bar{x}\cdot \bar{y} =\bar{1}$}
\] 
 if and only if  
 \[
 \mbox{ there is $y\in \{0,1,...,80\}$ so that $xy\equiv 1\mod 81$}
 \] 
 if and only if $\hcf(x,81)|1      \; \Leftrightarrow   \; \hcf(x,81)=1   \; \Leftrightarrow \;  3\not|x$.  
 
 
There are $81/3=27$ numbers in $\{0,...,80\}$ divisible by 3,   so there are $81-27=56$ that are not.  Thus, there are $56$ invertible elements. 
 
 





\section{Fermat's Little Theorem}
 
 Remember that the great thing about congruence is how it effectively shrinks large numbers in any equation you're solving. The following is a little but powerful theorem due to Fermat that allows you to eliminate large powers if you are working modulo some prime. 
 
\begin{theorem} (Fermat's Little Theorem, FLT)
If $p$ is prime and $p\not|a\in \mathbb{Z}$, then
\[
a^{p-1}\equiv 1\mod p. 
\]
\end{theorem}

The proof below may seem intimidating since some ideas seem to just come from nowhere, but don't panic. Now that you've seen the proof, you can use the ideas in future problems.

\begin{proof}
Let $p$ be prime and $a$ coprime to $p$. We consider the numbers $a,2a,3a,...,(p-1)a$. We first claim that each of these numbers is distinct modulo $p$, that is, for each $j,k\in \{1,...,p-1\}$, $ja\equiv ka\mod p$ if and only if $j=k$. If $ja\equiv ka$, then $p|ja-ka=(j-k)a$. Since $p\not|a$, $p|j-k$, but since $j,k\in \{1,...,p-1\}$, this is only possible if $j=k$. This proves the claim. 

Thus, to each $j$, there is a unique $k_j\in \{1,2,...,p-1\}$ so that $ja\equiv k_j \mod p$. This means that 
\[
a\cdot 2a \cdot 3a\cdots (p-1)a \equiv 
k_1\cdots k_{p-1} = 1\cdot 2 \cdots \cdot (p-1),\]
that is, the numbers $k_1,...,k_{p-1}$ are just the integers $1,2,...,p-1$ in a different order. The above equation implies
\[
(p-1)!a^{p-1}\equiv (p-1)!\mod p.
\]
Since $p$ and $(p-1)!$ are coprime, Proposition \ref{p:xa=ya-x=a} implies $a^{p-1}\equiv 1 \mod p$. 
\end{proof}

\begin{example}
Find $5^{100}\mod 7$. \\

We could first compute all even powers of $5$ modulo $7$ and then find some even powers of $5$ that multiply up to $7$, but the FLT will save us some work. Since $7\not|5$, the FLT says $5^{6}\equiv 1\mod 7$. Thus, we can shave off any power of $5$ from $5^{100}$ that is a multiple of $6$. Thus,
\[
5^{100}\equiv 5^{96}5^{4}=(5^{6})^{16}5^{4}\equiv 1^{16}5^{4}= 5^{4}.
\]
Then the rest is as before: 
\begin{align*}
5^{2} & =25\equiv 4\mod 7\\
5^{4} & =(5^{2})^{2}\equiv 4^{2}=16\equiv 2\mod 7.
\end{align*}
\end{example}

{\bf Application:}   How to solve

\begin{equation}
x^{n}\equiv b\mod p.
\end{equation} 
 
\begin{itemize}
\item {\color{magenta} Assume $n$ and $p-1$ are coprime and $p\not |b$.} 
\item Use Euclidean Alg. to find $s,t>0$ so that $sn-t(p-1)=1$.  
\item Then $b\equiv x^{n}\mod p$ implies 
\[
b^{s}\equiv (x^{n})^{s} 
 = x^{ns} 
  =x^{t(p-1)+1}  
  = x\cdot (x^{p-1})^{t} 
\stackrel{FLT}{\equiv} x\cdot (1)^{t}  
= x.\] 
Hence, if $x\equiv b^{s}$, then $x$ is a solution. It is unique mod $m$, i.e.  

\begin{itemize}
\item If $y$ is another solution, then $x\equiv y\mod m$, or equivalently, 
\item there is exactly one $x\in \{0,1,...,p-1\}$ that solves  (\theequation). 
\end{itemize}
\end{itemize}






{\bf Example:}   Find $x\in\{0,1,...16\}$ so that $ x^{7}\equiv 4\mod 17$.  

\vspace{10pt}

\begin{itemize}
\item $\hcf(7,17-1)=\hcf(7,16)=1$.  
\item Euclidean Algorithm gives $6\cdot 7=1+3\cdot 16$.  
\item Thus, if $x$ solves the above equation, then $4\equiv x^{7}$,   and so
\[
4^{6}\equiv (x^{7})^{6}  
= x^{7\cdot 6}  
= x^{1+3\cdot 16} 
 = x\cdot (x^{16})^{3} 
  \equiv x\cdot (1)^{3} 
   = x.
\] 
\item Thus, $x\equiv 4^{6}$,   so we need to find a representative in $\{0,1,...,16\}$.  
\item $4^{2}=16\equiv -1\mod 17$,  so 
\[
4^{6}=(4^{2})^{3} 
\equiv (-1)^{3} 
=-1\equiv 16\mod 17. 
\] 
Thus, $x=16$ is the unique solution in $\{0,1,...,16\}$. 
\end{itemize}







\begin{example}
Find a solution to $x^{22}\equiv 3 \mod 11$.  
\end{example}

\begin{itemize}
\item Note that $\hcf(22,11-1)=\hcf(22,10)=2$, so we can't use previous method.  
\item However, if $x$ is a solution, then $11\not|x$   (otherwise $x^{22}\equiv 0$.  
\item Thus, FLT implies $x^{10}\equiv 1 \mod 11$.  
\item Hence, $x^{22}=x^{2}(x^{10})^{2}  \equiv x^{2}$.  
\item Now we just need to solve $x^{2}\equiv 3 \mod 11$.  
\item Again, $\hcf(2,11-1)=\hcf(2,10)=2$, so we can't use the earlier method.  
\item But now the power $2$ is small enough we can just try values $x\in \{0,1,...,10\}$ and see what works: 

\begin{itemize}
\item $1^{2}=1$ 
\item $2^{2}=4$ 
\item $3^{2}=9$ 
\item $4^{2}=16\equiv 5 \mod 11$.  
\item $5^{2}=25\equiv 3\mod 11$.  
\end{itemize}
Thus, $x=5$ is a solution to $x^{22}\equiv 3 \mod 11$:  
\[
5^{22}=5^{2}(5^{10})^{2}\equiv 3\cdot (1)^{2}=3.
\]

\end{itemize}

\begin{example}
 Find $x\in\{0,1,...,10\}$ that solves $x^{3}\equiv 9 \mod 11$.  
\end{example}

\begin{itemize}
\item Note $\hcf(3,11-1)=\hcf(3,10)=1$.  
\item Note also that $7\cdot 3 = 21=1+2\cdot 10$.   
\item Thus, if $x$ is a solution, then $9\equiv x^{3}$ implies  
\[
9^{7}\equiv (x^{3})^{7}  
=x^{3\cdot 7}  
 =x^{1+2\cdot 10}   
  =x\cdot (x^{10})^{2}  
  \equiv x\cdot(1)^{2}  
  =x.\]
  Thus, if $x$ is the solution, then $x\equiv 9^{7}$.   
  \item Now we must find $x\in \{0,1,...,10\}$ so that $x\equiv 9^{7}$.   
  \begin{itemize}
  \item $9^{2}=81  \equiv 4$
  \item $9^{4}=(9^{2})^{2}  \equiv 4^{2}  =16\equiv 5\mod 11$.   
  \item $9^{7}=9^{4}\cdot 9^{2}\cdot 9 \equiv 5\cdot 4\cdot 9 =180 =176+4\equiv 4\mod 11$. 
  \end{itemize}
    
\item   Thus, $x=4$ is the solution.  We can also check $4^{3}=64=55+9\equiv 9\mod 11$. 
\end{itemize}






\subsection{Diophantine equations}

As congruences relate to divisibility, we can also use them to solve diophantine equations. 

\begin{example}
What are the integer solutions to $9x^{2}+9x+2=y^{4}$?
\end{example}

 
 We will present two methods for comparison, first using the usual divisbility methods from last week, and then a second method using modular arithmetic.\\

{\bf Method 1:} 
\begin{itemize}
\item  If $x,y$ are integer solutions, then
\[
y^{4}=9x^{2}+9x+2 =(3x+2)(3x+1).
\] 
\vspace{-10pt}
\item Since $y^{4}\geq 0$, we know $3x+1,3x+2> 0$ or $<0$.  First assume $>0$.   They are coprime since $\hcf(3x+1,3x+2)| (3x+2)-(3x+1)=1$.  
\item  Thus, $3x+1=a^{4}$ and $3x+2=b^{4}$ for some integers $a$ and $b$.   (We can assume they are nonnegative.) 
\item But then $b^{4}-a^{4} =3x+2-(3x+1) =1$.   This is only possible if $(a,b)=(0,1)$  (exercise :) 
\item We can't have $a=0$ since then $3x+1=0$,  which is impossible if $x$ is an integer.  
\item Same happens if $3x+1,3x+2<0$.  Thus, there are no solutions.
\end{itemize}





One can use congruences to help solve Diophantine equations.  
\vspace{10pt}


 

{\bf Method 2:} 
\begin{itemize}
\item If $x,y$ are integer solutions to $9x^{2}+9x+2=y^{4}$, then 
\[
y^{4}=9x^{2}+9x+2  
\equiv 2\mod 3.
\] 
\item If $z\in \{0,1,2\}$ is such that $z\equiv y\mod 3$, then   $z^{4}\equiv y^{4}\equiv 2\mod 3$.  
\item So let's test some $z$'s! 
\begin{itemize}
\item $0^{4}=0$ 
\item $1^{4}=1$ 
\item $2^{4}=16\equiv 1 \mod 3$.  
\end{itemize}
So there are no $z\in \{0,1,2\}$ so that $z^{4}\equiv 2 \mod 3$.  
\item Thus, there are no integer solutions to $9x^{2}+9x+2=y^{4}$.
\end{itemize}





 

\section{Exercises}




\begin{exercise} Determine whether the following equations have integer solutions:
\begin{itemize}
\item $x^2+y^2=9z+3$.

\begin{solution}
If $(x,y,z)$ was a solution, then $x^2+y^2\equiv 3\mod 9$, but the squares modulo $9$ are $0,1,4,0,7,0,3,1$, and adding any pair of these gives $0,1,2,4,5,7,8$, and none of these are $3$, thus there cannot be any solutions to the original equation.
\end{solution}


\item $3x^2-y^2=-2$.

\begin{solution}
If $(x,y)$ was a solution, then $-y^2\equiv -2\mod 3$,a dn so $y^2\equiv 2\mod 3$, but if we look at the squares of integers mod 3, we see that they are just $0$ and $1$, so there is no integer $y$ for which $y^2\equiv 2 \mod 3$, thus there are no solutions to the original equation.
\end{solution}



\item $3x^2+2=y^2+6z^3$.

\begin{solution}
If $(x,y,z)$ is a  solution, we see then that $y^2\equiv 2\mod 3$, and we have already seen that this has no solutions. 

\end{solution}


\end{itemize}

\end{exercise}



\begin{exercise} Show that If $n\in\mathbb{N}$, then $n^5$ and $n$ have the same last digit (that is, the digit in the $1$'s place). 

\begin{solution}
FLT says that $n^{4}\equiv 1 \mod 5$, hence $n^{5}\equiv n \mod 5$, and so $n^{5}-n\equiv 0\mod 5$. Also, $n^5-n=n^4(n-1)$, and since either $n$ or $n-1$ is even, we have $n^5-n\equiv 0\mod 2$. Thus, we have $2|n^5-n$ and $5| n^5-n$, hence $10|n^5-n$, so the last digit of $n^5$ and $n$ must be the same.
\end{solution}

\end{exercise}


\begin{exercise} Find the last digit of $7^{7^{7^{7}}}$.

\begin{solution}
$7^2=49\equiv -1\mod 10$, $7^3\equiv -7\equiv 3\mod 10$, $7^4\equiv 1 \mod 10$. Thus, as we take powers of $7$, the 1's digit cycles through 7,9,3, and 1. So we should look at what $7^{7^7}\mod 4$ is. 

Note that $7\equiv 3\mod 4$, and $7^2\equiv 1\mod 4$, thus odd powers of $7$ are equivalent to $3\mod 4$. Since $7^7$ is odd, $7^{7^7}\equiv 3\mod 4$, and so $7^{7^{7^{7}}}\equiv 3\mod 10$.
\end{solution}



\end{exercise}


\begin{exercise} Suppose $n\in\mathbb{N}$ satisfies $n\equiv 3\mod 4$. Show that $n$ has a prime factor $p$ with $p\equiv 3\mod 4$. 

\begin{solution}

Note that $n$ is not even, so $n$ has prime factorizaiton $p_{1}^{r_{1}}\cdots p_{k}^{r_{k}}$ for some primes $2<p_{1}<\cdots < p_{k}$ and integers $r_{i}>0$. Suppose $p_{i}\equiv 1 \mod 4$ for all $i$. Then $p_{i}^{r_{i}}\equiv 1 \mod 4$ as well, and so
\[
n=p_{1}^{r_{1}}\cdots p_{k}^{r_{k}}\equiv 1 \mod 4,
\]
which is a contradiction. Thus, there is $i$ so that $p_{i}\equiv 3\mod 4$.


\end{solution}

\end{exercise}

\begin{exercise}
Find $k=1,2,3,4,5,6$ so that there are no integer solutions to $m^3-n^3\equiv k\mod 7$.

\begin{solution}
Let us look at what cubes are $\mod 7$:
\begin{align*}
1^3 & =1\equiv 1 \mod 7 \\
2^3 & =8 \equiv 1 \mod 7
3^3 & =27 \equiv -1 \mod 7 \\
4^3 & =64 \equiv 1 \mod 7\\
5^3 & =56\equiv 1 \mod 7\\
6^3 & = 216 \equiv -1 \mod 7
\end{align*}
Thus, the only values $m^3-n^3$ can be $\mod 7$ are $0,\pm 1,$ and $\pm 2$, or alternatively, $0,1,2,5,6$. In particular, $m^3-n^3=3$ and $m^3-n^3$ have no solutions.

\end{solution}

\end{exercise}



\begin{exercise} Show that for all primes $p>3$, $p\equiv \pm 1\mod 6$. Use this to show $24|p^2-1$ for all primes $p>3$.

\begin{solution}
Note that $p$ is odd so $p\not\equiv 2,4\mod 6$. Also, $p>3$, hence $p\neq 3\mod 6$. Thus, the only remaining possibilities are $\pm 1 \mod  6$. Hence, $p=6k\pm 1$ for some integer $k$, so that
\[
p^2-1 = (6k\pm 1)^2-1=36k^2\pm 12k=12(3k^2\pm k).
\]
Note that $3k^2\pm k$ is always even (just consider the differnet cases when $k$ is odd or even), so in fact $p^2-1$ is divisible by 24. 



\end{solution}

\end{exercise}


\begin{exercise} ({\bf Challenge!}) For which primes $p$ does $(p-1)!+1=p^{k}$ for some integer $k$?

\begin{solution}
Let's rearrange this equality: this would imply
\[
(p-1)! = p^{k}-1=(p-1)\sum_{j=0}^{k-1} p^{j}
\]
and so
\[
(p-2)! = \sum_{j=0}^{k-1} p^{j}.
\]
Notice that $p\equiv 1\mod (p-1)$, hence 
\[
(p-2)!\equiv \sum_{j=0}^{k-1} 1=k\mod (p-1).
\]
Also notice that for $p>5$, $p-1|(p-2)!$, and so $(p-2)!\equiv 0\mod (p-1)$, hence the above equation implies $k\equiv 0 \mod (p-1)$, thus $k=\ell(p-1)$ for some integer $\ell$. But then $p^{k}\geq p^{p-1}>(p-1)!$, a contradiction. Hence, $p$ cannot be greater than $5$, so $p=2,3$ or $5$.

\begin{itemize}
\item If $p=2$, then $(p-1)!+1=2=2^{1}$, so $(p,k)=(2,1)$ is a solution.
\item If $p=3$, then $(p-1)!+1=2!+1=3=3^{1}$, so $(3,1)$ is a solution.
\item If $p=5$, then $(p-1)!+1=4!+1=25=5^2$, so $(5,2)$ is a solution. 
\end{itemize}

Thus, these are the only solutions. 


\end{solution}


\end{exercise}


\begin{exercise} Show that if $p_{1}<\cdots < p_{31}$ are prime and $30|p_{1}^{4}+\cdots + p_{31}^{4}$, then $p_{1}=2$, $p_{2}=3$ and $p_{3}=5$. {\it Hint: Assume the contrary and use FLT.}

\begin{solution}

\begin{itemize}
\item  If $p_1\neq 2$ then $p_1>2$ and so $p_1,p_2,\cdots p_{31}$ are all odd, hence $p_1^4+p_2^4+\cdots+p_{31}^4$ is odd, so in particular $30\not| p_{1}^{4}+\cdots + p_{n}^{4}$, and we get a contradiction. Thus, $p_{1}=2$.

\item If $p_2\neq 3$ $p_{1}=2\equiv -1\mod 3$, $p_2\equiv \pm 1(\mod 3)$, and for $i>2$, $p_{i}^{2}\equiv 1\mod 3$ by FLT, so $p_i^{4}\equiv 1(\mod 3)$ for $i=1,2,\cdots,31$. Thus, 
\[
p_1^4+p_2^4+\cdots+p_{31}^4\equiv 31\equiv 1 \mod 3.
\]
Thus, $p_1^4+p_2^4+\cdots+p_{31}^4$ is not a multiple of $30$. So $p_3=3$. 
\item A similar argument shows that $p_3$ should be $5$: $p_3\neq 5 $ implies $ p_i^4\equiv 1 (\mod 5)$ for $i=1,2,\cdots,31$ by FLT and $p_1^4+p_2^4+\cdots+p_{31}^4\equiv 1 (\mod 5)$.
\end{itemize}
\end{solution}


\end{exercise}


\begin{exercise} (Regional Mathematical Olympiad 1998, India) Show that if $5<p_{1}<\cdots < p_{n}$ and $6|p_{1}^{2}+\cdots + p_{n}^{n}$, then $6|n$. 

\begin{solution}
From an earlier exercise, we know that $p_{i}\equiv \pm 1 \mod 6$, hence $p_{i}^{2} = 1 \mod 6$, thus 
\[
0=p_{1}^{2}+\cdots + p_{n}^{2} \equiv n\mod 6,
\]
where in the first equation we used our problem assumption, thus $n\equiv 0\mod 6$, hence $6|n$. 
\end{solution}


\end{exercise}


%\begin{exercise} How  many  prime  numbers $p$ are  there  such that $29^p+ 1$ is a multiple of $p$?
%
%\begin{solution}
%If $p|29^{p}+1$, then $0\equiv 29^{p}+1 \mod p$, and by FLT, $29^{p}+1\equiv 29+1=30\mod p$, so $p|30$, hence $p=2,3,$ or $5$. 
%\end{solution}
%
%\end{exercise}


\begin{exercise} Given that $p$ and $8p^2+1$ are prime, find $p$. 

\begin{solution}
By trying out some primes, we can see that it seems like $3$ is the only option, so let's try to prove $p=3$. Suppose $p\neq 3$. Note that any prime $p\neq 3$ satisfies $p\equiv \pm 1 \mod 3$, so $p^2 \equiv 1 \mod 3$. Thus, since $8\equiv -1\mod 3$, 
\[
8p^2+1\equiv (-1)1+1=0\mod 3,\]
thus $8p^2+1$ is a prime divisible by 3, which is only possible if $8p^2+1=3$, which is impossible.
\end{solution}


\end{exercise}


\begin{exercise} Let $a,x,y\in \mathbb{N}$. Show that if $a>1$ and $a^x+1|a^y+1$, then $x|y$. 

{\it Hint: Recall the geometric series formula.}

\begin{solution}
Recall from the geometric series formula that
\[
\frac{a^y+1}{a+1}   = \sum_{k=0}^{y-1}(-a)^{k}
\]
Since $a=-1 \mod (a+1)$,
\[
\frac{a^y+1}{a+1} = \sum_{k=0}^{y-1}(-a)^{k}\equiv \sum_{k=0}^{y-1}1^k=y\mod(a+1)
\]
The same holds with $x$ in place of $y$. If $a^x+1|a^y+1$, then $a^y+1 = n (a^x+1)$ for some integer $n$, hence
\[
y \equiv \frac{a^{y}+1}{a+1}= n \frac{a^{x}+1}{a+1} = nx \mod (a+1)
\]
Thus, $y= nx + m(a+1)$ for some integer $m$. We could do the same argument $\mod (ka+1)$ instead of $\mod (a+1)$ for any $k\in\mathbb{N}$ (since $a\equiv -1 \mod (ka+1)$, and so for each such $k$ there is an integer $q$ so that $y=nx+q(ka+1)$. Since $x,y,n>0$, we have that $y>q(ka+1)>qka$, so if we pick $k>y$, then the only way this inequality can hold is if $q=0$, and so $y=nx$. 
\end{solution}


\end{exercise}


\begin{exercise} Let $a,b\in \mathbb{N}$. Show there are infinitely many $n\in \mathbb{N}$ so that $n^b+1 $ does not divide $a^n+1$. 

\begin{solution}
Suppose the contrary that $n^b+1$ does not divide $a^b+1$ for only finitely many $n$, so we have $n^b+1| a^n+1$ for all $n$ large enough (specifically, $n>N$ where $N$ is the last integer where they don't divide).  

Let $n=a^k$, then for $k$ large enough, $a^k>N$ and so
\[
(a^k)^b +1 =a^{kb}+1| a^n+1=a^{a^{k}}+1.
\]
Thus, by the previous problem, $kb|a^k$. Hence, if $k$ is any integer coprime to $a$.


\end{solution}




\end{exercise}


\begin{exercise} ({\bf Challenging}) Show that for any prime $p\not\in \{2,5\}$, there is a number whose digits are only 1's (that is, an integer of the form $111.....1$) that is divisible by $p$.

\begin{solution}
By Fermat's little theorem, $10^{p-1}\equiv 1 \mod p$. So let $a$ be the integer consisting of $p-1$ 1's. Then $10^{p-1}a+a$ consists of $2(p-1)$ $1$'s since $10^{p-1}a$ has zeros for the first $p-1$ places and then $p-1$ 1's after that. Similarly, $10^{(p-1)^2}a+10^{p-1}a+a$ has $3(p-1)$ 1's, since $10^{p-1}a+a$ was $2(p-1)$ 1's, and $10^{2(p-1)}a$ is zeros for the first $2(p-1)$ places and then $1$'s for another $(p-1)$ places.

We claim that $10^{(p-1)n}a+c\dots + 10^{(p-1)}a+a$ consists of $(n+1)(p-1)$ 1's. We have already proven the base case, so suppose this is true for some $n$, then $10^{(p-1)(n+1)}a$ consists of $(p-1)(n+1)$ 0's followed by $(p-1)$ 1's, and by the induction hyptohesis, $10^{(p-1)n}a+c\dots + 10^{(p-1)}a+a$ consists of $(n+1)(p-1)$ 1's. Thus, $10^{(p-1)(n+1)}a+c\dots + 10^{(p-1)}a+a$ consists of $(n+2)(p-1)$ 1's, which proves the claim. 

By Fermat's little theorem $10^{(p-1)n}\equiv 1^{n} \equiv 1  \mod p$ for all $p$. Thus,
\[
10^{(p-1)(p-1)}a+c\dots + 10^{(p-1)}a+a
\equiv a+\cdots + a = pa\equiv 0 \mod p.
\]
Thus, $10^{(p-1)^2}+\cdots + 10^{(p-1)}a+a$ is a digit consisting of only $1$'s that is divisible by $p$. 


\end{solution}

\end{exercise}







%Interesting open problems:
%
%
%\begin{question}[Bocard's problem]
%Are there infinitely many $n\in\mathbb{N}$ so that $n!+1$ is a square? Only 3 integers are known: $4!+1=5^2$, $5!+1=11^2$, and $7!=71^2$. 
%\end{question}






%----------------------------------------------------------------------------------------
%	CHAPTER 3
%----------------------------------------------------------------------------------------

%\chapterimage{ima2} % Chapter heading image


%----------------

%----------------------------------------------------------------------------------------
%	BIBLIOGRAPHY
%----------------------------------------------------------------------------------------
%
%\chapter*{Bibliografía}
%\addcontentsline{toc}{chapter}{\textcolor{ocre}{Bibliografía}}
%\section*{Books}
%\addcontentsline{toc}{section}{Books}
%\printbibliography[heading=bibempty,type=book]
%
%\begin{itemize}
%	\item GREENE, W.H. (2003) “Econometric Analysis”5ª edición. Prentice Hall N.J. Capítulo 21
%\\\\
%    \item WOOLDRIDGE, J.M. (2010) “Introducción a la Econometría: Un Enfoque Moderno". 4ª edición. Cengage Learning. Capítulo 17
%
%\end{itemize}


%----------------------------------------------------------------------------------------
%	INDEX
%----------------------------------------------------------------------------------------

\cleardoublepage
\phantomsection
\setlength{\columnsep}{0.75cm}
\addcontentsline{toc}{chapter}{\textcolor{ocre}{Índice Alfabético}}
\printindex

%----------------------------------------------------------------------------------------

\end{document}