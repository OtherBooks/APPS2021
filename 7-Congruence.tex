%%%%%%%%%%%%%%%%%%%%%%%%%%%%%%%%%%%%%%%%%
% The Legrand Orange Book
% LaTeX Template
% Version 2.0 (9/2/15)
%
% This template has been downloaded from:
% http://www.LaTeXTemplates.com
%
% Mathias Legrand (legrand.mathias@gmail.com) with modifications by:
% Vel (vel@latextemplates.com)
%
% License:
% CC BY-NC-SA 3.0 (http://creativecommons.org/licenses/by-nc-sa/3.0/)
%
% Compiling this template:
% This template uses biber for its bibliography and makeindex for its index.
% When you first open the template, compile it from the command line with the 
% commands below to make sure your LaTeX distribution is configured correctly:
%
% 1) pdflatex main
% 2) makeindex main.idx -s StyleInd.ist
% 3) biber main
% 4) pdflatex main x 2
%
% After this, when you wish to update the bibliography/index use the appropriate
% command above and make sure to compile with pdflatex several times 
% afterwards to propagate your changes to the document.
%
% This template also uses a number of packages which may need to be
% updated to the newest versions for the template to compile. It is strongly
% recommended you update your LaTeX distribution if you have any
% compilation errors.
%
% Important note:
% Chapter heading images should have a 2:1 width:height ratio,
% e.g. 920px width and 460px height.
%
%%%%%%%%%%%%%%%%%%%%%%%%%%%%%%%%%%%%%%%%%

%----------------------------------------------------------------------------------------
%	PACKAGES AND OTHER DOCUMENT CONFIGURATIONS
%----------------------------------------------------------------------------------------

%\documentclass[11pt,fleqn,dvipsnames]{book} % Default font size and left-justified equations
\documentclass[11pt,dvipsnames]{book} 

%----------------------------------------------------------------------------------------

\input{structure} % Insert the commands.tex file which contains the majority of the structure behind the template



%%agregué




%%%My stuff


%\usepackage[utf8x]{inputenc}
\usepackage[T1]{fontenc}
\usepackage{tgpagella}
%\usepackage{due-dates}
\usepackage[small]{eulervm}
\usepackage{amsmath,amssymb,amstext,amsthm,amscd,mathrsfs,eucal,bm,xcolor}
\usepackage{multicol}
\usepackage{array,color,graphicx}



\usepackage{epigraph}
%\usepackage[colorlinks,citecolor=red,linkcolor=blue,pagebackref,hypertexnames=false]{hyperref}

%\theoremstyle{remark} 
%\newtheorem{definition}[theorem]{Definition}
%\newtheorem{example}[theorem]{\bf Example}
%\newtheorem*{solution}{Solution:}


\usepackage{centernot}


\usepackage{filecontents}

\usepackage{tcolorbox} 





% Ignore this part, this is the former way of hiding and unhiding solutions, new version is after this
%
%\begin{filecontents*}{MyPackage.sty}
%\NeedsTeXFormat{LaTeX2e}
%\ProvidesPackage{MyPackage}
%\RequirePackage{environ}
%\newif\if@hidden% \@hiddenfalse
%\DeclareOption{hide}{\global\@hiddentrue}
%\DeclareOption{unhide}{\global\@hiddenfalse}
%\ProcessOptions\relax
%\NewEnviron{solution}
%  {\if@hidden\else \begin{tcolorbox}{\bf Solution: }\BODY \end{tcolorbox}\fi}
%\end{filecontents*}
%
%
%
%\usepackage[hide]{MyPackage} % hides all solutions
%\usepackage[unhide]{MyPackage} %shows all solutions




%\usepackage[unhide,all]{hide-soln} %show all solutions
\usepackage[unhide,odd]{hide-soln} %hide even number solutions
%\usepackage[hide]{hide-soln} %hide all solutions






\def\putgrid{\put(0,0){0}
\put(0,25){25}
\put(0,50){50}
\put(0,75){75}
\put(0,100){100}
\put(0,125){125}
\put(0,150){150}
\put(0,175){175}
\put(0,200){200}
\put(25,0){25}
\put(50,0){50}
\put(75,0){75}
\put(100,0){100}
\put(125,0){125}
\put(150,0){150}
\put(175,0){175}
\put(200,0){200}
\put(225,0){225}
\put(250,0){250}
\put(275,0){275}
\put(300,0){300}
\put(325,0){325}
\put(350,0){350}
\put(375,0){375}
\put(400,0){400}
{\color{gray}\multiput(0,0)(25,0){16}{\line(0,1){200}}}
{\color{gray}\multiput(0,0)(0,25){8}{\line(1,0){400}}}
}



%\usepackage{tikz}

%\pagestyle{headandfoot}
%\firstpageheader{\textbf{Proofs \& Problem Solving}}{\textbf{Homework 1}}{\textbf{\PSYear}}
%\runningheader{}{}{}
%\firstpagefooter{}{}{}
%\runningfooter{}{}{}

%\marksnotpoints
%\pointsinrightmargin
%\pointsdroppedatright
%\bracketedpoints
%\marginpointname{ \points}
%\totalformat{[\totalpoints~\points]}

\def\R{\mathbb{R}}
\def\Z{\mathbb{Z}}
\def\N{{\mathbb{N}}}
\def\Q{{\mathbb{Q}}}
\def\C{{\mathbb{C}}}
\def\gcd{{\rm gcd}}


%%end of my stuff


\usepackage[hang, small,labelfont=bf,up,textfont=it,up]{caption} % Custom captions under/above floats in tables or figures
\usepackage{booktabs} % Horizontal rules in tables
\usepackage{float} % Required for tables and figures in the multi-column environment - they




\usepackage{graphicx} % paquete que permite introducir imágenes

\usepackage{booktabs} % Horizontal rules in tables
\usepackage{float} % Required for tables and figures in the multi-column environment - they

\numberwithin{equation}{section} % Number equations within sections (i.e. 1.1, 1.2, 2.1, 2.2 instead of 1, 2, 3, 4)
\numberwithin{figure}{section} % Number figures within sections (i.e. 1.1, 1.2, 2.1, 2.2 instead of 1, 2, 3, 4)
\numberwithin{table}{section} % Number tables within sections (i.e. 1.1, 1.2, 2.1, 2.2 instead of 1, 2, 3, 4)

\newcommand{\cell}[1]% #1 = text
{\makebox[1em]{#1}}

%\setlength\parindent{0pt} % Removes all indentation from paragraphs - comment this line for an assignment with lots of text

%%hasta aquí


\begin{document}

%----------------------------------------------------------------------------------------
%	TITLE PAGE
%----------------------------------------------------------------------------------------

\begingroup
\thispagestyle{empty}
\begin{tikzpicture}[remember picture,overlay]
\coordinate [below=12cm] (midpoint) at (current page.north);
\node at (current page.north west)
{\begin{tikzpicture}[remember picture,overlay]
\node[anchor=north west,inner sep=0pt] at (0,0) {\includegraphics[width=\paperwidth]{Figures/blank.png}}; % Background image
\draw[anchor=north] (midpoint) node [fill=ocre!30!white,fill opacity=0.6,text opacity=1,inner sep=1cm]{\Huge\centering\bfseries\sffamily\parbox[c][][t]{\paperwidth}{\centering Proofs and Problem Solving \\[15pt] % Book title
{\huge Week 7: Congruence}\\[20pt] % Subtitle
{\Large Notes  based on Martin Liebeck's \\ \textit{A Concise Introduction to Pure Mathematics}}}}; % Author name
\end{tikzpicture}};
\end{tikzpicture}
\vfill
\endgroup



%----------------------------------------------------------------------------------------
%	COPYRIGHT PAGE
%----------------------------------------------------------------------------------------

%\newpage
%~\vfill
%\thispagestyle{empty}

%\noindent Copyright \copyright\ 2013 John Smith\\ % Copyright notice

%\noindent \textsc{Published by Publisher}\\ % Publisher

%\noindent \textsc{book-website.com}\\ % URL

%\noindent Licensed under the Creative Commons Attribution-NonCommercial 3.0 Unported License (the ``License''). You may not use this file except in compliance with the License. You may obtain a copy of the License at \url{http://creativecommons.org/licenses/by-nc/3.0}. Unless required by applicable law or agreed to in writing, software distributed under the License is distributed on an \textsc{``as is'' basis, without warranties or conditions of any kind}, either express or implied. See the License for the specific language governing permissions and limitations under the License.\\ % License information

%\noindent \textit{First printing, March 2013} % Printing/edition date

%----------------------------------------------------------------------------------------
%	TABLE OF CONTENTS
%----------------------------------------------------------------------------------------

\chapterimage{Figures/blank.png} % Table of contents heading image

%\chapterimage{chapter_head_1.pdf} % Table of contents heading image

\pagestyle{empty} % No headers

 \tableofcontents % Print the table of contents itself

\cleardoublepage % Forces the first chapter to start on an odd page so it's on the right

\pagestyle{fancy} % Print headers again

%----------------------------------------------------------------------------------------
%	PART
%----------------------------------------------------------------------------------------



\part{Week 7: Modular Arithmetic}


\chapterimage{Figures/blank.png} 




%\chapterimage{} 

\setcounter{chapter}{11}

\setcounter{page}{0}


\chapter{Modular Arithmetic}


%
%\epigraph{\it And this proposition is generally true for all progressions and for all prime numbers; the proof of which I would send to you, if I were not afraid to be too long.}{Pierre de Fermat, 1640, in a letter to his friend Fr\'{e}nicle de Bessy commenting on his what later was later to be called, ironically, {\it Fermat's Little Theorem}.}





\indent Is $30^{99}+61^{100}$  divisible by 31? This seems like a daunting question. However, in a couple of pages, you'll learn a trick that will make this and other problems involving divisibility of very large numbers much easier. This method involves {\it modular arithmetic}.  

The framework of modular arithmetic as we are familiar with it was first developed by Gauss in his book {\it  Disquisitiones Arithmeticae} in 1801, whereas many fundamental results that are now stated in terms of congruences were in fact proven much earlier; for example, Fermat's Little Theorem below was proven in 1640, though now it is standard to formulate it using congruences. 

Despite its classical roots, modular arithmetic is important for both mathematics and applications. In mathematics, it is fundamental for studying {\it abstract algebra}, which you will learn more about in {\it Fundamentals of Pure Math} next year and in {\it Honours Algebra} the year after that. Moreover, it is also a cornerstone of modern cryptography (Chapter 15 in Liebeck's book has a good introduction to {\it public key cryptography}). \\

If you like the material from this week, you might enjoy taking the following classes:

\begin{itemize}
\item Fundamentals of Pure Math (2nd year)
\item Honours Algebra, Introduction to Number Theory (3rd year)
\item Commutative Algebra, Group Theory, Galois Theory, Algebraic Geometry (4th Year)
\end{itemize}

\section{Arithmetic modulo $m$}

\begin{definition}
\label{d:mod}
For $m\in \mathbb{N}$ and $a,b\in \mathbb{Z}$, we write $a = b\mod m$ or sometimes $a  = b\mod m$ (read {\it $a$ is congruent to $b$ modulo $m$}) if one of the following equivalent conditions holds:
 
\begin{itemize}
\item $m$ divides $a-b$;
\item $a$ and $b$ have same remainder when divided by $m$;
\item $b=qm+a$ for some $q\in \mathbb{Z}$.
\end{itemize}
\end{definition}

\begin{proposition}
For any integer $n$, there exists a unique element $\overline{n}$ such that $n = \overline{n} \mod m$.
\end{proposition}
\begin{proof}
Exercise.
\end{proof}

\begin{example}
$32=2\mod 10$. $82=1 \mod 3$. 
\end{example}
 
\begin{theorem}
We have the following properties for integers $a,b,c\in\mathbb{Z}$ and $m\in\mathbb{N}$:
\begin{enumerate}[label=(\alph*)]
\item $a = a\mod m$. 
\item If $a = b\mod m$, then $b = a\mod m$. 
\item If $a = b\mod m$ and $b = c\mod m$, then $a = c\mod m$. 
\end{enumerate}
\end{theorem}

\begin{proof}
\begin{enumerate}[label=(\alph*)]
\item Since $m$ divides $a-a=0$, we have that $a = a$.
\item If $a = b\mod m$, then $m$ divides $(a-b)$, so $m$ divides $(b-a)$ as well, hence $b = a \mod m$. 
\item Suppose $a = b \mod m$ and $b = c\mod m$. Then $m$ divides $a-b$, so $a-b=km$ for some integer $k$. Moreover, $m$ divides $b-c$, so $b-c=jm$ for some integer $j$. Thus,
\[
a-c = a-b+b-c = km+jm = (k+j)m,\]
hence $m$ divides $(a-c)$, thus $a = c \mod m$.\qedhere
\end{enumerate}
\end{proof}

\begin{theorem}%[Propositions 13.3 and 13.4]
\label{t:mod-artithmetic}
Suppose $a  = x \mod m$ and $b = y \mod m$. Then
\begin{enumerate}[label=(\alph*)]
\item $-a = -x \mod m$;
\item $a+b = x+y \mod m$;
\item $ab  = xy \mod m$;
\item for any natural number $k$, one has $a^{k} = x^{k} \mod m$.
\end{enumerate}
\end{theorem}

\begin{proof}
We have that $m$ divides $a-x$ and $m$ divides $b-y$
\begin{enumerate}[label=(\alph*)]
\item It is clear that $m$ divides $-(a-x)=(-a)-(-x)$, so that $-a = -x \mod m$.
\item The Linear Combination Lemma implies that $m$ divides
\[
(a-x)+(b-y)=(a+b)-(x+y),
\]
and so $a+b = x-y\mod m$.   
\item $a = x \mod m$ implies $a=x+pm$ for some $p\in\mathbb{Z}$ and $b = y \mod m$ implies $b=y+qm$ for some $q\in\mathbb{Z}$. Thus,
\[
ab = (x+pm)(y+qm) = xy+pmy+xqm+ pqm^2 = xy + m(py+xq+pqm),\]
and so $ab = xy$ by Definition \ref{d:mod}(3).
\item This can be proven by induction. For the base case of $k=1$, this is immediate. Now suppose $a^{k} = x^{k}\mod m$ for some $k\geq 1$. Then by part (b) of this theorem with $b=a^{k}$ and $x=y^{k}$ and our induction hypothesis
\[
a^{k+1}=a\cdot a^{k} = a x^{k} \mod m\]
and then applying part (b) again but with $b=y=x^k$,
\[
a x^{k} = x\cdot x^k=x^{k+1}.\]
This proves the induction and hence the theorem. \qedhere
\end{enumerate}
\end{proof}

This result proves that many of the algebraic manipulations we enjoy performing on integers -- addition, subtraction, and multiplication -- can be done modulo $m$ systematically.
This simplifies many challenging-looking problems.
Let's revisit our earlier example: is $30^{99}+61^{100}$ divisible by 31?
Note that $30=31-1=-1\mod 31$ and $61=2\cdot 31-1=-1\mod 31$, and so 
\[
30^{99}+61^{100} = (-1)^{99}+(-1)^{100}=-1+1=0 \mod 31.
\]
Indeed, $31$ divides $30^{99}+61^{100}$.\\

Theorem \ref{t:mod-artithmetic} says we can add and multiply modulo $m$. Can we also divide? That is, if $ad = bd\mod m$, do we also have $a = b\mod m$? This is \emph{not} always the case.
Perhaps the simplest example is the following: $2 = 0 \mod 2$, but we cannot divide this equation by $2$, because $1 \neq 0 \mod 2$.
More subtly, $12 = 24 \mod 6$, but here we cannot divide by $3$, because $4 \neq 8 \mod 6$. 
The following theorem says when we can divide in this way. 

\begin{proposition}
\label{p:xa=ya-x=a}
Let $d$ and $m$ be coprime. If $x,y\in \mathbb{Z}$ and $xd=yd\mod m$, then $x = y\mod m$.
In particular, if $p$ is prime and $p$ does not divide $a$, then $xa = ya\mod p$ implies that $x = y\mod p$. 
\end{proposition}

\begin{proof}
Suppose $a$ and $m$ are coprime and $xa = ya\mod m$. Then $m$ divides $xa-ya=(x-y)a$. Since $m$ and $a$ are coprime, $m$ divides $x-y$, and so $x = y\mod m$ by definition. The second part of the proposition follows from the first.
\end{proof}

\begin{example}
Find an integer $x\in \{0,1,..,6\}$ so that $4^{6} = x\mod 7$. \\

Let's look at some powers of $4$ and see what they are modulo $ 7$. 

\begin{align*}
4^2 & = 16 =14+2 = 2 \mod 7 \\
4^4 & = (4^2)^2  = 2^2 =4 \mod 7.
\end{align*}


Thus,
\[
4^6=4^2\cdot 4^4 = 2\cdot 4=8 = 1 \mod 7.
\]
This is certainly much quicker than computing $4^6$ by hand and then doing long division!
\end{example}



\section{Solving linear equations modulo $m$}


When can we solve $ax = b\mod m$ for $x$? Before we even start, it's good to have a test to see if there is actually a solution:

\begin{theorem}%[(Liebeck Proposition 13.6)] 
The equation $ax = b\mod m$ has a solution if and only if $\gcd(a,m)$ divides $b$. 
\end{theorem}


\begin{proof} An integer $x$ is a solution to $ax = b\mod m$ if and only if $ax=b-my$ for some integer $y$.
In other words, $ax = b\mod m$ has a solution if and only if $b$ can be written as $ax + my$ for some integers $x$ and $y$.

If $b = ax+my$, then the Linear Combination Lemma ensures that $\gcd(a,m)$ divides $b$. 

Conversely, assume that $\gcd(a,m)$ divides $b$, so that $b = h\gcd(a,m)$ for some integer $h$.
Bezout's identity implies that there exist integers $x'$ and $y'$ such that $\gcd(a,m)=ax'+my'$.
Hence if $x = hx'$ and $y=hy'$, then
\[
b = h\gcd(a,m) = ax + my,
\]
as desired.
\end{proof}

\begin{example}
Find $x\in \{0,1,...,13\}$ so that $18x = 10\mod 14$. \\

We see that $\gcd(18,14)=2$ divides $10$, so there is at least one solution.
We can use the Euclidean Algorithm to express $2$ as an integer linear combination:
since $18 - 14 = 4$, and $14 - 3\cdot 4 = 2$ is the gcd, so we can write
\[
2 = 14 - 3\cdot 4 = 4\cdot 14 - 3 \cdot 18.
\]
Multiplying this by $5$, we obtain
\[
10 = 20\cdot 14 - 15 \cdot 18.
\]
So $-15$ is a solution, and so is $3 = -15 \mod 18$.
\end{example}

Note that when finding a solution to $a = bx\mod m$, any integer $x$ is congruent modulo $m$ to a number in $\{0,1,...,m-1\}$. Thus, whenever we ask for a solution, we prefer to write it as one of these integers.

\begin{exercise}
Find $x$ so that $7x = 8\mod 12$. \\
\begin{solution}
Since $\gcd(7,12)=1$, we can use the Euclidean Algorithm:
\[
12-7 = 5 \text{\quad and \quad} 7-5 = 2 \text{\quad and \quad} 5-2\cdot 2 = 1.
\]
We find
\[
1 = 5 - 2\cdot 2 = 3\cdot 5 - 2\cdot 7 = 3\cdot 12 - 5\cdot 7,
\]
so by multiplying by $8$, we get
\[
8 = 24 \cdot 12 - 40 \cdot 7.
\]

Thus $x = -40$ is a solution, and so is $8 = -40 \mod 12$. 
\end{solution}
\end{exercise}

\section{$\mathbb{Z}/m$}
 
In this section we introduce a set $\mathbb{Z}/m$ that will be important your future algebra classes.
Just like the integers, the rationals, and the reals, it is what is called a \emph{ring}: it is equipped with addition, subtraction, and multiplication operations.
Unlike $\mathbb{Z}$, $\mathbb{Q}$, or $\mathbb{R}$, the set $\mathbb{Z}/m$ is {\it finite}.

As a set,
\[
\mathbb{Z}/m=\{0,1,\dots,m-1\}. 
\]
On this set we shall define addition and multiplication operations: for any $x,y \in \mathbb{Z}/m$,
\[
x + y = z \mbox{ where $z\in \{0,1,...,m-1\}$ is the unique element such that } x + y = z\mod m,
\] 
and
\[
x \cdot y = z \mbox{ where $z\in \{0,1,...,m-1\}$ is the unique element such that } x \cdot y = z\mod m,
\]
There is an additive inverse:
\[
- x = m-x
\]
(Some books write $\mathbb{Z}_m$ for $\mathbb{Z}/m$, but most algebraists reserve this notation for a very different object!!)

\begin{example}
Let's write the addition and multiplication tables for $\mathbb{Z}/5=\{0,1,2,3,4\}$.

\begin{center}
\begin{tabular}{| c| c | c | c | c | c |}
\hline
\cell{$+$} & \cell{0} & \cell{1} & \cell{2} & \cell{3} & \cell{4}\\
\hline
\cell{0} & \cell{0} & \cell{1} & \cell{2} & \cell{3} & \cell{4} \\ 
\hline
\cell{1} & \cell{1} & \cell{2} & \cell{3} & \cell{4} & \cell{0} \\ 
\hline
\cell{2} & \cell{2} & \cell{3} & \cell{4} & \cell{0} & \cell{1} \\ 
\hline
\cell{3} & \cell{3} & \cell{4} & \cell{0} & \cell{1} & \cell{2} \\ 
\hline
\cell{4} & \cell{4} & \cell{0} & \cell{1} & \cell{2} & \cell{3} \\ 
\hline
\end{tabular}

\smallskip

\begin{tabular}{| c| c | c | c | c | c |}
\hline
\cell{$\times$} & \cell{0} & \cell{1} & \cell{2} & \cell{3} & \cell{4}\\
\hline
\cell{0} & \cell{0} & \cell{0} & \cell{0} & \cell{0} & \cell{0} \\ 
\hline
\cell{1} & \cell{0} & \cell{1} & \cell{2} & \cell{3} & \cell{4} \\ 
\hline
\cell{2} & \cell{0} & \cell{2} & \cell{4} & \cell{1} & \cell{3} \\ 
\hline
\cell{3} & \cell{0} & \cell{3} & \cell{1} & \cell{4} & \cell{2} \\ 
\hline
\cell{4} & \cell{0} & \cell{4} & \cell{3} & \cell{2} & \cell{1} \\ 
\hline
\end{tabular}
\end{center}
\end{example}
 
We can check that for $m\in\mathbb{N}$, $\mathbb{Z}/m$ has all the nice algebraic properties of being a field (recall this definition from Week 1) apart from perhaps property (M4), which said that each nonzero element must have a multiplicative inverse.
An element $x\in \mathbb{Z}/m$ is {\it invertible} if there exists $y\in \mathbb{Z}/m$ so that $x\cdot y=1$.

If $m$ is composite, then not every nonzero element $\mathbb{Z}/m$ is invertible.
Consider the multiplication table of $\mathbb{Z}/4$:
\begin{center}
\begin{tabular}{| c| c | c | c | c |}
\hline
\cell{$\times$} & \cell{0} & \cell{1} & \cell{2} & \cell{3} \\
\hline
\cell{0} & \cell{0} & \cell{0} & \cell{0} & \cell{0}  \\ 
\hline
\cell{1} & \cell{0} & \cell{1} & \cell{2} & \cell{3}  \\ 
\hline
\cell{2} & \cell{0} & \cell{2} & \cell{0} & \cell{2}  \\ 
\hline
\cell{3} & \cell{0} & \cell{3} & \cell{2} & \cell{1}  \\ 
\hline
\end{tabular}
\end{center}
As you can see, $2x=1$ has no solution modulo $4$.

Observe that an element $a \in \mathbb{Z}/m$ is invertible if and only if the equation $ax = 1 \mod m$ admits a solution.
Such a solution exists if and only if $\gcd(a,m)$ divides $1$, hence if and only if $a$ and $m$ are coprime.
Thus the invertible elements of $\mathbb{Z}/m$ are precisely those elements that are coprime to $m$.
In particular, if $p$ is prime, then every nonzero element of $\mathbb{Z}/p$ is coprime to $p$, so we deduce the following.

\begin{proposition}
Let $m\geq 2$ be a natural number.
Then $\mathbb{Z}/m$ is a field if and only if $m$ is prime.
\end{proposition}

\begin{exercise}
How many invertible elements are there in $\mathbb{Z}/{81}$? 
\end{exercise}

\begin{solution}
An element $y\in\mathbb{Z}/{81}$ is invertible if and only if $x$ and $81$ are coprime.
The is possible if and only if $3$ does not divide $y$.
The number of numbers in $\{0,1,...,80\}$ divisible by $3$ is $\frac{81}{3}=27$, so the number of invertible elements is $81-27 = 54$. 
\end{solution}

Just like there is a notion of inverses for $\mathbb{Z}/m$, we also have a notion of a root. We say $y\in\mathbb{Z}/m$ is the $n$-th root of $x\in\mathbb{Z}$ if $x^n=y$. 


\begin{example}
In $\mathbb{Z}/{5}$, does $2$ have a cubed root?    
Let us test the values: 

\begin{itemize}
\item $0^{3}=0$. 
\item $1^{3}=1$. 
\item $2^{3}=3$ since $2^{3} = 8 = 3\mod 5$.  
\item $3^{3}=2$ since $3^{3} = 27 = 2\mod 5$.  
\item $4^{3}=4$ since $4^{3} = (-1)^{3}=-1 = 4 \mod 5$.  
\end{itemize}
Thus, the unique cubed root of $2$ is $3$. \\
\end{example}
 
However, not all numbers in every $\mathbb{Z}/m$ have roots! Confirm for yourself that $2\in\mathbb{Z}/{4}$ has no cubed root. 

\section{Fermat's Little Theorem}
 
The great thing about congruence is that arithmetic  becomes easier since you can replace big numbers with smaller numbers that are congruent. The following is a little but powerful theorem due to Fermat that allows you to eliminate large powers if you are working modulo some prime. 
 
\begin{theorem}[Fermat's Little Theorem (FLT)]
Let $a$ be an integer, and let $p$ be a prime number.
If $p$ does not divide $a$, then
\[
a^{p-1} = 1 \mod p. 
\]
\end{theorem}

\begin{proof}
Without loss of generality, we may assume that $a \in \{1,\dots,p-1\}$.
We consider the numbers $a,2a,3a,\dots,(p-1)a$. We first observe that each of these numbers is distinct modulo $p$;
that is, for each $c,d\in \mathbb{Z}/p$, one has $ca = da\mod p$ if and only if $c=d$.
This follows from Proposition \ref{p:xa=ya-x=a}.

If $\overline{a},\overline{2a}, \dots, \overline{(p-1)a}$ are the remainders of $a,2a,\dots,(p-1)a$ after dividing by $p$, then we deduce that the sets $\{\overline{a},\overline{2a}, \dots, \overline{(p-1)a}\}$ and $\{1,2,\dots,p-1\}$ are equal
Consequently, 
\[
a\cdot 2a \cdots (p-1)a = \overline{a}\cdot\overline{2a}\cdot \cdots\cdot \overline{(p-1)a} = 1\cdot 2 \cdots \cdot (p-1) \mod p.\]
Thus
\[
(p-1)!a^{p-1} = (p-1)!\mod p.
\]
Now since $p$ does not divide $(p-1)!$, Proposition \ref{p:xa=ya-x=a} implies that $a^{p-1} = 1 \mod p$, as desired.
\end{proof}

\begin{example}
Find $5^{100}\mod 7$. \\

Since $7$ does not divide $5$, the FLT says $5^{6} = 1\mod 7$. Thus, we can shave off any power of $5$ from $5^{100}$ that is a multiple of $6$. Thus,
\[
5^{100} = 5^{96}5^{4}=(5^{6})^{16}5^{4} = 1^{16}5^{4}= 5^{4}.
\]
Then the rest is as before: 
\begin{align*}
5^{2} & =25 = 4\mod 7\\
5^{4} & =(5^{2})^{2} = 4^{2}=16 = 2\mod 7.
\end{align*}
\end{example}

Fermat's Little Theorem gives us a way of solving a large class of polynomial equations mod $m$:

\begin{theorem}
\label{t:x^n=bmodp}
Let $n\in\mathbb{N}$ and $p$ be a prime number. Assume $n$ and $p-1$ are coprime and $p$ does not divide $b$. Then the equation
\begin{equation}
\label{e:x^n=bmodp}
x^{n} = b\mod p.
\end{equation} 
has exactly one solution $x\in \{1,\dots,p-1\}$.
\end{theorem}

\begin{proof}
Since $n$ and $p-1$ are coprime, there are integers $s,t$ so that $sn+t(p-1)=1$. 

Let us prove the uniqueness.
If $x_1,x_2 \in \{1,\dots,p-1\}$ are two such solutions, then $x_1^n=x_2^n \mod p$, and so for $i \in \{1,2\}$,
\[
x_i = x_i^{sn+t(p-1)} = x_i^{sn}x_i^{t(p-1)} = b^s(x_i^t)^{p-1} \mod p.
\]
Since $p$ does not divide $x_i^t$, Fermat's Little Theorem that each $x_i = b^s \mod p$, whence $x_1=x_2$.

Now let us prove the existence.
Let $x$ be the remainder after dividing $b^s$ by $p$.
Now in $\mathbb{Z}/p$,
\[
x^{n}  = (b^{s})^{n}=b^{ns}  = b^{1-t(p-1)}=b\cdot (b^{p-1})^{-t}.\]
(Note that this makes sense, since $p$ does not divide $b$, so $b$ is invertible.)
Since $p$ does not divide $b$, Fermat's Little Theorem implies that $x^n = b \mod p$.
\end{proof}

Notice that even though the choice of $s, t$ is not unique, the remainder after dividing $b^s$ by $p$ \emph{is} unique.
In other words, if $s_1,t_1,s_2,t_2$ are integers such that for $i \in \{1,2\}$, we have
\[
s_in+t_i(p-1)=1,
\]
then $b^{s_1} = b^{s_2} \mod p$.
To see this directly, note that in $\mathbb{Z}/p$, we have by FLT:
\[
b^{s_1} = (b^{s_1})^{s_2n+t_2(p-1)} = b^{s_1s_2}(b^{s_1t_2})^{p-1}=b^{s_1s_2},
\]
and by the same reasoning:
\[
b^{s_2} = (b^{s_2})^{s_1n+t_1(p-1)} = b^{s_1s_2}(b^{s_2t_1})^{p-1}=b^{s_1s_2}.
\]

We now have a blueprint to how to solve equations of the form \eqref{e:x^n=bmodp} when $n$ and $p-1$ are coprime and $p$ does not divide $b$. 

\begin{example}
 Find $x\in\{0,1,...16\}$ so that $ x^{7} = 4\mod 17$.  \\

Since $\gcd(7,17-1)=\gcd(7,16)=1$, we can use the Euclidean Algorithm to show that $7\cdot 7=1+3\cdot 16$.  Thus, if $x$ solves the above equation, then $4 = x^{7}$,   and so
\[
4^{7} = (x^{7})^{7}  
= x^{7\cdot 7}  
= x^{1+3\cdot 16} 
 = x\cdot (x^{16})^{3} 
   = x\cdot (1)^{3} 
   = x.
\] 
Thus, $x = 4^{7}$, so we need to find a representative in $\{0,1,...,16\}$. Note that $4^{2}=16 = -1\mod 17$,  so 
\[
4^{7}=(4^{2})^{3} \cdot 4
 = (-1)^{3} \cdot 4
=-4 = 13\mod 17. 
\] 
Thus, $x=13$ is the unique solution in $\{0,1,...,16\}$. 
\end{example}



Not all polynomial equations can be solved using the above method:
 

\begin{example}
Find a solution to $x^{22} = 3 \mod 11$.  \\

Note that $\gcd(22,11-1)=\gcd(22,10)=2$, so we can't use the method in Theorem \ref{t:x^n=bmodp}. However, let's suppose $x$ is a solution to the above equation and try to narrow down what it must be. Note that $11$ does not divide $x$   (otherwise $x^{22} = 0$).  Thus, FLT implies $x^{10} = 1 \mod 11$. Hence, 
\[
x^{22}=x^{2}(x^{10})^{2}   = x^{2}.\]  
Now we just need to solve $x^{2} = 3 \mod 11$.  Again, $\gcd(2,11-1)=\gcd(2,10)=2$, so we still can't use the method of Theorem \ref{t:x^n=bmodp}.  But now the power $2$ is small enough we can just try values $x\in \{0,1,...,10\}$ and see what works: 

\begin{itemize}
\item $1^{2}=1$ 
\item $2^{2}=4$ 
\item $3^{2}=9$ 
\item $4^{2}=16 = 5 \mod 11$.  
\item $5^{2}=25 = 3\mod 11$.  
\end{itemize}
Thus, $x=5$ is a solution to $x^{22} = 3 \mod 11$:  
\[
5^{22}=5^{2}(5^{10})^{2} = 3\cdot (1)^{2}=3.
\]

Note that if we can't use Theorem \ref{t:x^n=bmodp}, then the solution may not be unique. In this example, note that $x=6$ is also a solution
\end{example}

\begin{example}
 Find $x\in\{0,1,...,10\}$ that solves $x^{3} = 9 \mod 11$.  \\

Note $\gcd(3,11-1)=\gcd(3,10)=1$, so this tells us that the method of Theorem \ref{t:x^n=bmodp} should work in finding the unique solution. Note also that $7\cdot 3 = 21=1+2\cdot 10$.  Thus, if $x$ is the solution, then $9 = x^{3}\mod 11$ implies  
\[
9^{7} = (x^{3})^{7}  
=x^{3\cdot 7}  
 =x^{1+2\cdot 10}   
  =x\cdot (x^{10})^{2}  
   = x\cdot(1)^{2}  
  =x \mod 11.\]
  Thus, $x = 9^{7}$. Now we must find $x\in \{0,1,...,10\}$ so that $x = 9^{7}$.   
  \begin{itemize}
  \item $9^{2}=81   = 4 \mod 11$
  \item $9^{4}=(9^{2})^{2}   = 4^{2}  =16 = 5\mod 11$.   
  \item $9^{7}=9^{4}\cdot 9^{2}\cdot 9  = 5\cdot 4\cdot 9 =180 =176+4 = 4\mod 11$. 
  \end{itemize}
    
Thus, $x=4$ is the solution.  We can also check $4^{3}=64=55+9 = 9\mod 11$. 


\end{example}



\subsection{Diophantine equations}

As congruences relate to divisibility, we can also use them to solve diophantine equations. 

\begin{example}
What are the integer solutions to $9x^{2}+9x+2=y^{4}$?

 
 We will present two methods for comparison, first using the usual divisbility methods from last week, and then a second method using modular arithmetic.\\

\noindent {\bf Method 1:}  If $x,y$ are integer solutions, then
\[
y^{4}=9x^{2}+9x+2 =(3x+2)(3x+1).
\] 
 Since $y^{4}\geq 0$, we know $3x+1,3x+2> 0$ or $<0$.  First assume $>0$.   They are coprime since $\gcd(3x+1,3x+2)$ divides $ (3x+2)-(3x+1)=1$.   Thus, $3x+1=a^{4}$ and $3x+2=b^{4}$ for some integers $a$ and $b$.   (We can assume they are non-negative). But then $b^{4}-a^{4} =3x+2-(3x+1) =1$.   This is only possible if $(a,b)=(0,1)$  (exercise :). We can't have $a=0$ since then $3x+1=0$,  which is impossible if $x$ is an integer. The same happens if $3x+1,3x+2<0$.  Thus, there are no solutions.




\vspace{10pt}

As you can see, the above solution is quite long (and we didn't do some of the details). Now let's instead use modular arithmetic for a shorter proof:



 \vspace{10pt}

\noindent {\bf Method 2:} If $x,y$ are integer solutions to $9x^{2}+9x+2=y^{4}$, then 
\[
y^{4}=9x^{2}+9x+2  
 = 2\mod 3.
\] 
Let $z\in \{0,1,2\}$ be such that $z = y\mod 3$, then   $z^{4} = y^{4} = 2\mod 3$.  
 So let's test some $z$'s! 
\begin{itemize}
\item $0^{4}=0$ 
\item $1^{4}=1$ 
\item $2^{4}=16 = 1 \mod 3$.  
\end{itemize}
So there are no $z\in \{0,1,2\}$ so that $z^{4} = 2 \mod 3$.  Thus, there are no integer solutions to $9x^{2}+9x+2=y^{4}$.




\end{example}


 

\section{Exercises}


The relevant exercises in Liebeck's book are in Chapters 13 and 14.

\begin{exercise} Determine whether the following equations have integer solutions:
\begin{itemize}
\item $x^2+y^2=9z+3$.

\begin{solution}
If $(x,y,z)$ was a solution, then $x^2+y^2 = 3\mod 9$, but the squares modulo $9$ are $0,1,4,0,7,0,0,1$, and adding any pair of these gives $0,1,2,4,5,7,8$ modulo $9$, and none of these are $3$, thus there cannot be any solutions to the original equation.
\end{solution}


\item $3x^2-y^2=-2$.

\begin{solution}
If $(x,y)$ was a solution, then $-y^2 = -2\mod 3$,and so $y^2 = 2\mod 3$, but if we look at the squares of integers mod 3, we see that they are just $0$ and $1$, so there is no integer $y$ for which $y^2 = 2 \mod 3$, thus there are no solutions to the original equation.
\end{solution}



\item $3x^2+2=y^2+6z^3$.

\begin{solution}
If $(x,y,z)$ is a  solution, we see then that $y^2 = 2\mod 3$, and we have already seen that this has no solutions. 

\end{solution}


\end{itemize}

\end{exercise}



\begin{exercise} Show that If $n\in\mathbb{N}$, then $n^5$ and $n$ have the same last digit (that is, the digit in the $1$'s place). 

\begin{solution}
FLT says that $n^{4} = 1 \mod 5$, hence $n^{5} = n \mod 5$, and so $n^{5}-n = 0\mod 5$. Also, $n^5-n=n^4(n-1)$, and since either $n$ or $n-1$ is even, we have $n^5-n = 0\mod 2$. Thus, we have $2$ divides $n^5-n$ and $5$ divides $ n^5-n$, hence $10$ divides $n^5-n$, so the last digit of $n^5$ and $n$ must be the same.
\end{solution}

\end{exercise}


\begin{exercise} Find the last digit of $7^{7^{7^{7}}}$.

\begin{solution}
$7^2=49 = -1\mod 10$, $7^3 = -7 = 3\mod 10$, $7^4 = 1 \mod 10$. Thus, as we take powers of $7$, the 1's digit cycles through 7,9,3, and 1. So we should look at what $7^{7^7}\mod 4$ is. 

Note that $7 = 3\mod 4$, and $7^2 = 1\mod 4$, thus odd powers of $7$ are equivalent to $3\mod 4$. Since $7^7$ is odd, $7^{7^7} = 3\mod 4$, and so $7^{7^{7^{7}}} = 3\mod 10$.
\end{solution}



\end{exercise}


\begin{exercise} Suppose $n\in\mathbb{N}$ satisfies $n = 3\mod 4$. Show that $n$ has a prime factor $p$ with $p = 3\mod 4$. 

\begin{solution}

Note that $n$ is not even, so $n$ has prime factorizaiton $p_{1}^{r_{1}}\cdots p_{k}^{r_{k}}$ for some primes $2<p_{1}<\cdots < p_{k}$ and integers $r_{i}>0$. Suppose $p_{i} = 1 \mod 4$ for all $i$. Then $p_{i}^{r_{i}} = 1 \mod 4$ as well, and so
\[
n=p_{1}^{r_{1}}\cdots p_{k}^{r_{k}} = 1 \mod 4,
\]
which is a contradiction. Thus, there is $i$ so that $p_{i} = 3\mod 4$.


\end{solution}

\end{exercise}

\begin{exercise}
Show that there are no integer solutions to $m^3-n^3=3$ or $m^3-n^3=4$. (Hint: if there were solutions, there would also be a solution $\mod 7$).

\begin{solution}
Let us look at what cubes are $\mod 7$:
\begin{align*}
1^3 & =1 = 1 \mod 7 \\
2^3 & =8  = 1 \mod 7\\
3^3 & =27  = -1 \mod 7 \\
4^3 & =64  = 1 \mod 7\\
5^3 & =125 = -1 \mod 7\\
6^3 & = 216  = -1 \mod 7
\end{align*}
Thus, the only values $m^3-n^3$ can be $\mod 7$ are $0,\pm 1,$ and $\pm 2$, or alternatively, $0,1,2,5,6$. In particular, $m^3-n^3=3$ and $m^3-n^3=4$ have no solutions.

\end{solution}

\end{exercise}



\begin{exercise} Show that for all primes $p>3$, $p = \pm 1\mod 6$. Use this to show $24$ divides $p^2-1$ for all primes $p>3$.

\begin{solution}
Note that $p$ is odd so $p\not = 2,4\mod 6$. Also, $p>3$, hence $p\neq 3\mod 6$. Thus, the only remaining possibilities are $\pm 1 \mod  6$. Hence, $p=6k\pm 1$ for some integer $k$, so that
\[
p^2-1 = (6k\pm 1)^2-1=36k^2\pm 12k=12(3k^2\pm k).
\]
Note that $3k^2\pm k$ is always even (just consider the different cases when $k$ is odd or even), so in fact $p^2-1$ is divisible by 24. 



\end{solution}

\end{exercise}

%
%\begin{exercise} ({\bf Challenge!}) For which primes $p$ does $(p-1)!+1=p^{k}$ for some integer $k$?
%
%\begin{solution}
%Let's rearrange this equality: this would imply
%\[
%(p-1)! = p^{k}-1=(p-1)\sum_{j=0}^{k-1} p^{j}
%\]
%and so
%\[
%(p-2)! = \sum_{j=0}^{k-1} p^{j}.
%\]
%Notice that $p = 1\mod (p-1)$, hence 
%\[
%(p-2)! = \sum_{j=0}^{k-1} 1=k\mod (p-1).
%\]
%Also notice that for $p>5$, $p-1$ divides $(p-2)!$, and so $(p-2)! = 0\mod (p-1)$, hence the above equation implies $k = 0 \mod (p-1)$, thus $k=\ell(p-1)$ for some integer $\ell$. But then $p^{k}\geq p^{p-1}>(p-1)!$, a contradiction. Hence, $p$ cannot be greater than $5$, so $p=2,3$ or $5$.
%
%\begin{itemize}
%\item If $p=2$, then $(p-1)!+1=2=2^{1}$, so $(p,k)=(2,1)$ is a solution.
%\item If $p=3$, then $(p-1)!+1=2!+1=3=3^{1}$, so $(3,1)$ is a solution.
%\item If $p=5$, then $(p-1)!+1=4!+1=25=5^2$, so $(5,2)$ is a solution. 
%\end{itemize}
%
%Thus, these are the only solutions. 
%
%
%\end{solution}
%
%
%\end{exercise}


\begin{exercise} Show that if $p_{1}<\cdots < p_{31}$ are prime and $30$ divides $p_{1}^{4}+\cdots + p_{31}^{4}$, then $p_{1}=2$, $p_{2}=3$ and $p_{3}=5$. {\it Hint: Assume the contrary and use FLT.}

\begin{solution}

\begin{itemize}
\item  If $p_1\neq 2$ then $p_1>2$ and so $p_1,p_2,\cdots p_{31}$ are all odd, hence $p_1^4+p_2^4+\cdots+p_{31}^4$ is odd, so in particular $30$ does not divide $ p_{1}^{4}+\cdots + p_{n}^{4}$, and we get a contradiction. Thus, $p_{1}=2$.

\item If $p_2\neq 3$ then $p_{i} = \pm 1\mod 3$ since $p_{1}=2 = -1$, and for $i\geq 2$, $p_{i}\neq 3$. Thus, $p_{i}^{2} = 1\mod 3$ by FLT, so $p_i^{4} = 1(\mod 3)$ for $i=1,2,\cdots,31$. Thus, 
\[
p_1^4+p_2^4+\cdots+p_{31}^4 = 31 = 1 \mod 3.
\]
Thus, $p_1^4+p_2^4+\cdots+p_{31}^4$ is not a multiple of $30$. So $p_3=3$. 
\item A similar argument shows that $p_3$ should be $5$: $p_3\neq 5 $ implies $ p_i^4 = 1 (\mod 5)$ for $i=1,2,\cdots,31$ by FLT and $p_1^4+p_2^4+\cdots+p_{31}^4 = 1 (\mod 5)$.
\end{itemize}
\end{solution}


\end{exercise}


\begin{exercise} (Regional Mathematical Olympiad 1998, India) Show that if $5<p_{1}<\cdots < p_{n}$ and $6$ divides $p_{1}^{2}+\cdots + p_{n}^{2}$, then $6$ divides $n$. 

\begin{solution}
From an earlier exercise, we know that $p_{i} = \pm 1 \mod 6$, hence $p_{i}^{2} = 1 \mod 6$, thus 
\[
0=p_{1}^{2}+\cdots + p_{n}^{2}  = n\mod 6,
\]
where in the first equation we used our problem assumption, thus $n = 0\mod 6$, hence $6$ divides $n$. 
\end{solution}


\end{exercise}


%\begin{exercise} How  many  prime  numbers $p$ are  there  such that $29^p+ 1$ is a multiple of $p$?
%
%\begin{solution}
%If $p$ divides $29^{p}+1$, then $0 = 29^{p}+1 \mod p$, and by FLT, $29^{p}+1 = 29+1=30\mod p$, so $p$ divides $30$, hence $p=2,3,$ or $5$. 
%\end{solution}
%
%\end{exercise}


\begin{exercise} Given that $p$ and $8p^2+1$ are prime, find $p$. 

\begin{solution}
By trying out some primes, we can see that it seems like $3$ is the only option, so let's try to prove $p=3$. Suppose $p\neq 3$. Note that any prime $p\neq 3$ satisfies $p = \pm 1 \mod 3$, so $p^2  = 1 \mod 3$. Thus, since $8 = -1\mod 3$, 
\[
8p^2+1 = (-1)1+1=0\mod 3,\]
thus $8p^2+1$ is a prime divisible by 3, which is only possible if $8p^2+1=3$, which is impossible. Thus $p=3$ is the only solution.
\end{solution}


\end{exercise}

%%
%%\begin{exercise} Let $a,x,y\in \mathbb{N}$. Show that if $a>1$ and $a^x+1$ divides $a^y+1$, then $x$ divides $y$. 
%%
%%{\it Hint: Recall the geometric series formula.}
%%
%%\begin{solution}
%%Recall from the geometric series formula that
%%\[
%%\frac{a^y+1}{a+1}   = \sum_{k=0}^{y-1}(-a)^{k}
%%\]
%%Since $a=-1 \mod (a+1)$,
%%\[
%%\frac{a^y+1}{a+1} = \sum_{k=0}^{y-1}(-a)^{k} = \sum_{k=0}^{y-1}1^k=y\mod(a+1)
%%\]
%%The same holds with $x$ in place of $y$. If $a^x+1$ divides $a^y+1$, then $a^y+1 = n (a^x+1)$ for some integer $n$, hence
%%\[
%%y  = \frac{a^{y}+1}{a+1}= n \frac{a^{x}+1}{a+1} = nx \mod (a+1)
%%\]
%%Thus, $y= nx + m(a+1)$ for some integer $m$. We could do the same argument $\mod (ka+1)$ instead of $\mod (a+1)$ for any $k\in\mathbb{N}$ (since $a = -1 \mod (ka+1)$, and so for each such $k$ there is an integer $q$ so that $y=nx+q(ka+1)$. Since $x,y,n>0$, we have that $y>q(ka+1)>qka$, so if we pick $k>y$, then the only way this inequality can hold is if $q=0$, and so $y=nx$. 
%%\end{solution}
%%
%%
%%\end{exercise}
%
%
%\begin{exercise} Let $a,b\in \mathbb{N}$. Show there are infinitely many $n\in \mathbb{N}$ so that $n^b+1 $ does not divide $a^n+1$. 
%
%\begin{solution}
%Suppose the contrary that $n^b+1$ does not divide $a^b+1$ for only finitely many $n$, so we have $n^b+1$ divides $ a^n+1$ for all $n$ large enough (specifically, $n>N$ where $N$ is the last integer where they don't divide).  
%
%Let $n=a^k$, then for $k$ large enough, $a^k>N$ and so
%\[
%(a^k)^b +1 =a^{kb}+1$ divides $ a^n+1=a^{a^{k}}+1.
%\]
%Thus, by the previous problem, $kb$ divides $a^k$. Hence, if $k$ is any integer coprime to $a$.
%
%
%\end{solution}
%



%\end{exercise}



\begin{exercise}
Prove that $42$ divides $n^7-n$ for all $n\in\mathbb{N}$. 
\begin{solution}
Since $42 = 2\cdot 3 \cdot 7$, we just need to verify each of these factors divides $n^7-n$ for each $n\in\mathbb{N}$.
\begin{itemize}
\item Since $n^7-n=n^6(n-1)$, one of $n$ and $n-1$ must be even, so one of $n^6$ and $n$ must be even, so $2$ divides $n^{7}-n$.
\item $n^7-n = n^4\cdot (n-1)n\cdot (n-1)$, and one of $n-1,n,n+1$ must be divisible by 3. 
\item Finally, by FLT,
\[
n^7-n = n-n=0 \mod 7.
\]
\end{itemize}
\end{solution}
\end{exercise}

\begin{exercise}
If $7$ does not divide $a$, then $7$ divides $a^3-1$ or $7$ divides $a^3+1$. 
\begin{solution}

Note that since $7$ does not divide $a$, FLT implies
\[
(a^{3}-1)(a^3+1) = a^6-1 = 1-1=0 \mod 7
\]
and so $7$ divides $(a^{3}-1)(a^3+1) $, and since $7$ is prime, it must divide one of the factors. 
\end{solution}
\end{exercise}

%
%\begin{exercise}
%If $p$ is prime, what is $(p-1)!\mod p$? (Hint: Recall that $ax = b\mod m$ has a solution if and only if $\gcd(a,m)$ divides $b$.)
%\begin{solution}
%By using the hint, we know that for each $q\in \{1,2,...,p-1\}$, there is $r\in \{1,2,...,p-1\}$ so that $qr = 1\mod p$. For $q=1$, this is just $r=1$. If $q\neq 1$ and $r=q$, then $q^2 = 1\mod p$ and so $p$ divides $q^2-1=(q-1)(q+1)$, thus $q$ is either $p-1$ or $p+1 = 1$. Hence, for $q\in \{2,3,...,p-2\}$, the inverse $r$ of $q$ is another number. Thus, all these numbers can be paired up and they cancel each other out in the product $(p-1)!$, so that $(p-1)!=p-1 = 01\mod p$. 
%\end{solution}
%\end{exercise}
%





%Interesting open problems:
%
%
%\begin{question}[Bocard's problem]
%Are there infinitely many $n\in\mathbb{N}$ so that $n!+1$ is a square? Only 3 integers are known: $4!+1=5^2$, $5!+1=11^2$, and $7!=71^2$. 
%\end{question}






%----------------------------------------------------------------------------------------
%	CHAPTER 3
%----------------------------------------------------------------------------------------

%\chapterimage{ima2} % Chapter heading image


%----------------

%----------------------------------------------------------------------------------------
%	BIBLIOGRAPHY
%----------------------------------------------------------------------------------------
%
%\chapter*{Bibliografía}
%\addcontentsline{toc}{chapter}{\textcolor{ocre}{Bibliografía}}
%\section*{Books}
%\addcontentsline{toc}{section}{Books}
%\printbibliography[heading=bibempty,type=book]
%
%\begin{itemize}
%	\item GREENE, W.H. (2003) “Econometric Analysis”5ª edición. Prentice Hall N.J. Capítulo 21
%\\\\
%    \item WOOLDRIDGE, J.M. (2010) “Introducción a la Econometría: Un Enfoque Moderno". 4ª edición. Cengage Learning. Capítulo 17
%
%\end{itemize}


%----------------------------------------------------------------------------------------
%	INDEX
%----------------------------------------------------------------------------------------

\cleardoublepage
\phantomsection
\setlength{\columnsep}{0.75cm}
\addcontentsline{toc}{chapter}{\textcolor{ocre}{Índice Alfabético}}
\printindex

%----------------------------------------------------------------------------------------

\end{document}


